\section{GODProcessors - Módulo de Processadores}
\begin{center}
Thiago Dias: tdsimao [at] gmail.com
\end{center}


O módulo \textbf{GODProcessors} tem a responsabilidade de realizar o processamento sobre
coleções de dados.

Possui duas classes principais, a \texttt{PCSStatisticsCalculator} que realiza os principais
cálculos estatísticos e a \texttt{PCSTagger} que realiza classificação de texto através de
métodos de recuperação de informação. Possui ainda duas classes auxiliares, a
\texttt{PCSPreprocessor} que realiza tratamento de strings e a \texttt{PCSGODTagger} uma fachada
para objetos \texttt{GODData}.

\subsection{PCSStatisticsCalculator} 

Esta classe calcula medidas estatísticas de tendencia central e dispersão para um determinada
coleção. Possui ainda um método para contagem de palavras.


\subsubsection{Métodos}
Todos os cálculos dessa classe podem ser realizados de duas formas, com ou sem bloco. A chamada
sem bloco realiza os cálculos sobre o valores absolutos da coleção, enquanto a chamada com
bloco executa os cálculos sobre o resultado da execução do bloco nos elementos da coleção.

Os cálculos que a classe realiza são:
\begin{description}
    \item[average] média 
    \item[median] mediana
    \item[std] desvio padrão
    \item[var] variância
\end{description}

Além dos cálculos estatísticos, possui um método para contagem de palavras

\begin{description}
    \item[countIn: aString ocurrencesOf: aWord] conta o número de ocorrências da palavra
                   \texttt{aWord} na \textit{string} \texttt{aString}.
\end{description}

\subsubsection{Exemplos}
Para utilizar os métodos sobre uma coleção de números.
    \begin{verbatim}
    pcs := PCSStatisticsCalculator new.
    pcs average: aCollection.
    pcs median: aCollection.
    pcs std: aCollection.
    pcs var: aCollection.
    \end{verbatim}

Pode-se utilizar os métodos sobre os atributos de coleção uma coleção de objetos, passando um
bloco como parâmetro.
    \begin{verbatim}
    pcs := PCSStatisticsCalculator new.
    pcs average: aCollection key: [ :x | x width].
    pcs median: aCollection key: [ :x | x width].
    pcs std: aCollection key: [ :x | x width].
    pcs var: aCollection key: [ :x | x width].
    \end{verbatim}


    
\subsection{PCSTagger}

Essa classe utiliza métodos de treinamento não supervisionado, que recebe uma coleção de
\textit{strings} para treinamento e em seguida é capaz de retornar os objetos mais relevantes
de uma nova \textit{string} em relação a toda a coleção.

Essa classe usa a medida \textit{tf\_idf} para avaliar a relevância de cada termo. Detalhes sobre
essa técnica podem ser encontrados no livro Introduction to Information Retrieval
\footnote{http://nlp.stanford.edu/IR-book/}

\subsubsection{Variáveis de Instância}
    
\begin{description}
    \item[dictIdf] Dictionary<idf> -- um dicionário de \textit{idf}(raridade do termo na
                   coleção)
    \item[maxTf] float $[0..1]$ -- frequência máxima dos termos da string que será considerada
                 pelo algoritmo.
    \item[minTf] float $[0..1]$ -- frequência mínima dos termos da string que será considerada
                 pelo algoritmo.
    \item[minRelevance] float $[0..1]$ --  define o menor \textit{tf\_idf} (relevância) a ser
                        considerada
    
\end{description}


\subsubsection{Exemplos}
Para utilizar os métodos sobre uma coleção de \textit{strings}.
    \begin{verbatim}
    pcsTagger = PCSTagger new.
    pcsTagger createDictIdf: aStringCollection.
    bag := Bag new.
    bag := pcsTagger getMoreRelevantsOf: aString.
    \end{verbatim}
    
\subsection{PCSGODTagger}
Essa classe é uma fachada da classe \texttt{PCSTagger} para objetos \texttt{GODData}. A classe
\texttt{PCSGODTaggerExample} mostra um exemplo completo de uso dessa classe.

É importante notar que é possível alterar o \texttt{PCSTagger} segundo necessário.

\subsubsection{Variáveis de Instância}
\begin{description}
 \item  [tagger] PCSTagger 
\end{description}

\subsubsection{Exemplos}
    Para adicionar \textit{tags} a um objeto \texttt{GODData}.
    \begin{verbatim}
    pcsTagger := PCSGODTagger new.
    pcsTagger training: aGODDataCollection.
    pcsTagger addTagsTo: aGODData.
    \end{verbatim}
    
    Para adicionar \textit{tags} a todos os elementos de uma coleção de \texttt{GODData} é
    possível chamar o método \texttt{tagCollection}.
    \begin{verbatim}
    pcsTagger := PCSGODTagger new.
    pcsTagger tagCollection: aGODDataCollection.
    \end{verbatim}


\subsection{PCSPreprocessor}
Essa classe realiza operações comuns de pré-processamento de \textit{strings}.


\subsubsection{Variáveis de Instância}
\begin{description}
 \item  [fileType] String $\in$ \{`TXT',`HTML'\} -- 
 \item  [puctuation] String --
 \item  [stopWords] Set<string>--
 
\end{description}


\subsubsection{Métodos}
As operações que essa classe realiza são:
\begin{description}
    \item[removeStopwords] média 
    \item[median] mediana
    \item[std] desvio padrão
    \item[var] variância
\end{description}

