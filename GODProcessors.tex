

\section{Processadores (GODProcessors)}

O módulo \textbf{GODProcessors} realiza o processamento de coleções de dados.
Possui duas classes principais, a \texttt{PCSStatisticsCalculator} que realiza os principais
cálculos estatísticos e a \texttt{PCSTagger} que realiza classificação de texto através de
um método estatístico de recuperação de informação conhecido como tf-idf. Possui ainda duas classes auxiliares, a 
\texttt{PCSPreprocessor} que realiza tratamento de strings e a \texttt{PCSGODTagger} uma fachada
para objetos \texttt{GODData}.

Em caso de dúvidas, entre em contato com \emph{Thiago: tdsimao [at] ime.usp.br}.


\subsection{PCSStatisticsCalculator} 

Esta classe calcula medidas estatísticas de tendência central e dispersão para um determinada
coleção. Possui ainda um método para contagem de palavras.


\subsubsection{Métodos}
Todos os cálculos dessa classe podem ser realizados de duas formas: com ou sem bloco. A chamada
sem bloco realiza os cálculos sobre o valores absolutos da coleção, enquanto a chamada com
bloco executa os cálculos sobre o resultado da execução do bloco nos elementos da coleção.

Os cálculos que a classe realiza são:
\begin{itemize}
    \item \textbf{average}: média 
    \item \textbf{median}: mediana
    \item \textbf{std}: desvio padrão
    \item \textbf{var}: variância
\end{itemize}

Além dos cálculos estatísticos, possui um método para contagem de palavras
\begin{itemize}
    \item \textbf{countIn: aString ocurrencesOf: aWord}: conta o número de ocorrências da palavra
                   \textbf{aWord} na \textit{string} \textbf{aString}.
\end{itemize}

\subsubsection{Exemplos}
Para utilizar os métodos sobre uma coleção de números.
    \begin{verbatim}
    pcs := PCSStatisticsCalculator new.
    pcs average: aCollection.
    pcs median: aCollection.
    pcs std: aCollection.
    pcs var: aCollection.
    \end{verbatim}

Pode-se utilizar os métodos sobre os atributos de coleção uma coleção de objetos, passando um
bloco como parâmetro.
    \begin{verbatim}
    pcs := PCSStatisticsCalculator new.
    pcs average: aCollection key: [ :x | x width].
    pcs median: aCollection key: [ :x | x width].
    pcs std: aCollection key: [ :x | x width].
    pcs var: aCollection key: [ :x | x width].
    \end{verbatim}


    
\subsection{PCSTagger}

Essa classe utiliza uma técnica estatistica de recuperação de informaçao conhecida como \textbf{tf-idf} 
(term frequency-inverse document frequency). Esse método de treinamento não supervisionado recebe uma coleção de
\textit{strings} para treinamento e retorna os termos mais relevantes em relação a toda a coleção. 
Detalhes sobre essa técnica podem ser encontrados no livro Introduction to Information Retrieval
\footnote{http://nlp.stanford.edu/IR-book/}

\subsubsection{Variáveis de Instância}
    
\begin{itemize}
    \item \textbf{dictIdf}: Dictionary<idf> -- um dicionário de \textit{idf}(raridade do termo na
                   coleção)
    \item \textbf{maxTf}: float $[0..1]$ -- frequência máxima dos termos da string a ser considerada.
    \item \textbf{minTf}: float $[0..1]$ -- frequência mínima dos termos da string a ser considerada.
    \item \textbf{minRelevance}: float $[0..1]$ --  define o menor \textit{tf\_idf} (relevância) a ser
                        considerada.
\end{itemize}

\subsubsection{Métodos}

\begin{itemize}
    \item \textbf{createDictIdf: aStringCollection}: cria o \texttt{dictIdf}.
    \item \textbf{getMoreRelevantsOf: aString}: recupera os termos mais relevantes de \texttt{aString}.
\end{itemize}


\subsubsection{Exemplos}
Para utilizar os métodos sobre uma coleção de \textit{strings}.
    \begin{verbatim}
    pcsTagger := PCSTagger new.
    pcsTagger createDictIdf: aStringCollection.
    bag := Bag new.
    bag := pcsTagger getMoreRelevantsOf: aString.
    \end{verbatim}
    
\subsection{PCSGODTagger}
Essa classe é uma fachada da classe \texttt{PCSTagger} para objetos \texttt{GODData}. A classe
\texttt{PCSGODTaggerExample} mostra um exemplo completo de uso dessa classe.

É importante notar que é possível alterar o \texttt{PCSTagger} segundo necessário.

\subsubsection{Variáveis de Instância}
\begin{itemize}
 \item \textbf{tagger}: PCSTagger -- objeto da classe \texttt{PCSTagger} que treina com uma coleção de 
                              strings recuperadas de uma coleção de \texttt{GODData} e recupera
                              os principais termos dessa \textit{String}.
\end{itemize}


\subsubsection{Métodos}

\begin{itemize}
    \item \textbf{training: aGODDataCollection}: realiza o treinamento com o atributo \texttt{content}                                  
                                          dos objetos de \texttt{aGODDataCollection}.
    \item \textbf{addTagsTo: aGODData}: define tags para o atributo \texttt{tags} de \texttt{aGODData}
    \item \textbf{tagCollection: aGODDataCollection}: atalho para realizar o treinamento sobre uma
                                         coleção e em seguida adicionar tags a todos os objetos
                                         da mesma.
 \end{itemize}


\subsubsection{Exemplos}
    Para adicionar \textit{tags} a um objeto \texttt{GODData}.
    \begin{verbatim}
    pcsTagger := PCSGODTagger new.
    pcsTagger training: aGODDataCollection.
    pcsTagger addTagsTo: aGODData.
    \end{verbatim}
    
    Para adicionar \textit{tags} a todos os elementos de uma coleção de \texttt{GODData} é
    possível chamar o método \texttt{tagCollection}.
    \begin{verbatim}
    pcsTagger := PCSGODTagger new.
    pcsTagger tagCollection: aGODDataCollection.
    \end{verbatim}


\subsection{PCSPreprocessor}
Essa classe realiza operações comuns de pré-processamento de \textit{strings}. 
Para utilizar essa classe você deve configurá-la definindo os valores de suas variáveis 
\ref{pre-variaveis} através de seus métodos de acesso e em seguida usar o método
\texttt{preprocess: aString} para recuperar uma coleção de \textit{tokens} da string 
\textit{aString}


\subsubsection{Variáveis de Instância} \label{pre-variaveis}
\begin{itemize}
 \item  \textbf{fileType}: String $\in$ \{`TXT',`HTML'\} -- define o tipo de arquivo de entrada.
 \item  \textbf{puctuation}: String -- string com todos os caracteres de pontuação que  serão removidos,
                     ex: `,.!?'.
 \item  \textbf{stopWords}: Set<string>-- um conjunto com palavras comuns da língua que serão ignoradas.
\end{itemize}


\subsubsection{Métodos} 
Além do método principal \texttt{preprocess:} 
As operações que essa classe realiza são:
\begin{itemize}
    \item \textbf{treatType}: realiza tratamento relativo ao tipo de arquivo, exemplo para arquivos HTML
                     remove as \textit{tags} HTML.
    \item \textbf{removeStopwords}: remove as palavras comuns de uma \textit{string}.
    \item \textbf{tokenizer}: quebra a \textit{string} em uma coleção de \textit{tokens}.
\end{itemize}

\subsubsection{Exemplos}
    \begin{verbatim}
    preprocessor := PCSPreprocessor new.
    aBag := preprocessor preprocess: aString.
    \end{verbatim}  

