{

\def\godss{\textsc{GodSpreadsheet}}

\def\classe#1{\textsc{#1}}
\def\code#1{\texttt{#1}}


\lstset{%
 basicstyle=\small\ttfamily\color{black!85},
 breaklines = true,
 keywordstyle=\bfseries\color{black},
 emphstyle=\color{blue},
 columns=fullflexible,
 showstringspaces=false,
 numbers=left
}%

\lstdefinelanguage{GODSpreadsheet}
  {keywords={SSSpreadsheetData, SSSheet, SSRow, SSCell},
  morestring=[b]{'},
  stringstyle=\color{purple},
  alsoletter={:}, 
  emph={new}
  }


% Python environment
\lstnewenvironment{godSS}[1][]
{
\lstset{language=GODSpreadsheet,
    #1}
}
{}


\section{Spreadsheet}

O grupo de Spreadsheet preocupou-se, principalmente, em prover
funcionalidades de entrada e saída de planilhas, mas também implementou
alguns métodos que auxiliem a manipulação das mesmas.

Em caso de dúvida, entrar em contato com Diogo Haruki (\texttt{haruki} \textit{arroba} \texttt{ime} \textit{ponto} \texttt{usp} \textit{ponto} \texttt{br})

\subsection{Estrutura}
Nesta seção, explicaremos a organização das principais classes do
\godss. Quase todos os objetos das classes apresentadas
aqui possuem uma referência para seu pai (\code{parent})\footnote{Com
exceção apenas da SSSpreadsheetData, que não tem pai}.

\begin{tabular}{r p{0.6\textwidth}}
\classe{SSCell} &  Classe que corresponde a uma célula da planilha.
Assumimos que a célula tem sempre um conteúdo texto (\code{string}), e
consideramos como célula vazia aquelas que possuem \code{value} $=$
\texttt{nil}.\\

\classe{SSRow} & Classe que corresponde a uma linha da tabela.
Cada linha contém uma coleção de células, que foi mantida encapsulada.
Dessa forma, podemos garantir a coerencia entre os elementos dentro dessa
lista e a referência dos pais desses elementos.\\

\classe{SSSheet} & Classe que corresponde a uma tabela. Uma tabela
é formada por uma coleção de linhas (\classe{SSRow}), deve possuir um 
nome e, assim como em \classe{SSRow}, a coleção de linhas é encapsulada
para garantir a coesão.\\

\classe{SSSpreadsheetData} & Classe que corresponde a uma coleção de
tabelas. Cada tabela é indexida numericamente, mas há a possibilidade
de acessá-las a partir de seu nome. A coleção de tabelas
(\classe{SSSheet}) também é encapsulada, para manter a coesão entre os
diversos objetos.
\end{tabular}


\subsubsection{Indexação em todos os níveis}
Uma breve explicação dos modos de acesso aos filhos de cada classe.

Em \classe{SSSpreadsheetData}, as \classe{SSSheets} estão indexadas numericamente, mas podem ser acessadas por seu nome.

Em \classe{SSSheet}, as \classe{SSRows} estão indexadas numericamente, e seu acesso se dá unicamente pelo número da linha.

Em \classe{SSRow}, as \classe{SSCells} estão indexadas numericamente, mas podem ser acessadas pelo nome da coluna (\code{'A', 'B', \dots, 'Z, 'AA', ...}).


\subsection{How to do}
Nesta seção, vamos tentar mostrar o que pode ser feito\footnote{coisas
básicas que podem ser feitas} com o módulo e um modo de fazê-lo.

\subsubsection{Abrir arquivos}
A abertura de arquivos pelo módulo \godss~ é algo bem simples. Basta
utilizar o método de classe \code{fromFile:} da classe \classe{SSSpreadsheetData}.

Os formatos de arquivo suportados atualmente são: \code{csv, ods, xlsx}.

\begin{godSS}[moreemph={fromFile:,}]
csvSSData := SSSpreadsheetData fromFile: 'path/to/file.csv'.
odsSSData := SSSpreadsheetData fromFile: 'path/to/file.ods'.
xlsxSSData := SSSpreadsheetData fromFile: 'path/to/file.xlsx'.
\end{godSS}

\subsubsection{Salvar arquivos}
Salvar arquivos também é tarefa fácil para o módulo \godss. Para isto,
basta usar o método de instância \code{toFile:} de um objeto de
\classe{SSSpreadsheetData}. Dessa forma, suportamos saída para arquivo
\code{ods}.

\begin{godSS}[moreemph={toFile:,}]
ssData toFile: 'path/to/file.ods'.
\end{godSS}

Também podemos exportar uma \classe{SSSheet} para \code{csv}. Isso se deve
ao fato de \code{csv} guardar uma planilha simples, e não uma coleção de
planilhas como os outros formatos. Para isso, podemos usar o método de
instância \code{exportToCSV:} de um objeto de \classe{SSSheet}.

\begin{godSS}[moreemph={getSheet:,exportToCSV:}]
sheet := ssData getSheet: 1.
sheet exportToCSV: 'path/to/file.csv'.
\end{godSS}

\subsubsection{Criar uma SSSpreadsheetData e populá-la}
Caso seja necessário, também é possível criar uma \classe{SSSpreadsheetData} 
e populá-la manualmente de forma bem fácil. Primeiramente, para criar uma planilha (um objeto \classe{SSSheet}):

\begin{godSS}[moreemph={new,createSheetWithName:,createCellAtRow:,atColumn:,atColumnIndex:,withValue:,createRow}]
ssdata := SSSpreadsheetData new.
sheet := ssdata createSheetWithName: 'sheet'.
\end{godSS}

Depois de criada podemos popular a planilha de vários jeitos. O mais simples é criando célula por célula
diretamente:
\begin{godSS}[moreemph={createCellAtRow:,atColumn:,atColumnIndex:,withValue:}]
cell1 := sheet createCellAtRow: 1 atColumnIndex: 1 withValue: 'valor'.
cell2 := sheet createCellAtRow: 1 atColumn: 'B' withValue: 'valor2'.
\end{godSS}

Mas também podemos criar uma \classe{SSRow}, e definir as célular pela linha:
\begin{godSS}[moreemph={createRow,createRowAtIndex:,createCellAtColumn:,createCellAtColumnIndex:,withValue:}]
row := sheet createRow.
row := sheet createRowAtIndex: 3.
row createCellAtColumn: 'A' withValue: 'valor'.
row createCellAtColumnIndex: 2 withValue: 'valor2'.
\end{godSS}

Ou também podemos popular uma \classe{SSSheet} diretamente com valores de um tabela
smalltalk, que é uma coleção de \textit{linhas}, onde cada \textit{linha} é uma coleção
com os valores das suas células. Também é possível popular a planilha com um dicionário
smalltalk - nesse caso, cada par \code{(chave, valor)} do dicionário vira uma linha
da planilha, com as chaves na coluna ``A'' e os valores na coluna ``B''.
\begin{godSS}[moreemph={fillFromCollection:,fillFromDictionary:}]
sheet fillFromCollection: collection.
sheet fillFromDictionary: dictionary.
\end{godSS}

\subsubsection{Executar alguma ação em todas as células da planilha}
As classes \classe{SSSpreadsheetData}, \classe{SSSheet} e \classe{SSRow} contém
coleções das classes seguintes, mas cada uma encapsula essa coleção para manter coesão.

Porém, além das classes permitirem acesso de algum elemento específico dessas coleções
por mensagens específicas de cada classe, elas também permitem que o usuário
itere pelas coleções usando a mensagem \code{do:} do objeto.

Partindo de um \classe{SSSpreadsheetData} é trivial iterar por todas células de uma planilha:
\begin{godSS}[moreemph={getSheet:,do:,show:}]
sheet := ssData getSheet: 1.
sheet do: [ :row |
  row do: [ :cell |
    Transcript show: (cell value).
    ].
  ].
\end{godSS}


\subsubsection{Diferentes modos de acessar uma célula}
Temos acesso às células em diferentes níveis. Tanto em \classe{SSSheet} como em \classe{SSRow}

\begin{godSS}[moreemph={getCellAtRow:, atColumn:, atColumnIndex:, getCell:, getCellAtIndex:, getRow:}]
cell1 := sheet getCellAtRow: 1 atColumn: 'A'
cell2 := sheet getCellAtRow: 1 atColumnIndex: 3

row := sheet getRow: 6
cell3 := row getCell: 'C'
cell4 := row getCellAtIndex: 2
\end{godSS}

\subsection{Próximos Passos}

\begin{itemize}
\item Saída para xlsx - É necessário implementar um método que produz um arquivo de saída xlsx como já é feito com o formato ods
\item Suporte a formatação - Até o momento o módulo utiliza a formatação padrão de dados em uma planilha.\\ É necessário implementar métodos para configurar fonte, cor e tamanho dos caracteres, largura e altura da célula, etc.
\item Tratamento de dados de diferentes tipos - O módulo trata todos os dados como string. Porém, é necessário fazer com que os métodos funcionem para dados em outros tipos como int, double, date. Isso é importante para que as funções de planilha funcionem para estes dados, por exemplo.
\item Busca - O módulo não possui suporte a busca global de dados na planilha
\item Gráficos - O módulo também não produz gráficos baseados nas informações presentes nele
\item Fórmulas - É necessário oferecer funcionalidades que permitem o uso de fórmulas e funções matemáticas
\end{itemize}

}