\section{GODGraphGenerator}
	Esta seção apresenta o documento de projeto do módulo do gerador de gráficos e está seccionada em tópicos de
	interesse.\\

	Em caso de dúvidas, entre em contato com \emph{Renan Teruo Carneiro: renanteruoc@gmail.com}.

	\subsection{Principais Características}
		O sistema em questão trata-se de um gerador de gráficos em línguagem Squeak Smalltalk e apresenta as
		seguintes características:
		\begin{itemize}
			\item Necessidade de dados préviamente processados para gerar o gráfico.
			\item Flexibilidade com relação ao output.
			\item Expansibilidade com relação aos tipos de gráficos.
			\item O gerador funciona independentemente dos outros módulos do GOD.
			\item Não necessita de bibliotecas externas.
		\end{itemize}

	\subsection{Funcionalidades}
		As seguintes funcionalidades foram implementadas:
		\begin{itemize}
			\item Gerar gráficos de barra.
			\item Gerar gráficos de linha.
			\item Exportar o gráfico como imagem PNG.
			\item Abrir o gráfico como um objeto do tipo Morph no Squeak world.
			\item Mudar a cor do conteúdo do gráfico.
		\end{itemize}
		Devido a arquitetura escolhida, é possível expandir os tipos de gráficos gerados e as opções de
		output de maneira simples e independente das funcionalidades já implementadas.

	\subsection{Input}
		O input é feito preenchendo uma espécie de formulário com os seguintes campos:\\
		\makebox[\linewidth]{
			\begin{tabular}{l l}
				Label horizontal: &   String \emph{opcional}\\
				Label vertical:   &   String \emph{opcional}\\
				Tabela de Dados:  &   Dictionary \emph{mandatório}
			\end{tabular}
		}\\

		O campo Tabela de Dados deve ser um Dictionary no formato: \emph{palavra -> número}. E um ponto a
		ser resaltado é que apesar de possível, não é recomendado a inclusão de novas palavras no após o
		gráfico ter sido gerado.\\

		Apesar dos campos de labels serem opcionais, é recomendado o seu uso, pois aumenta o comprimento
		dos eixos e gera mais espaço para o desenho do gráfico.\\

		É possível criar customizações, de modo que a Tabela de Dados seja de outro tipo. Para isso criamos
		uma subclasse de \emph{GGGraph}, como se fosse um tipo diferente de gráfico. Dizemos o tipo do
		gráfico com a mensagem \emph{delegate}, e "arrumamos" a entrada dentro do generateGraph dessa nova
		subclasse.


	\subsection{Output}
		O output é uma instância do tipo do gráfico, e todos os gráficos são subclasses de Morph (mais
		detalhes na seção 1.5).\\

		Esse objeto pode ser exportado para um arquivo de imagem PNG, que será salvo em
		\emph{\$squeak\_dir/Contents/Resources}, ou aberto no Squeak world.\\

		Casso seja necessário mudar a extenssão do arquivo de imagem, o output pode receber qualquer
		mensagem destinada a objetos do tipo Morph.

	\subsection{Arquitetura}
		\emph{GGGraph} é uma classe abstrata que representa um gráfico, ela por sua vez é um Morph do Squeak.
		As duas subclasses \emph{GGBarGraph} e \emph{GGLineGraph} representam o tipo do gráfico. E esta
		terceira subclasse \emph{GGGraphOrderedCollectionDecorator} é uma customização de input.\\

		A única especialização dos gráficos de barra e linha é a forma como eles são desenhados, todo o
		resto da estrutura é comum para todos os gráficos.\\

		Abaixo o digrama de classes descreve tal arquitetura:\\

		\makebox[\linewidth]{
			\includegraphics[scale=0.53]{gg_class_diagram}
		}

	\subsection{Exemplo de uso}
		O código a seguir exemplifica a rotina para gerar o gráfico:\\

		\makebox[\linewidth]{
			\includegraphics[scale=0.78]{gg_code_gbar}
			\includegraphics[scale=0.78]{gg_gbar}
		}

		Rotina:
		\begin{enumerate}
			\item Instanciar o gráfico que se quer gerar, \emph{GGBarGraph} ou \emph{GGLineGraph}
			\item Adicionar os labels utilizando as mensagens \emph{horizonlLabel} e \emph{verticalLabel}
			\item Instanciar um Dictionary e adicionar as palavras e valores
			\item Gerar o gráfico com a mensagem \emph{generateGraph}
			\item Exportar ou abrir com \emph{exportToFile} e \emph{openInWorld}
		\end{enumerate}

		OBS: Uma restrição do gráfico de linhas é que as chaves no Dictionary precisam ser números não
		negativos. E recomendamos que sejam os itervalos $0 \bmod(100)$.\\

		\makebox[\linewidth]{
			\includegraphics[scale=1.10]{gg_code_gline}
			\includegraphics[scale=0.88]{gg_gline}
		}
