\section{Gráficos (GODGraphGenerator)}
	O módulo do gerador de gráficos recebe um Dictionary ou uma OrderedCollection e devolve um gráfico correspondente. O gráfico pode ser um objeto Morph ou uma imagem PNG. Quanto ao tipo de gráfico, pode ser gerado um gráfico de barras ou de linhas. Em caso de dúvidas, entre em contato com \emph{Renan Teruo Carneiro: renanteruoc@gmail.com}.
	
	\subsection{Classes}
		\begin{itemize}
		    \item \emph{GGGraph}: classe abstrata que representa um gráfico, ela por sua vez é um Morph do Squeak. Possui métodos para definir os nomes dos eixos do gráfico. Também é possível definir as dimensões do gráfico a ser gerado. O gráfico pode ser gerado no World do Squeak ou salvo como um arquivo PNG.
		    \item \emph{GGBarGraph} e \emph{GGLineGraph}: subclasses de GGGraph que representam o tipo do gráfico. Recebem como entrada um Dictionary.
		    \item \emph{GGGraphOrderedCollectionDecorator} esta subclasse de GGGraph recebe uma OrderedCollection\\
		\end{itemize}
		A única especialização dos gráficos de barra e linha é a forma como eles são desenhados, todo o
		resto da estrutura é comum para todos os gráficos.\\
	
	\subsection{Entrada}
		
		O Dictionary com os dados de entrada precisa ter o formato: \emph{palavra -> número}. Não é recomendado a inclusão de novas palavras no após o
		gráfico ter sido gerado.\\

		Apesar dos campos de labels serem opcionais, é recomendado o seu uso, pois aumenta o comprimento dos eixos e gera mais espaço para o desenho do gráfico.\\

		Outro tipo de dado de entrada pode ser uma OrderedCollection composta por duas OrderedCollections com o seguinte formato: \emph{OrderedCollection(palavra),OrderedCollection(valor)}. Esse tipo de coleção foi definido para tornar possível ordenar os dados de acordo com a necessidade do gráfico a ser gerado. O tipo do gráfico é definido pela mensagem \emph{delegate}, e "arrumamos" a entrada dentro do generateGraph dessa classe (GGGraphOrderedCollectionDecorator).

	\subsection{Saída}
	
		A saída é uma instância do tipo do gráfico, e todos os gráficos são subclasses de Morph (mais detalhes na seção 1.5).\\

		Esse objeto pode ser exportado para um arquivo de imagem PNG, que será salvo em\\\emph{\$squeak\_dir/Contents/Resources}, ou aberto no Squeak world.\\

		Casso seja necessário mudar a extenssão do arquivo de imagem, a saída pode receber qualquer mensagem destinada a objetos do tipo Morph.

	
	\subsection{Exemplo de uso}
		O código a seguir exemplifica a rotina para gerar o gráfico:\\

		\makebox[\linewidth]{
			\includegraphics[scale=0.78]{gg_code_gbar}
			\includegraphics[scale=0.78]{gg_gbar}
		}

		Rotina:
		\begin{enumerate}
			\item Instanciar o gráfico que se quer gerar, \emph{GGBarGraph} ou \emph{GGLineGraph}
			\item Adicionar os labels utilizando as mensagens \emph{horizonlLabel} e \emph{verticalLabel}
			\item Instanciar um Dictionary e adicionar as palavras e valores
			\item Gerar o gráfico com a mensagem \emph{generateGraph}
			\item Exportar ou abrir com \emph{exportToFile} e \emph{openInWorld}
		\end{enumerate}

		OBS: Uma restrição do gráfico de linhas é que as chaves no Dictionary precisam ser números não negativos. Recomendamos estejam no intervalo de $0 \bmod(100)$.\\

		\makebox[\linewidth]{
			\includegraphics[scale=1.10]{gg_code_gline}
			\includegraphics[scale=0.88]{gg_gline}
		}
