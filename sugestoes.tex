\section{Sugestões para requisitos futuros}
\subsection{E-mails}
\begin{enumerate}
\item Enviar arquivos em anexo
\item Enviar emails em html
\end{enumerate}

\subsection{Redes sociais}
\begin{enumerate}
\item Utilizar dados de um usuário de verdade para retornar mais informações do Facebook. No momento, utiliza-se um usuário de teste, o que limita os dados que o Facebook oferece.
\item Salvar no banco de dados, a cada semana, os resultados das buscas mais utilizadas no Twitter, pois ele só retorna tweets de até uma semana atrás nas buscas.
\item Utilizar paginação de resultados para contornar o limite de 100 tweets retornados em buscas do Twitter.
\end{enumerate}

\subsection{Filtros}
\begin{enumerate}
\item Implementar filtros por expressões regulares
\item Filtrar HTML por XPath
\item Filtrar por conteúdo, título, tags e origem ao mesmo tempo
\item Implementar a opção de filtrar os dados que já estão no banco de dados
\item Criar testes com mocks
\end{enumerate}

\subsection{Processadores}
\begin{enumerate}
\item Busca de documentos utilizando tf-idf
\item Utilizar tf-idf para analisar textos
\item Implementar stemming ou lematização para melhorar a geração de tags
\end{enumerate}

\subsection{Banco de dados}
\begin{enumerate}
\item Criação de backups
\item Controle de acesso dos usuários
\end{enumerate}

\subsection{Análise de sentimento de consumo}
\begin{enumerate}
\item Melhorar a análise utilizando relações de uma palavra com outra
\item Adicionar cores ao gráfico (verde para GOOD, amarelo para NEUTRAL e vermelho para BAD)
\item Adicionar cores à planilha
\item Mostrar o gráfico e a planilha na mesma busca
\item Melhorar o layout
\end{enumerate}

\subsection{Análise de sentimento político}
\begin{enumerate}
\item Mostrar planilha e gráfico na mesma busca
\item Utilizar cores para o sentimento da pesquisa (verde, amarelo e vermelho)
\item Adicionar técnicas alternativas de análise, que não utilizem dicionários
\item Criar opção de restringir a análise a um período de tempo (intervalo de tempo, ou primeiro turno das eleições de algum ano)
\item Adicionar autocomplete para candidatos e partidos
\end{enumerate}

\subsection{Agregação de informações acadêmicas}
\begin{enumerate}
\item Realização de busca fuzzy por journals (por exemplo, 'plos comp biology' dá 'plos computation biology')
\item Busca em mais fontes
\end{enumerate}

\subsection{Agregação de conferências}
\begin{enumerate}
\item Implementação de crawling incremental
\item Busca em outras fontes (como o Academic Search da Microsoft)
\item Adição de novas informações (local da conferência, transporte, histórico de clima, pontos turísticos e hotéis na região)
\item Adição de informações acadêmicas (estatísticas sobre os trabalhos apresentados em uma conferência)
\end{enumerate}