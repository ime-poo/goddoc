\section{Redes sociais (GODSocialNetIO)}
\label{sec-1}
\subsection{Team}
\label{sec-1-1}

  Eduardo Alexandre, Aline Borges, Leonardo Haddad, Thiago Araujo
\subsection{Description}
\label{sec-1-2}

  GODSocialNetIO allows others modules to communicate with the Facebook and Twitter servers. It does that by implementing both public APIs.
  The module is divided in two main parts: the fetchers classes and the GODData classes. The fetchers classes are responsable for the communication with the APIs.
  The GODData classes receive the JSON response from a request returned by a fetcher and transform it into subclasses of GODData.\\
  
  It's important to mention that both Facebook and Twitter APIs need a client id and password to work, this can be done creating a specific type of user in the developer section of both APIs.
  Last but not least, we have classes responsible for testing all methods within the fetchers and GODDatas, keep in mind that we don't have control of the Facebook and Twitter servers, so some test may fail some times simply because the server response is not what was expected.\\
  
  If you have questions, contact us: thd.araujo@gmail.com.

\subsection{Classes of Fetchers}
\label{sec-1-4}
\subsubsection{SNETFetcher}
\label{sec-1-4-1}
SNETFetcher is a abstract class that contains the common parts used by SNETFacebookFetcher and SNETTwitterFetcher. The methods are described below.
\begin{itemize}
\item \textbf{private} - This abstract method sends some HTTP request to some url, it will be specialized in SNETFacebookFetcher and SNETTwitterFetcher.
\item \textbf{initialize} - This is the default method to initialize the class.
\item \textbf{connection} - This abstract method connects to some server.
\end{itemize} % ends low level

\subsubsection{SNETFacebookFetcher}
\label{sec-1-4-2}
SNETFacebookFetcher is a class that controls all the access to the Facebook API.
It sends requests to the facebook server and returns the result as a SNETFacebookUsersData object, SNETFacebookPostsData, SNETFacebookPagesData, SNETFacebookPageDescriptionData, SNETFacebookGroupsData, SNETFacebookGroupDescriptionData and SNETFacebookEventsData.
The methods are described below.
\begin{itemize}
\item \textbf{sendRequest} - sends some HTTP request to Facebook API.
\item \textbf{byId: id} - is a helper to some methods that search in the Facebook API using some id.
\item \textbf{initialize} - This is the default method to initialize the class, it will initialize all the class variables.
\item \textbf{connection} - connects to Facebook API.
\item \textbf{events: event} - searches for events with the parameter name.
\item \textbf{events: event since: since} - searches for events with the parameter name since some date.
\item \textbf{events: event since: since until: until} - searches for events with the parameter name since some date until another date.
\item \textbf{groupDescription: groupId} - gets group information by id.
\item \textbf{groups: group} - searches for groups with the parameter name.
\item \textbf{pageDescription: pageId} - gets page information by id.
\item \textbf{pages: page} - searches for pages with the parameter name.
\item \textbf{posts: user} - gets posts of a user.
\item \textbf{posts: user since: since} - gets posts of a user since some date.
\item \textbf{posts: user since: since until: until} - gets posts of a user since some date until another date.
\item \textbf{users: user} - searches for users with the parameter name.
\end{itemize}

\subsubsection{SNETTwitterFetcher}
\label{sec-1-4-3}
SNETTwitterFetcher is a class that control all the access to the Twitter API.
It send requests to the twitter server and return the result as data class SNETTwitterTweetsData.
The methods are described below.
\begin{itemize}
\item \textbf{sendRequest: url} - sends some HTTP request to Twitter API.
\item \textbf{sendRequest: url method: method} - sends some HTTP request with either GET or POST method to Twitter API.
\item \textbf{initialize} - This is the default method to initialize the class, it will initialize all the class and instance variables.
\item \textbf{connection} - connects to Twitter API.
\item \textbf{posts: user} - gets posts of a user.
\item \textbf{posts: user maximum: max} - gets posts of a user specifying the maximum amount of posts.
\item \textbf{tweets: text} - gets tweets from text.
\item \textbf{tweets: text maximum: max} - gets tweets from text specifying the maximum amount of tweets.
\item \textbf{tweets: text since: date maximum: max} - gets tweets from text since date specifying the maximum amount of tweets.
\item \textbf{tweets: text since: initDate until: endDate maximum: max} - gets tweets from text since date until another date specifying the maximum amount of tweets.
\item \textbf{tweets: text until: date} - gets tweets from text until date.
\item \textbf{tweets: text until: date maximum: max} - gets tweets from text until date specifying the maximum amount of tweets.
\item \textbf{tweetsFrom: text user: user} - gets tweets from text from user.
\item \textbf{tweetsFrom: text user: user maximum: max} - gets tweets from text from user specifying the maximum amount of tweets.
\item \textbf{tweetsFrom: text user: user since: date} - gets tweets from text from user since date.
\item \textbf{tweetsFrom: text user: user since: date maximum: max} - gets tweets from text from user since date specifying the maximum amount of tweets.
\item \textbf{tweetsFrom: text user: user since: initDate until: endDate maximum: max} - gets tweets from text from user since date until another date specifying the maximum amount of tweets.
\item \textbf{tweetsFrom: text user: user until: date} - gets tweets from text from user until date.
\item \textbf{tweetsFrom: text user: user until: date maximum: max} - gets tweets from text from user until date specifying the maximum amount of tweets.
\item \textbf{language: newLanguage} - sets the language of the tweets.
\end{itemize}

\subsection{Classes of GODDatas}
\label{sec-1-5}
\subsubsection{SNETTwitterTweetsData}
\label{sec-1-5-1}
SNETTwitterTweetsData is a class that contains a list of SNETTwitterTweet. The methods are:
\begin{itemize}
\item \textbf{addTweet: json} - adds a new tweet from a json to the collection.
\item \textbf{initialize} - This is the default method to initialize the class.
\item \textbf{size} - shows the number of tweets in the collection.
\item \textbf{tweetAt: position} - gets a tweet at a position in the collection.
\item \textbf{tweets} - gets the list of tweets.
\end{itemize}

\subsubsection{SNETTwitterTweet}
\label{sec-1-5-2}
SNETTwitterTweet is a class that contains a Twitter tweet. There are getters and setters for the following variables:
\begin{itemize}
\item \textbf{favorited: number} - number of favorited of tweet.
\item \textbf{id: idNumber} - tweet id.
\item \textbf{message: text} - tweet message.
\item \textbf{retweets: number} - number of retweets.
\end{itemize}

\subsubsection{SNETFacebookPostsData}
\label{sec-1-5-3}
SNETFacebookPostsData is a class that contains a list of SNETFacebookPost. The methods are:
\begin{itemize}
\item \textbf{addPost: json} - adds a new post from a json to the collection.
\item \textbf{initialize} - This is the default method to initialize the class.
\item \textbf{size} - shows how much posts the collection contains.
\item \textbf{postAt: position} - gets a post at a position in the collection.
\item \textbf{posts} - gets the list of posts.
\end{itemize}

\subsubsection{SNETFacebookPost}
\label{sec-1-5-4}
SNETFacebookPost is a class that contains a Facebook post. There are getters and setters for the following variables:
\begin{itemize}
\item \textbf{addComment: json} - adds a new comment to the collection.
\item \textbf{caption: text} - post caption.
\item \textbf{description: text} - post description.
\item \textbf{likes: number} - number of likes of post.
\item \textbf{message: text} - post message.
\item \textbf{picture: url} - picture url from post.
\item \textbf{shares: number} - post shares.
\item \textbf{type: postType} - type of post.
\end{itemize}


\subsubsection{SNETFacebookComment}
\label{sec-1-5-5}
SNETFacebookComment is a class that contains a Facebook comment. There are getters and setters for the following variables:
\begin{itemize}
\item \textbf{id: idNumber} - comment id.
\item \textbf{likes: number} - number of likes of a comment.
\item \textbf{message: text} - comment message.
\end{itemize}


\subsubsection{SNETFacebookPagesData}
\label{sec-1-5-6}
SNETFacebookPagesData is a class that contains a list of SNETFacebookPage. The methods are:
\begin{itemize}
\item \textbf{addPage: json} - adds a new page from a json to the collection.
\item \textbf{initialize} - This is the default method to initialize the class.
\item \textbf{size} - shows the number of pages in the collection.
\item \textbf{pageAt: position} - gets a page at a position in the collection.
\item \textbf{pages} - gets the list of pages.
\end{itemize}

\subsubsection{SNETFacebookPage}
\label{sec-1-5-7}
SNETFacebookPage is a class that contains a Facebook page. There are getters and setters for the following variables:
\begin{itemize}
\item \textbf{category: pageCategory} - page category.
\item \textbf{id: idNumber} - page id.
\item \textbf{name: pageName} - page name.
\end{itemize}


\subsubsection{SNETFacebookPageDescriptionData}
\label{sec-1-5-8}
SNETFacebookPageDescriptionData is a class that contains the description of a certain Facebook page. There are getters and setters for the following variables:
\begin{itemize}
\item \textbf{category: pageCategory} - page category.
\item \textbf{description: text} - page description.
\item \textbf{id: idNumber} - page id.
\item \textbf{name: pageName} - page name.
\item \textbf{likes: number} - likes of a page.
\end{itemize}


\subsubsection{SNETFacebookGroupsData}
\label{sec-1-5-9}
SNETFacebookGroupsData is a class that contains a list of SNETFacebookGroup. The methods are:
\begin{itemize}
\item \textbf{addGroup: json} - adds a new group from a json to the collection.
\item initialize - This is the default method to initialize the class.
\item size - shows the number of groups in the collection.
\item \textbf{groupAt: position} - gets a group at a position in the collection.
\item \textbf{groups} - gets the list of groups.
\end{itemize}


\subsubsection{SNETFacebookGroup}
\label{sec-1-5-10}
SNETFacebookGroup is a class that contains a Facebook group. There are getters and setters for the following variables:
\begin{itemize}
\item \textbf{id: idNumber} - group id.
\item \textbf{name: groupName} - group name.
\end{itemize}


\subsubsection{SNETFacebookGroupDescriptionData}
\label{sec-1-5-11}
SNETFacebookGroupDescriptionData is a class that contains the description of a certain Facebook group. There are getters and setters for the following variables:
\begin{itemize}
\item \textbf{description: text} - group description.
\item \textbf{id: idNumber} - group id.
\item \textbf{name: groupName} - group name.
\end{itemize}


\subsubsection{SNETFacebookEventsData}
\label{sec-1-5-12}
SNETFacebookEventsData is a class that contains a list of SNETFacebookEvent. The methods are:
\begin{itemize}
\item \textbf{addEvent: json} - adds a new event from a json to the collection.
\item \textbf{initialize} - This is the default method to initialize the class.
\item \textbf{size} - shows the number of events in the collection.
\item \textbf{eventAt: position} - gets a event at a position in the collection.
\item \textbf{events} - gets the list of events.
\end{itemize}


\subsubsection{SNETFacebookEvent}
\label{sec-1-5-13}
SNETFacebookEvent is a class that contains a Facebook event. There are getters and setters for the following variables:
\begin{itemize}
\item \textbf{location: eventLocation} - event location.
\item \textbf{id: idNumber} - event id.
\item \textbf{name: eventName} - event name.
\end{itemize}


\subsubsection{SNETFacebookUserData}
\label{sec-1-5-14}
SNETFacebookUsersData is a class that contains a list of SNETFacebookUser. The methods are:
\begin{itemize}
\item \textbf{initialize} - This is the default method to initialize the class.
\item \textbf{size} - show the number ofusers in the collection.
\item \textbf{userAt: position} - gets a user at a position in the collection.
\item \textbf{users} - gets the list of users.
\end{itemize}
\\
A more detailed information about this module can be found in the Appendix section.
