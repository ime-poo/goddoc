\section{GODEmail}
%% Gabriel Ferreira Guilhoto &  4404279
%% Luiz Fernando da Silva Armesto & 5176378
%% Rafael Campos Cruz & 7991062
%% Renan Fichberg & 7991131

Pacote responsável pelo envio e recebimento de e-mails, tendo suporte a conexões seguras. É formado por 4 classes: uma abstrata (GODService), duas que implementam os serviços de envio e recebimento de e-mails (GODSender e GODReceiver, respectivamente) e uma que implementa o cliente do protocolo Post Office Protocol 3 (POP3) utilizando conexão segura (MAILSecurePOP3Client). Possui o pacote SqueakSSL como dependência externa. Para o envio de e-mails usando SSL é utilizada a classe SecureSMTPClient já implementada no SqueakSSL-SMTP.

Em caso de dúvidas, entre em contato com \emph{Luiz Armesto: luiz.armesto@gmail.com}.

\subsection{MAILService}

Classe abstrata que reune o código compartilhado pela MAILSender e pela MAILReceiver. Os métodos de instância estão organizados em três categorias diferentes:

\begin{itemize}
	\item \textbf{Métodos de acesso (accessing):}
	\begin{itemize}
		\item \textbf{hostname} - Getter para hostname.
		\item \textbf{hostname:} - Setter para hostname.
		\item \textbf{password:} - Setter para password.
		\item \textbf{port} - Getter para port.
		\item \textbf{port:} - Setter para a port.
		\item \textbf{user} - Getter para user.
		\item \textbf{user:} - Setter para user.
	\end{itemize}
	\item \textbf{Métodos de Inicialização (initialize-release):}
	\begin{itemize}
		\item \textbf{initialize} - Faz uma chamada para o método initializeConnectionInfo.
		\item \textbf{initializeConnectionInfo} - Inicializa os dados pertinentes à conexão do usuário.
	\end{itemize}
	\item \textbf{Métodos Privados (private):}
	\begin{itemize}
		\item \textbf{validateConnectionInfo} - Considera válida uma conexão na qual os dados estejam preenchidos adequadamente. Retorna um erro caso exista algum campo vazio.
	\end{itemize}
\end{itemize}
Os métodos estáticos estão organizados em duas categorias diferentes:
\begin{itemize}
	\item \textbf{Métodos de criação (instance creation):}
	\begin{itemize}
		\item \textbf{connectTo:identifiedAs:password:} - Construtor que recebe o host do servidor, nome de usuário e senha e cria uma instância já configurada. É utilizada a porta padrão definida pela classe filha.
		\item \textbf{connectTo:port:identifiedAs:password:} - Construtor que recebe o host do servidor, porta, nome de usuário e senha e cria uma instância já configurada. 
	\end{itemize}
	\item \textbf{Métodos de acesso (accessing):}
	\begin{itemize}
		\item \textbf{defaultPortNumber} - Método que deve ser implementado pelas classes filhas para definir a porta padrão usada para conectar aos servidores caso o cliente não a configure.
	\end{itemize}
\end{itemize}

\subsection{MAILSender}

Classe responsável pelo envio de e-mail utilizando o protocolo Post Office Protocol 3 (POP3). Herda de MAILService. Os métodos de instância desta classe estão organizados em quatro categorias diferentes:

\begin{itemize}
	\item \textbf{Métodos de acesso (accessing):}
	\begin{itemize}
		\item \textbf{smtpClient:} - Setter para um cliente SMTP (Simple Mail Transfer Protocol)
	\end{itemize}
	\item \textbf{Métodos de Inicialização (initialize-release):}
	\begin{itemize}
		\item \textbf{initializeConnectionInfo} - Realiza um chamada para o método initializeConnectionInfo da classe superior (MAILService), considerando que é um cliente SMTP.
	\end{itemize}
	\item \textbf{Métodos de requisição (requests):}
	\begin{itemize}
		\item \textbf{send:to:} - Envia um GODData para um destino (o destino é uma string contendo o e-mail). Internamente, ele realiza os seguintes processos relacionados ao envio: validação da conexão, checagem do dado (se o conteúdo está vazio), conversão de GODData para mensagem de e-mail, inicialização do cliente SMTP e, finalmente, o próprio envio.
	\end{itemize}
	\item \textbf{Métodos privados (private):}
	\begin{itemize}
		\item \textbf{getClientClass:} - Se a variável de instância já estiver iniciada (smtpClient), a usará. Caso contrário, busca obter a classe correta do cliente a partir do numero da porta. Por padrão, este método tentará usar uma conexão segura, chamando a classe SecureSMTPClient. Retorna o cliente.
	\end{itemize}
\end{itemize}

Já os métodos estáticos estão organizados em três categorias diferentes:

\begin{itemize}
	\item \textbf{Métodos de acesso (accessing):}
	\begin{itemize}
		\item \textbf{defaultPortNumber} - Getter para a porta padrão usada quando nenhuma for especificada pelo cliente (465).
	\end{itemize}
	\item \textbf{Métodos de conversão (converting):}
	\begin{itemize} 
		\item \textbf{convertToHTMLEmail:recipient:} - Recebe um objeto GODData e um endereço de e-mail e retorna uma mensagem com o conteúdo do objeto, destinada ao e-mail fornecido. Utiliza o formato HTML.
		\item \textbf{convertToSimpleEmail:recipient:} - Recebe um objeto GODData e um endereço de e-mail e retorna uma mensagem com o conteúdo do objeto, destinada ao e-mail fornecido. O conteúdo é criado em texto puro.
	\end{itemize}
	\item \textbf{Métodos de exemplo (example):}
	\begin{itemize}
		\item \textbf{example} - Contem um exemplo simples de como usar a classe para enviar e-mails.
	\end{itemize}
\end{itemize}

\subsection{MAILReceiver}

Classe responsável pelo recebimento de emails utilizando o protocolo Simple Mail Transfer Protocol (SMTP). Herda de MAILService. Os métodos de instância desta classe estão organizados em quatro categorias diferentes:

\begin{itemize}
	\item \textbf{Métodos de acesso (accessing):}
	\begin{itemize}
		\item \textbf{pop3Client:} - Setter de um cliente POP3.
	\end{itemize}
	\item \textbf{Métodos de Inicialização (initialize-release):}
	\begin{itemize}
		\item \textbf{initialize} - Faz uma chamada para o método initialize da super classe (MAILService) e inicializa os objetos relacionados ao recebimento de e-mails.
		\item \textbf{initializeConnectionInfo} - Faz uma chamada para o método initializeConnectionInfo da super classe (MAILService) para um cliente POP3.
	\end{itemize}
	\item \textbf{Métodos de requisição (requests):}
	\begin{itemize}
		\item \textbf{receive} - Este método conecta ao servidor, recebe os novos e-mails e os converte para GODData. Internamente, ele realiza os seguintes processos relacionados ao recebimento: validação da conexão, inicialização do cliente POP3, realização do login do cliente, obtenção das mensagens e o fechamento da conexão.
	\end{itemize}
	\item \textbf{Métodos privados (private):}
	\begin{itemize}
		\item \textbf{getClientClass:} - Se a variável de instância já estiver iniciada (pop3Client), a usará. Caso contrário, busca obter a classe correta do cliente a partir do numero da porta. Por padrão, este método tentará usar uma conexão segura, chamando a classe MAILSecurePOP3Client. Retorna o cliente.
	\end{itemize}
\end{itemize}

Os métodos estáticos estão organizados em três categorias diferentes:

\begin{itemize}
	\item \textbf{Métodos de acesso (accessing):}
	\begin{itemize}
		\item \textbf{defaultPortNumber} - Getter para a porta padrão usada quando nenhuma for especificada pelo cliente (995).
	\end{itemize}
	\item \textbf{Métodos de conversão (converting):} 
	\begin{itemize}
		\item \textbf{convertToGODData:} - Recebe uma mensagem de e-mail e devolve um GODData preenchido.
		\item \textbf{getContentFrom:} - Devolve o conteúdo principal de uma mensagem de e-mail. Caso a mensagem seja multipart, tenta pegar por padrão o conteúdo em texto puro, se não encontrar tenta em HTML. Utilizada pelo convertToGODData.
		\item \textbf{getField:From:} - Devolve o conteúdo de um determinado campo da mensagem de e-mail fornecida. Utilizado internamente pelo convertToGODData.
		\item \textbf{getTimestampFrom:} - Devolve a data da mensagem de e-mail. Usado pelo convertToGODData.
	\end{itemize}
	\item \textbf{Métodos de exemplo (example):}
	\begin{itemize}
		\item \textbf{example} - Contem um exemplo simples de como usar a classe para receber e-mails.
	\end{itemize}
\end{itemize}

\subsection{MAILSecurePOP3Client}

Classe responsável pela implementação do protocolo Post Office Protocol 3 (POP3) a patir do SqueakSSL para criar uma conexão segura. Herda da classe POP3Client do próprio Squeak e sobrescreve o método que cria a conexão com o servidor.

\begin{itemize}
	\item \textbf{Possui apenas um método privado (private):}
	\begin{itemize}
		\item \textbf{ensureConnection} - Cria conexão com o servidor do mesmo modo que a classe POP3Client faria, com a única diferença de utilizar o SqueakSSL para permitir uma conexão segura.
	\end{itemize}
\end{itemize}
