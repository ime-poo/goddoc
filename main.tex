\documentclass[12pt]{article}

\usepackage{graphicx,url}

\usepackage[brazilian]{babel}
\usepackage[utf8x]{inputenc}
\usepackage{amssymb,amsmath}
\usepackage[T1]{fontenc}
\usepackage{verbatim}
\usepackage{hyperref}
\usepackage{fullpage}
\usepackage{multicol}
\usepackage{lmodern}
\usepackage{xcolor}
\usepackage{listings}
\usepackage[colorinlistoftodos]{todonotes}
\usepackage{fixltx2e}
\usepackage{longtable}
\usepackage{float}
\usepackage{wrapfig}
\usepackage{soul}
\usepackage{textcomp}
\usepackage{marvosym}
\usepackage{wasysym}
\usepackage{latexsym}
\usepackage{amssymb}
\usepackage{hyperref}
\tolerance=1000
\usepackage{geometry}
\geometry{left=0.7in,right=0.7in,top=1in,bottom=1in}
\providecommand{\alert}[1]{\textbf{#1}}

\lstset{%
 basicstyle=\small\ttfamily\color{black!85},
 breaklines = true,
 keywordstyle=\bfseries\color{black},
 emphstyle=\color{blue},
 columns=fullflexible,
 showstringspaces=false
}%

\lstdefinelanguage{GODSmalltalk}
  {keywords={SSSpreadsheetData, SSSheet, SSRow, SSCell},
  morestring=[b]{'},
  stringstyle=\color{purple},
  alsoletter={:}, 
  emph={new}
}

% Python environment
\lstnewenvironment{godCode}[1][]
{
\lstset{language=GODSmalltalk,
    #1}
}
{}


\graphicspath { {figures/} }
\setlength{\parindent}{0cm}
% Algum problema em manter o \parindent em zero? Eu particularmente não gosto da indentação e normalmente pulo uma
% linha extra entre os paragrafos. --GraphGenerator


\title{Grande Organizador de Dados (GOD) \\Great Data Organizer}
\author{
Programação Orientada a Objetos (MAC0441/5714)\\
Departamento Ciência da Computação -- DCC\\
Instituto de Matemática e Estatística -- IME\\
Universidade de Sao Paulo -- USP
}
\date{Dezembro, 2014}

\begin{document}
\maketitle
\newpage
\tableofcontents
\newpage

\section{Introdução}
Grande Organizador de Dados (GOD) é um sistema composto por módulos que realizam a entrada e saída de diversas fontes de informação, como arquivos texto, planilhas, html, emails e redes sociais, entre outros. 
Essas informações podem ser utilizadas para a criação de aplicações que fazem a análise de dados de diferentes formas, como análise de sentimentos e agregadores de informações.
O GOD foi desenvolvido em Smalltalk pelos alunos da disciplina de Programação Orientada a Objetos. A seguir são descritas as principais funcionalidades dos módulos e das aplicações do GOD.


\section{Features}
\subsection{Emails}
\begin{enumerate}
\item Recebimento de e-mails\\
É possível receber e-mails utilizando uma conexão segura.

\item Envio de e-mails\\
É possível enviar e-mails utilizando uma conexão segura.

\end{enumerate}

\subsection{Redes sociais}
\subsubsection{Facebook}
\begin{enumerate}
\item Busca de usuários
\item Busca por posts de usuário\\
É possível também filtrar os posts.
\item Busca de páginas
\end{enumerate}

\subsubsection{Twitter}
\begin{enumerate}
\item Busca na timeline
\item Busca por hashtags
\item Busca de tweets\\
Também é possível filtrar por data e por língua e realizar buscas avançadas:
\begin{enumerate}
\item Busca contendo todas as palavras
\item Busca por qualquer uma das palavras
\item Busca pela frase exata
\item Busca por qualquer hashtag
\item Busca por todas as hashtags
\item Busca contendo nenhuma das palavras
\end{enumerate}
\end{enumerate}

\subsection{Arquivos de texto}

\begin{enumerate}
\item Entrada e saída em RTF
\item Entrada e saída em ODT
\item Entrada e saída em TXT
\item Saída em PDF
\end{enumerate}

\subsection{Planilhas}
\begin{enumerate}
\item Entrada e saída em CSV
\item Entrada e saída em ODS
\item Merge de duas planilhas
\item População de planilhas a partir de dicionários e listas
\end{enumerate}

\subsection{Gráficos}
\begin{enumerate}
\item Saída de dados em gráfico de barras, em PNG
\end{enumerate}

\subsection{Páginas web}
\begin{enumerate}
\item Construção de páginas por meio de componentes
\end{enumerate}

\subsection{Filtros}
\begin{enumerate}
\item Filtragem de dados por título
\item Filtragem por conteúdo
\item Filtragem por origem
\item Filtragem por tags
\item Filtragem por começo e fim de string
\end{enumerate}

\subsection{Processadores}
\begin{enumerate}
\item Tratamento de HTML e TXT
\item Geração de tags
\item Cálculo de tf-idf
\item Cálculo de média, mediana, variância e desvio padrão para números
\end{enumerate}

\subsection{Banco de dados}
\begin{enumerate}
\item Armazenamento de objetos
\item Realização de consultas sobre objetos armazenados
\end{enumerate}


\subsection{Análise de sentimento de consumo}
\begin{enumerate}
\item Contagem de palavras positivas ou negativas em textos
\item Saída da análise em gráfico
\item Saída da análise em planilha
\end{enumerate}

\subsection{Análise de sentimento político}
\begin{enumerate}
\item Utilização de dados do TSE sobre candidatos e partidos
\item Contagem de palavras positivas ou negativas
\item Saída da análise em gráfico
\item Saída da análise em planilha
\end{enumerate}

\subsection{Agregação de informações acadêmicas}
\begin{enumerate}
\item Cálculo de estatísticas sobre um pesquisador
\item Geração de gráfico de fator de impacto e quantidade de papers publicados
\end{enumerate}

\subsection{Agregação de conferências}
\begin{enumerate}
\item Leitura de dados do ConfSearch
\item Leitura de dados do WIKICFP
\item Leitura de dados do Quallis
\item Entrada e saída em planilha (CSV)
\item Geração de palavras-chave para conferências
\end{enumerate}
\newpage
\def\godkernel{\textbf{GODKernel}}
\def\goddata{\textbf{GODData}}
\def\configGod{\textbf{ConfigurationOfGOD}}

\section{Núcleo (\godkernel)}

\godkernel~ é o módulo que contém a classe que representa a estrutura básica para manipulação de dados do projeto.
Essa classe é chamada \goddata~ contém os atributos para armazenamento comuns aos diferentes tipos de fontes de dados usados no projeto. 
\goddata~ pode ser estendido para atender às especificidades dessas diferentes fontes de informações.\\
Esse módulo também é responsável por gerenciar o repositório do projeto, realizando o controle de versões e suas dependências.\\
Em caso de dúvidas, entre em contato com \emph{Higor: hamario [at] ime.usp.br}.


\subsection{\goddata} 

Classe que representa a estrutura básica a serem armazenadas das diferentes fontes de informação, comoarquivos texto, planilhas, páginas html, redes sociais,e-mails, entre outras.

\subsubsection{Atributos}

\begin{itemize}
 \item \textbf{author} - autor de um conteúdo.
 \item \textbf{content} - armazena o conteúdo principal de uma fonte de informação.
 \item \textbf{height} - altura do objeto a ser renderizado.
 \item \textbf{id} - valor exclusivo do objeto.
 \item \textbf{layoutGrid} - matriz que contém objetos \goddata~ internos. Atributo que representa a posição na qual os objetos internos devem ser renderizados em uma página web.
 \item \textbf{origin} - origem da informação.
 \item \textbf{tags} - termos principais associados ao objeto \goddata.
 \item \textbf{timestamp} - data do \goddata.
 \item \textbf{title} - Titulo da informação.
 \item \textbf{width} - largura do objeto a ser renderizado.
\end{itemize}


\subsubsection{Métodos}

\begin{itemize}
 \item \textbf{addOnLayoutGrid:a\goddata~ at: aRow at: aColumn} - adiciona um \goddata~ ao layoutGrid.
 \item \textbf{addTag:} - adiciona uma nova tag à coleção de tags.
 \item \textbf{is\goddata} - verifica se um objeto é do tipo \goddata.
 \item \textbf{isWho} - informa o tipo de subclasse de \goddata.
 \item \textbf{layoutGridRowSize: rowNumber columnSize: columnNumber} - define as dimensões do layoutGrid.
 \item \textbf{tagExists:} - verifica se uma determinada tag existe.
\end{itemize}


\subsubsection{Exemplos}

\begin{godCode}
'Preenchendo um \goddata'
goddata := \godData~ new.
goddata title:'titulo do objeto'.
goddata author:'proprietario da informacao'
goddata content: 'Conteudo que faz parte da informacao'
goddata addTag: 'termo'.

'Definindo o tamanho do layoutGrid e atribuindo objetos \goddata~ a este grid'
goddata {layoutGridRowSize: 5 columnSize: 3.
goddata addOnLayoutGrid: godddata1 at: 1 at: 1.
goddata addOnLayoutGrid: godddata1 at: 3 at: 3.

'Adicionando uma tag a colecao'
goddata addTag: 'termo'.

\end{godCode}


\subsection{\configGod} 

Classe responsável pelo gerenciamento dos módulos do projeto. Possui módulos para fazer o download de versões dos módulos e dependências do projeto, assim como a realização de backup do repositório.

\subsubsection{Métodos}

\begin{itemize}
 \item \textbf{backupPackages:} - realiza o backup dos módulos do projeto.
 \item \textbf{baseline01:} -  inclui as descrições dos pacotes e dependências do projeto.
\end{itemize}

\subsubsection{Exemplos}

\begin{godCode}
'Baixando a ultima versao dos pacotes do projeto'
(\configGod~ project version: '0.1-baseline01') load.
\end{godCode}



\newpage
\section{Banco de dados (GODBases)}

O módulo de banco de dados orientado a objetos (GODBases) foi elaborado com o intuito de compor o Projeto GOD e cumprir os requisitos da disciplina de Programação Orientada a Objetos. Tem por objetivo fornecer um banco de dados capaz de armazenar informações sobre o projeto GOD, possibilitando operações de inserção, remoção e busca. Seu desenvolvimento foi realizado no Squeak Smalltalk, através do banco de dados orientado a objetos Magma.\\
Contato: Lucy Mansilla (lucyacm@ime.usp.br) e Silvia Scheunemann Silva (silviass@ime.usp.br).


\subsection{Magma}
Magma é uma biblioteca de banco de dados orientado a objetos disponível no Squeak, no modo monousuário (single-user) e também multiusuário (arquitetura cliente-servidor). Os pacotes e métodos disponíveis podem ser obtidos através do download na página:
\begin{itemize}
\item{http://map.squeak.org/packagesbyname.}
\end{itemize}

\subsection{ Classes do GODBases}
O pacote GODBases contém os principais métodos do banco de dados orientado a objetos do projeto GOD. A seguir fazemos uma pequena descrição sobre cada uma de suas classes.


\subsubsection{GDBServerController}

A classe GDBControllerServer permite fazer transações diferentes no servidor do banco de dados, bem como criar e eliminar um repositório. 


\begin{itemize}
\item Para criar um repositório padrão, execute a seguinte linha no workspace:\\
{GDBServerController createRepository.}
\item Para criar um repositório especifico, execute:\\
{GDBServerController createRepository: 'nomeDoRepositorio'.}
\item Para inicializar o servidor com um repositório padrão, execute:\\
{GDBServerController startServer.}
\item Para inicializar o servidor com um repositório específico, execute:\\
{GDBServerController startServer:'nomeDoRepositorio'.}
\item{Para saber se o servidor foi iniciado, execute:}
{GDBServerController isStarted.}\\
esse comando retorna 'true' se o servidor estiver inicializado.
\end{itemize}

Uma vez inicializado o servidor, salve como uma imagem-servidor. Assim, para fechar o servidor, só precisa fechar a imagem salva, e quando você abri-lo novamente, o servidor já estará funcionando.

\begin{itemize}
\item Uma outra opção para fechar o servidor, é executar:\\
{GDBServerController stopServer.}
\item Para apagar um repositório específico, execute:\\
{GDBServerController deleteRepository:'nomeDoRepositorio'}
\end{itemize}


\subsubsection{GDBClientController}

A classe GDBClientController permite fazer a conexão de um usuário/ cliente com o servidor do banco de dados.
Primeiro é necessário inicializar o servidor em uma imagem separada, como indicado anteriormente, em seguida, executar as seguintes instruções:

\begin{itemize}
\item Para conectar um cliente com o banco de dados, execute o seguinte:\\
{GDBClientController initializeSession}
\end{itemize}

Uma vez inicializada a sessão do usuário é possível fazer qualquer transação no banco de dados, ou seja, é possível fazer uso dos métodos contidos nas classes: GDBDatabase, GDBData e GDBConference.


\begin{itemize}
\item Para atualizar qualquer alteração de outro cliente, antes de qualquer transação execute:\\
{GDBClientController refresh.}
\item Para alterar a sessão do cliente, execute o seguinte:\\
{GDBClientController refresh: 'nomeDeOutroCliente'.}
\end{itemize}

Uma vez que não se precise de fazer nenhuma transação a mais no banco de dados, só temos que desconectar a sessão do cliente, para isso execute o seguinte:\\
{GDBClientController release.}



\subsubsection{GDBDatabase}
A classe GDBDatabase permite criar o banco de dados. Nosso banco de dados é representado por um dicionário que armazena duas coleções, uma coleção do tipo MagmaCollection para armazenar os objetos GODData e uma outra do tipo OrderedCollection para armazenar uma coleção que contém objetos GCConference.\\

\begin{itemize}
\item Para criar o banco de dados, execute:\\
{GDBDatabase createDatabase}\\
\item Se você quiser criar apenas o banco de dados para armazenar objetos GODData, execute:\\
{GDBData createGODDataCollection.}\\
\item Se você quiser criar apenas o banco de dados para armazenar a coleção que vai conter objetos GCConference, execute:\\
{GDBConference createConferenceCollection.}
\end{itemize}

\subsubsection{GDBData}
A classe GDBData permite fazer transações diferentes no banco de dados para objetos GODData. 


\begin{itemize}
\item Para inserir um novo objeto GODData no banco de dados, execute:\\
{ GDBData add: meuGODData.}
\item Para remover um objeto GODData do banco de dados, execute:\\
{ GDBData remove: meuGODData.}
\item Para atualizar um objeto GODData por um outro objeto GODData, execute:\\
{ GDBData update: meuAntigoGODData with: meuNovoGODData.}
\item Se você quiser apagar todos os objetos GODData armazenados no banco de dados, execute:\\
{GDBData resetData.}

\item Para buscar um objeto GODData usando parte do nome do autor, execute:\\
{ GDBData searchAuthor: 'parteDoNomeDoAutor'.}
\item Para buscar um objeto GODData usando parte do nome do titulo, execute:\\
{ GDBData searchTitle: 'parteDoNomeDoTitulo'.}
\item Para buscar um objeto GODData usando um bloco, execute:\\
{ GDBData searchFor: umBloco.}\\
onde um bloco deve ter a forma:\\
$[:objeto | objeto atributoDoObjetoGODData = valorDeConparacaoParaABusca]$.

\item Para buscar um objeto GODData usando seu ID, execute:\\
{ GDBData searchById: umIdDoTipoInteiro.}
\item Para fazer uma busca exata pelo nome do autor, execute:\\
{ GDBData exactSearchByAuthor: 'nomeExatoDoAutor'.}
\item Para fazer uma busca exata pelo titulo, execute:\\
{ GDBData exactSearchByTitle: 'nomeExatoDoTitulo'.}	

\end{itemize}


\subsubsection{GDBConference}
A classe GDBConference permite fazer diferentes transações no banco de dados para objetos GCConference. 

\begin{itemize}
\item Para inserir uma nova coleção que armazena objetos GCConference, execute:\\
{GDBConference saveConferences: minhaColeçãoGCConference.}
\item Para recuperar essa coleção armazenada no banco de dados, execute:\\
{GDBConference loadConferences.}
\item Se você quiser apagar essa coleção armazenada no banco de dados, execute:\\
{GDBConference resetConferenceData.}
\end{itemize}

\newpage
\section{GODEmail}
%% Gabriel Ferreira Guilhoto &  4404279
%% Luiz Fernando da Silva Armesto & 5176378
%% Rafael Campos Cruz & 7991062
%% Renan Fichberg & 7991131

Pacote responsável pelo envio e recebimento de e-mails, tendo suporte a conexões seguras. É formado por 4 classes: uma abstrata (GODService), duas que implementam os serviços de envio e recebimento de e-mails (GODSender e GODReceiver, respectivamente) e uma que implementa o cliente do protocolo Post Office Protocol 3 (POP3) utilizando conexão segura (MAILSecurePOP3Client). Possui o pacote SqueakSSL como dependência externa. Para o envio de e-mails usando SSL é utilizada a classe SecureSMTPClient já implementada no SqueakSSL-SMTP.

Em caso de dúvidas, entre em contato com \emph{Luiz Armesto: luiz.armesto@gmail.com}.

\subsection{MAILService}

Classe abstrata que reune o código compartilhado pela MAILSender e pela MAILReceiver. Os métodos de instância estão organizados em três categorias diferentes:

\begin{itemize}
	\item \textbf{Métodos de acesso (accessing):}
	\begin{itemize}
		\item \textbf{hostname} - Getter para hostname.
		\item \textbf{hostname:} - Setter para hostname.
		\item \textbf{password:} - Setter para password.
		\item \textbf{port} - Getter para port.
		\item \textbf{port:} - Setter para a port.
		\item \textbf{user} - Getter para user.
		\item \textbf{user:} - Setter para user.
	\end{itemize}
	\item \textbf{Métodos de Inicialização (initialize-release):}
	\begin{itemize}
		\item \textbf{initialize} - Faz uma chamada para o método initializeConnectionInfo.
		\item \textbf{initializeConnectionInfo} - Inicializa os dados pertinentes à conexão do usuário.
	\end{itemize}
	\item \textbf{Métodos Privados (private):}
	\begin{itemize}
		\item \textbf{validateConnectionInfo} - Considera válida uma conexão na qual os dados estejam preenchidos adequadamente. Retorna um erro caso exista algum campo vazio.
	\end{itemize}
\end{itemize}
Os métodos estáticos estão organizados em duas categorias diferentes:
\begin{itemize}
	\item \textbf{Métodos de criação (instance creation):}
	\begin{itemize}
		\item \textbf{connectTo:identifiedAs:password:} - Construtor que recebe o host do servidor, nome de usuário e senha e cria uma instância já configurada. É utilizada a porta padrão definida pela classe filha.
		\item \textbf{connectTo:port:identifiedAs:password:} - Construtor que recebe o host do servidor, porta, nome de usuário e senha e cria uma instância já configurada. 
	\end{itemize}
	\item \textbf{Métodos de acesso (accessing):}
	\begin{itemize}
		\item \textbf{defaultPortNumber} - Método que deve ser implementado pelas classes filhas para definir a porta padrão usada para conectar aos servidores caso o cliente não a configure.
	\end{itemize}
\end{itemize}

\subsection{MAILSender}

Classe responsável pelo envio de e-mail utilizando o protocolo Post Office Protocol 3 (POP3). Herda de MAILService. Os métodos de instância desta classe estão organizados em quatro categorias diferentes:

\begin{itemize}
	\item \textbf{Métodos de acesso (accessing):}
	\begin{itemize}
		\item \textbf{smtpClient:} - Setter para um cliente SMTP (Simple Mail Transfer Protocol)
	\end{itemize}
	\item \textbf{Métodos de Inicialização (initialize-release):}
	\begin{itemize}
		\item \textbf{initializeConnectionInfo} - Realiza um chamada para o método initializeConnectionInfo da classe superior (MAILService), considerando que é um cliente SMTP.
	\end{itemize}
	\item \textbf{Métodos de requisição (requests):}
	\begin{itemize}
		\item \textbf{send:to:} - Envia um GODData para um destino (o destino é uma string contendo o e-mail). Internamente, ele realiza os seguintes processos relacionados ao envio: validação da conexão, checagem do dado (se o conteúdo está vazio), conversão de GODData para mensagem de e-mail, inicialização do cliente SMTP e, finalmente, o próprio envio.
	\end{itemize}
	\item \textbf{Métodos privados (private):}
	\begin{itemize}
		\item \textbf{getClientClass:} - Se a variável de instância já estiver iniciada (smtpClient), a usará. Caso contrário, busca obter a classe correta do cliente a partir do numero da porta. Por padrão, este método tentará usar uma conexão segura, chamando a classe SecureSMTPClient. Retorna o cliente.
	\end{itemize}
\end{itemize}

Já os métodos estáticos estão organizados em três categorias diferentes:

\begin{itemize}
	\item \textbf{Métodos de acesso (accessing):}
	\begin{itemize}
		\item \textbf{defaultPortNumber} - Getter para a porta padrão usada quando nenhuma for especificada pelo cliente (465).
	\end{itemize}
	\item \textbf{Métodos de conversão (converting):}
	\begin{itemize} 
		\item \textbf{convertToHTMLEmail:recipient:} - Recebe um objeto GODData e um endereço de e-mail e retorna uma mensagem com o conteúdo do objeto, destinada ao e-mail fornecido. Utiliza o formato HTML.
		\item \textbf{convertToSimpleEmail:recipient:} - Recebe um objeto GODData e um endereço de e-mail e retorna uma mensagem com o conteúdo do objeto, destinada ao e-mail fornecido. O conteúdo é criado em texto puro.
	\end{itemize}
	\item \textbf{Métodos de exemplo (example):}
	\begin{itemize}
		\item \textbf{example} - Contem um exemplo simples de como usar a classe para enviar e-mails.
	\end{itemize}
\end{itemize}

\subsection{MAILReceiver}

Classe responsável pelo recebimento de emails utilizando o protocolo Simple Mail Transfer Protocol (SMTP). Herda de MAILService. Os métodos de instância desta classe estão organizados em quatro categorias diferentes:

\begin{itemize}
	\item \textbf{Métodos de acesso (accessing):}
	\begin{itemize}
		\item \textbf{pop3Client:} - Setter de um cliente POP3.
	\end{itemize}
	\item \textbf{Métodos de Inicialização (initialize-release):}
	\begin{itemize}
		\item \textbf{initialize} - Faz uma chamada para o método initialize da super classe (MAILService) e inicializa os objetos relacionados ao recebimento de e-mails.
		\item \textbf{initializeConnectionInfo} - Faz uma chamada para o método initializeConnectionInfo da super classe (MAILService) para um cliente POP3.
	\end{itemize}
	\item \textbf{Métodos de requisição (requests):}
	\begin{itemize}
		\item \textbf{receive} - Este método conecta ao servidor, recebe os novos e-mails e os converte para GODData. Internamente, ele realiza os seguintes processos relacionados ao recebimento: validação da conexão, inicialização do cliente POP3, realização do login do cliente, obtenção das mensagens e o fechamento da conexão.
	\end{itemize}
	\item \textbf{Métodos privados (private):}
	\begin{itemize}
		\item \textbf{getClientClass:} - Se a variável de instância já estiver iniciada (pop3Client), a usará. Caso contrário, busca obter a classe correta do cliente a partir do numero da porta. Por padrão, este método tentará usar uma conexão segura, chamando a classe MAILSecurePOP3Client. Retorna o cliente.
	\end{itemize}
\end{itemize}

Os métodos estáticos estão organizados em três categorias diferentes:

\begin{itemize}
	\item \textbf{Métodos de acesso (accessing):}
	\begin{itemize}
		\item \textbf{defaultPortNumber} - Getter para a porta padrão usada quando nenhuma for especificada pelo cliente (995).
	\end{itemize}
	\item \textbf{Métodos de conversão (converting):} 
	\begin{itemize}
		\item \textbf{convertToGODData:} - Recebe uma mensagem de e-mail e devolve um GODData preenchido.
		\item \textbf{getContentFrom:} - Devolve o conteúdo principal de uma mensagem de e-mail. Caso a mensagem seja multipart, tenta pegar por padrão o conteúdo em texto puro, se não encontrar tenta em HTML. Utilizada pelo convertToGODData.
		\item \textbf{getField:From:} - Devolve o conteúdo de um determinado campo da mensagem de e-mail fornecida. Utilizado internamente pelo convertToGODData.
		\item \textbf{getTimestampFrom:} - Devolve a data da mensagem de e-mail. Usado pelo convertToGODData.
	\end{itemize}
	\item \textbf{Métodos de exemplo (example):}
	\begin{itemize}
		\item \textbf{example} - Contem um exemplo simples de como usar a classe para receber e-mails.
	\end{itemize}
\end{itemize}

\subsection{MAILSecurePOP3Client}

Classe responsável pela implementação do protocolo Post Office Protocol 3 (POP3) a patir do SqueakSSL para criar uma conexão segura. Herda da classe POP3Client do próprio Squeak e sobrescreve o método que cria a conexão com o servidor.

\begin{itemize}
	\item \textbf{Possui apenas um método privado (private):}
	\begin{itemize}
		\item \textbf{ensureConnection} - Cria conexão com o servidor do mesmo modo que a classe POP3Client faria, com a única diferença de utilizar o SqueakSSL para permitir uma conexão segura.
	\end{itemize}
\end{itemize}

\newpage
\section{GODFilter}

\textbf{Grupo:}\textit{Carlos Ribas, Yoshio Mori, Larissa Moraes}

\textbf{Contato:}larissam@ime.usp.br\\

Uma das necessidades do projeto GOD era filtrar os objetos da classe GODData para serem utilizados por aplicações específicas. Nesse contexto, foi criado o pacote GODFilter que visa criar métodos para atender a necessidade de filtros dos grupos de aplicações.

O pacote GODFilter contém três classes, FILTMainFilter, FILTTextFilter e FILTConferenceFilter, sendo que a primeira é uma superclasse e as duas últimas são subclasses da primeira. 

Além disso, foram implementados testes de unidade para cada classe do GODFilter, sendo criadas as classes FILTMainFilterTest, FILTTextFilterTest e FILTConferenceFilterTest. Os testes facilitam a manutenção, pois indicam se a unidade ainda está funcional após a realização de alterações no código.

Nas subseções abaixo serão detalhadas as características de cada classe.

\subsection{Classe FILTMainFilter}

A classe FILTMainFilter possui métodos para a realização de filtros em coleções de GODData que podem ser utilizados por qualquer aplicação. 
Foram criados quatro métodos com as seguintes assinaturas:

\begin{itemize}

\item \textit{\textbf{filter:}collectionOfGodData \textbf{byContent:}content} 

Esse método filtra uma coleção de objetos da classe GODData pelo seu conteúdo, ou seja, pelo atributo \textit{content}.\\

\item \textit{\textbf{filter:}collectionOfGodData \textbf{byOrigin:}origin} 

Esse método filtra uma coleção de objetos da classe GODData pela sua origem, ou seja, pelo atributo \textit{origin}.\\

\item \textit{\textbf{filter:}collectionOfGodData \textbf{byTags:}tags} 

Esse método filtra uma coleção de objetos da classe GODData pelas suas tags, ou seja, pelo atributo \textit{tags}.\\

\item \textit{\textbf{filter:}collectionOfGodData \textbf{byTitle:}title} 

Esse método filtra uma coleção de objetos da classe GODData pelo seu título, ou seja, pelo atributo \textit{title}.\\

\end{itemize}

Como foi utilizado o método \textit{\textbf{match:}text} de Smalltalk para encontrar as strings nos atributos dos objetos de GODData, pode ser passado como parâmetro cas* para encontrar strings como: casa, casado ou casamento, por exemplo. 

\subsection{Classe FILTMainFilterTest}

Esta classe possui um método chamado \textit{setUp} que contém uma coleção de \textit{GODData}. Estes dados são utilizados em todos os testes da classe \textit{FILTMainFilterTest}. Os seguintes testes foram implementados:

\begin{itemize}
\item \textit{testFIlterByTag}: testa a realização de filtro de GODData por Tags. 
\item \textit{testFilterByContent}: testa a realização de filtro de GODData por Conteúdo. 
\item \textit{testFilterByOrigin}: testa a realização de filtro de GODData por Origem. 
\item \textit{testFilterByTitle}: testa a realização de filtro de GODData por Título. 
\end{itemize}

\subsection{Classe FILTTextFilter}

A classe FILTTextFilter foi criada para atender os requisitos dos grupos de aplicação que precisavam filtrar uma página HTML. Como o HTML nada mais é que um texto contendo tags, esse filtro passou a servir para textos em geral.

Para isso, foi criado um método que recebe um texto como parâmetro e duas marcações, inicial e final. Esse método retorna uma coleção de strings que foram encontradas entre a marca inicial e a marca final. Lembrando que a string retornada não pode conter nenhuma das marcações.\\

O método criado tem como assinatura: 

\textit{\textbf{filterText:} text \textbf{between:} startkMark \textbf{and:}endMark.}

\subsection{Classe FILTTextFilterTest}

Esta classe possui um método chamado \textit{setUp}. Este método contém uma variável chamada \textit{"text"} que recebe um texto em formato HTML e outra variável chamada \textit{"newtext"} que recebe um texto simples. Além disso, este método possui três \textit{OrderedCollection} com os resultados que são esperados para os testes. Os seguintes testes foram implementados:

\begin{itemize}
\item \textit{testFilterTextBetweenAnd}: este teste verifica o conteúdo de um texto que se encontra entre duas \textit{strings}. A primeira \textit{string} indica o início do filtro e a segunda \textit{string} determina o término do filtro. O resultado da busca deve ser igual ao valor armazenado nas \textit{OrderedCollections} presentes no método \textit{setUp}. 
\item \textit{testFilterTextBetweenAndWithTextEmpty}: este teste realiza um filtro em uma variável que está vazia. Logo, espera-se que o resultado disso seja uma \textit{OrderedCollection} vazia. 
\end{itemize}

\subsection{Classe FILTConferenceFilter}

A classe FILTConferenceFilter foi criada a pedido do grupo da aplicação God's Call, pois eles precisavam filtrar os dados de todas as conferências que eles tinham para saber quais delas atendiam à busca do usuário.

Para isso, foi criado um método que recebe como parâmetro uma coleção de objetos da classe GODConference e uma chave de busca representada por um objeto da classe GCSearchFilter e retorna uma coleção de GODConference considerando as seguintes condições:
\begin{itemize}
\item keywords de GODConference contém keywords de GCSearchFilter;
\item categories de GODConference contém categories de GCSearchFilter;
\item deadlineDate de GODConference $ \le $ deadlineDate de GCSearchFilter;
\item startDate de GODConference $ \ge $ startDate de GCSearchFilter;
\item endDate de GODConference $ \le $ endDate de GCSearchFilter;
\end{itemize}

O método desconsidera os atributos do objeto GODConference ou do GCSearchFilter que são vazios, deixando de fazer a comparação entre os objetos.\\

O método criado tem como assinatura:

\textit{\textbf{filter:}collectionOfGodConference \textbf{byGCSearchFilter:}searchFilter}

\subsection{Classe FILTConferenceFilterTest}

Esta classe possui um método chamado \textit{setUp} que contém uma coleção de \textit{GCConference}. Estes dados são utilizados em todos os testes da classe \textit{FILTConferenceFilterTest}. Os seguintes testes foram implementados:

\begin{itemize}
\item \textit{testFilterBadObject}: testa uma coleção que possui objetos incorretos. Neste caso, o método deve ignorar os parâmetros que possuem valores incorretos. O resultado deste teste são todas as conferências que possuem a \textit{keyword} Automaton, pois este é único parâmetro que possui um valor válido.
\item \textit{testFilterByGCSearchFilter}: teste com valores específicos. Busca as conferências cadastradas no método \textit{setUp} que atendem aos parâmetros passados.
\item \textit{testFilterByGCSearchFilter02}: teste com valores específicos. Busca as conferências cadastradas no método \textit{setUp} que atendem aos parâmetros passados.
\item \textit{testFilterByGCSearchFilter03}: teste com valores específicos. Busca as conferências cadastradas no método \textit{setUp} que atendem aos parâmetros passados. Neste teste não foi passado o parâmetro \textit{categories}.
\item \textit{testFilterByGCSearchFilter04}: teste com valores específicos. Busca as conferências cadastradas no método \textit{setUp} que atendam aos parâmetros passados. Esse teste procura por conferências que possuem duas \textit{keywords} específicas.
\item \textit{testFilterByGCSearchFilterWrongType}: verifica se o método gera exceção ao receber um atributo difierente da classe GCSearchFilter.
\item \textit{testFilterWrongType}: verifica se o método gera exceção ao receber um atributo diferente da classe OrderedCollection.
\end{itemize}
\newpage
\section{Gráficos (GODGraphGenerator)}
	O módulo do gerador de gráficos recebe um Dictionary ou uma OrderedCollection e devolve um gráfico correspondente. O gráfico pode ser um objeto Morph ou uma imagem PNG. Quanto ao tipo de gráfico, pode ser gerado um gráfico de barras ou de linhas. Em caso de dúvidas, entre em contato com \emph{Renan Teruo Carneiro: renanteruoc@gmail.com}.
	
	\subsection{Classes}
		\begin{itemize}
		    \item \emph{GGGraph}: classe abstrata que representa um gráfico, ela por sua vez é um Morph do Squeak. Possui métodos para definir os nomes dos eixos do gráfico. Também é possível definir as dimensões do gráfico a ser gerado. O gráfico pode ser gerado no World do Squeak ou salvo como um arquivo PNG.
		    \item \emph{GGBarGraph} e \emph{GGLineGraph}: subclasses de GGGraph que representam o tipo do gráfico. Recebem como entrada um Dictionary.
		    \item \emph{GGGraphOrderedCollectionDecorator} esta subclasse de GGGraph recebe uma OrderedCollection\\
		\end{itemize}
		A única especialização dos gráficos de barra e linha é a forma como eles são desenhados, todo o
		resto da estrutura é comum para todos os gráficos.\\
	
	\subsection{Entrada}
		
		O Dictionary com os dados de entrada precisa ter o formato: \emph{palavra -> número}. Não é recomendado a inclusão de novas palavras no após o
		gráfico ter sido gerado.\\

		Apesar dos campos de labels serem opcionais, é recomendado o seu uso, pois aumenta o comprimento dos eixos e gera mais espaço para o desenho do gráfico.\\

		Outro tipo de dado de entrada pode ser uma OrderedCollection composta por duas OrderedCollections com o seguinte formato: \emph{OrderedCollection(palavra),OrderedCollection(valor)}. Esse tipo de coleção foi definido para tornar possível ordenar os dados de acordo com a necessidade do gráfico a ser gerado. O tipo do gráfico é definido pela mensagem \emph{delegate}, e "arrumamos" a entrada dentro do generateGraph dessa classe (GGGraphOrderedCollectionDecorator).

	\subsection{Saída}
	
		A saída é uma instância do tipo do gráfico, e todos os gráficos são subclasses de Morph (mais detalhes na seção 1.5).\\

		Esse objeto pode ser exportado para um arquivo de imagem PNG, que será salvo em\\\emph{\$squeak\_dir/Contents/Resources}, ou aberto no Squeak world.\\

		Casso seja necessário mudar a extenssão do arquivo de imagem, a saída pode receber qualquer mensagem destinada a objetos do tipo Morph.

	
	\subsection{Exemplo de uso}
		O código a seguir exemplifica a rotina para gerar o gráfico:\\

		\makebox[\linewidth]{
			\includegraphics[scale=0.78]{gg_code_gbar}
			\includegraphics[scale=0.78]{gg_gbar}
		}

		Rotina:
		\begin{enumerate}
			\item Instanciar o gráfico que se quer gerar, \emph{GGBarGraph} ou \emph{GGLineGraph}
			\item Adicionar os labels utilizando as mensagens \emph{horizonlLabel} e \emph{verticalLabel}
			\item Instanciar um Dictionary e adicionar as palavras e valores
			\item Gerar o gráfico com a mensagem \emph{generateGraph}
			\item Exportar ou abrir com \emph{exportToFile} e \emph{openInWorld}
		\end{enumerate}

		OBS: Uma restrição do gráfico de linhas é que as chaves no Dictionary precisam ser números não negativos. Recomendamos estejam no intervalo de $0 \bmod(100)$.\\

		\makebox[\linewidth]{
			\includegraphics[scale=1.10]{gg_code_gline}
			\includegraphics[scale=0.88]{gg_gline}
		}

\newpage
\def\godweb{\textbf{GODWeb}}
\def\goddata{\textbf{GODData}}

\section{Web (\godweb)}

\godweb~ é o módulo responsável pela geração das interfaces web do GOD. O \godweb~ é baseado no framework Seaside.
Usando o \godweb~ é possível gerar páginas contendo diversos componentes como formulários, tabelas, imagens e texto, praticamente sem ter que escrever código do Seaside.
Possui também classes para importar arquivos que serão usados nas páginas, como imagens e arquivos CSS. O \godweb~ também fornece uma classe para obter o conteúdo HTML de URLs. 
Por fim, o módulo contém as classes que formam a página principal do GOD.\\

A seguir há uma descrição das classes principais, assim como uma breve descrição das classes que representam os \textit{brushes} (tags HTML), como por exemplo brushes para texto, imagem, formulários, entre outros.
Em caso de dúvidas, entre em contato com \emph{Higor: hamario [at] ime.usp.br}.


\subsection{WEBPage} 

Classe responsável pela renderização das páginas web. Contém uma lista de elementos que serão renderizados para construir uma página web.

\subsubsection{Atributos}

\begin{itemize}
 \item \textbf{html} - Conteúdo dos elementos web que serão renderizados.
 \item \textbf{title} - Título da página.
 \item \textbf{elements} - Lista de elementos da página.
\end{itemize}

\subsubsection{Métodos}

\begin{itemize}
 \item \textbf{add:} - Adiciona um elemento à página.
 \item \textbf{render:} - Chama a renderização do título e dos elementos.
 \item \textbf{renderElements:} - Renderiza a coleção de elementos.
 \item \textbf{renderTitle:} - Renderiza o título da página.
\end{itemize}

\subsubsection{Exemplos}

\begin{godCode}
'Criando um WEBPage e incluindo diversos elementos'
page:= WEBPage new.
page initialize.
page title: 'Main page'.
page add: aParagraph.
page add: aHtmlText.
page add: anImage.
page add: aSpreadsheet.
^page.
\end{godCode}


\subsection{WEBElement} 

Classe que representa a estrutura básica de um elemento HTML. Usada para criar novos elementos.

\subsubsection{Atributos}

\begin{itemize}
 \item \textbf{html} - conteúdo dos elementos web que serão renderizados.
\end{itemize}


\subsubsection{Métodos}

\begin{itemize}
 \item \textbf{render} - método que contém o código a ser renderizado. Deve ser chamado pelas suas subclasses. O método \textbf{render} deve conter o código 
 do Seaside que renderiza o elemento.
\end{itemize}

\subsubsection{Classes de uma WEBPage}

As classes descritas abaixo representam os elementos básicos de uma página HTML.

\begin{itemize}
 \item \textbf{WEBForm} - representa um formulário, mais detalhes da classe são mostrados em \ref{subsec:webform}.
 \item \textbf{WEBFormElement} - elementos que fazem parte de um formulário. Mais detalhes sobre cada um dos elementos são mostrados em \ref{subsec:webformelement}
 \item \textbf{WEBHtmlText} - representa um texto de uma página. Permite adicionar tags html ao texto renderizado.
 \item \textbf{WEBImage} - representa uma imagem a ser renderizada. Deve-se passar o local da imagem em \textbf{initialize:}. Pode-se também alterar a altura e largura da imagem.
 \item \textbf{WEBMorph} - representa um objeto do tipo Morph, passado como parâmetro em \textbf{initialize:}. Também é possível redefinir sua altura e largura.
 \item \textbf{WEBParagraph} - um parágrafo de texto da página.
 \item \textbf{WEBSpreadsheet} - uma tabela que recebe um SSSpreadsheet para renderização na página. WEBSpreadsheet pode receber adicionalmente uma lista de imagens e/ou uma lista de links para serem renderizados como colunas adicionais da tabela. 
 Para isso, essas listas devem ser do mesmo tamanho do número de linhas do objeto SSSpreadsheet. Para isso, há diferentes métodos de inicilização que permitem iniciar uma tabela simples, uma tabela com uma lista de links, uma tabela com uma lista de imagens ou ainda uma tabela com ambas as listas.
 É possível ainda definir se a tabela possui cabeçalho (primeira linha de SSSpreadsheet), definir a largura das colunas e o tamanho da borda das linhas e das células.
\end{itemize}

\begin{godCode}
'Incluindo um texto com html na pagina.'
iHtmlText := WEBHtmlText new.
iHtmlText value:'Texto <br> <strong>formatado</strong> </br> em HTML'.

'Adicionando uma imagem.'
iImage := WEBImage new initialize: 'caminho/do/arquivo.png'.
iImage height: 200.
iImage width: 400.

'Criando uma tabela com cabecalho contendo uma coluna com imagens e outra com links.'
iWebSpreadsheet := WEBSpreadSheet new initialize: spreadSheet withImages: imageList withLinks: linkList.
iWebSpreadsheet sheetNumber: 1.
iWebSpreadsheet header: true.
iWebSpreadsheet width: 1000.
iWebSpreadsheet cellBorder: 1.
iWebSpreadsheet rowBorder: 0.
iWebSpreadsheet imageHeader: 'Icons'.
iWebSpreadsheet linkHeader: 'Web Page'.
\end{godCode}


\subsection{WEBForm} 
\label{subsec:webform}
Classe que representa um formulário web. Os elementos de um formulário, que são subclasses de \textbf{WEBFormElement}, devem ser adicionados nesta classe para serem utillizados.

\subsubsection{Atributos}

\begin{itemize}
 \item \textbf{elements} - lista de elementos do formulário.
\end{itemize}

\subsubsection{Métodos}

\begin{itemize}
 \item \textbf{add:} - adiciona os elementos do formulário.
 \item \textbf{render:} - método que renderiza a coleção de elementos do formulário.
 \item \textbf{save} - esse método recebe o objeto da classe responsável pela renderização do formulário. Os atributos dessa classe representam os campos do formulário, 
 que podem então ser usados pela aplicação para realizar o processamento dos dados.
\end{itemize}

\subsubsection{Exemplos}

\begin{godCode}
'Criando um WEBForm e adicionando elementos.'
iForm := WEBForm new.
iForm initialize.
iForm add: anInputText.
iForm add: aDate.
iForm add: aCheckbox.
iForm add: aRadio.
\end{godCode}


\subsection{WEBFormElement} 
\label{subsec:webformelement}
Classe que representa um elemento básico de um formulário. Possui os atributos \textbf{label} e \textbf{value}. Deve ser estendida pelos elementos que serão usados em um formulário.

\subsubsection{Classes de um formulário}

Foram desenvolvidas classes que representam os elementos básicos de um formulário, essas classes são listadas abaixo.

\begin{itemize}
 \item \textbf{WEBAnchor} - representa um link para uma url.
 \item \textbf{WEBCheckBox} - recebe e renderiza uma coleção de items do tipo checkbox.
 \item \textbf{WEBCheckBoxItems} - representa cada item de um checkbox.
 \item \textbf{WEBDate} - campo para seleção de data.
 \item \textbf{WEBDropDownList} - classe que recebe itens do tipo string para serem incluídos em um elemento do tipo dropdown list.
 \item \textbf{WEBFileUpload} - classe que recebe um arquivo para upload.
 \item \textbf{WEBInputText} - recebe uma string.
 \item \textbf{WEBRadioButton} - recebe e renderiza uma coleção de itens do tipo radio button.
 \item \textbf{WEBRadioButtonItem} - representa cada item de um radio group.
 \item \textbf{WEBSubmitButton} - classe que representa o botão de submit. 
\end{itemize}

 
\subsubsection{Exemplos}

\begin{godCode}
'Instanciando um WEBDropDownList'
iDrop := WEBDropDownList new.
iDrop label: 'Choose your preferred animal'.
iDrop add:'Dog'.
iDrop add:'Pig'.
iDrop add:'Sheep'.

'Criando itens de um checkbox e adicionando ao WEBCheckBox'
iCheckboxItem1 := WEBCheckBoxItem new.
iCheckboxItem1 label: 'Apples'.
iCheckboxItem1 value: false.

iCheckboxItem2 := WEBCheckBoxItem new.
iCheckboxItem2 label: 'Bananas'.
iCheckboxItem2 value: false.

iCheckboxItem3 := WEBCheckBoxItem new.
iCheckboxItem3 label: 'Oranges'.
iCheckboxItem3 value: false.

iCheckbox := WEBCheckBox new.
iCheckbox label: 'Some fruit options:'.
iCheckbox add: iCheckboxItem1.
iCheckbox add: iCheckboxItem2.
iCheckbox add: iCheckboxItem3.
\end{godCode}




\subsection{WEBFileLibrary} 

Classe que permite armazenar conteúdo de arquivos texto ou binários para serem usados nas aplicações. Dessa forma, pode-se armazenar as imagens de uma aplicação ou seu CSS, tornando a aplicação independente de arquivos locais e ainda gerenciando possíveis mudanças através de sua inclusão no repositório do projeto.


\subsubsection{Métodos}

\begin{itemize}
 \item \textbf{add: aFilePath} - adiciona o arquivo à classe, que passa a ser um método de WEBFileLibrary. O nome do método será o nome seguido da extensão do arquivo com a primeira letra da extensão em maiúsculo (Ex: arquivoTxt).
\end{itemize}

\subsubsection{Exemplos}

\begin{godCode}
'Adicionando uma imagem em WEBFileLibrary'

WEBFileLibrary add:'caminho/da/imagem.png'.

'Usando a imagem adicionada'

iImage := WEBImage new initialize: WEBFileLibrary / 'imagem.png'.

\end{godCode}



\subsection{WEBGodHome} 

Classe que é o componente principal da aplicação GOD. As demais aplicações são definidas como links a partir desta página principal.

\subsubsection{Atributos}

\begin{itemize}
 \item \textbf{MainArea:} - atributo estático que define o componente que será renderizado na área principal do GOD.
\end{itemize}

\subsubsection{Métodos}

\begin{itemize}
 \item \textbf{renderContentOn:} - chama os demais métodos de renderização da página principal do GOD.
 \item \textbf{renderFooterOn:} - renderiza o rodapé.
 \item \textbf{renderHeaderOn:} - renderiza o cabeçalho do GOD, que funciona como um menu para chamar as aplicações.
 \item \textbf{renderMainOn:} - renderiza a área principal, na qual as aplicações são renderizadas.
\end{itemize}


\subsection{WEBHomeView} 

Classe que representa um componente que serve como uma página inicial para a área principál do projeto GOD.


\subsection{WEBPageFetcher} 

Classe que realiza a captura do conteúdo HTML de páginas.

\subsubsection{Métodos}

\begin{itemize}
 \item \textbf{fetch: anURL} - retorna o string com o HTML da URL passada como parâmetro.
\end{itemize}

\subsubsection{Exemplos}

\begin{godCode}
'Obtendo o html de uma url'

WEBPageFetcher fetch:'url'.

\end{godCode}



\subsection{Usando o \godweb}
Para usar o \godweb~ é necessário criar uma classe (controladora) para fazer a renderização das páginas a serem criadas e controlar o fluxo de envio/recebimento dos dados a 
serem processados. A classe controladora deve receber em sua inicialização o componente que representa a página que será renderizada.
Essa classe controladora deve também conter atributos que representem os elementos de \godweb~ e que irão armazenar as informações preenchidas em um formulário para serem processadas.\\

Essa classe controladora deve possuir um método \textbf{render}, que vai conter a página inicial da aplicação. Essa classe pode ser usada para criar métodos de renderização para 
diversas páginas que existam para a aplicação. Para cada página da aplicação deve haver uma classe componente (subclasse de WAComponent). Essa classe deve conter o método 
\textbf{renderContentOn:}. Esse método é chamado pelo próprio seaside para renderizar a página. Dentro do método deve-se instanciar a classe controladora.

Um componente que representa a página inicial da aplicação deve conter um método \textbf{initialize} para registrar a aplicação no servidor web (conhecido como comanche).
O método \textbf{renderContentOn:} desse componente inicial deve instanciar a classe controladora passando-se como parâmetro para permitir que a classe controladora chame 
outros componentes. Portanto, a classe controladora deve ter um atributo que recebe esse componente inicial. No \textbf{renderContentOn:} deve-se também chamar o método 
\textbf{render} da classe controladora.\\

A classe controladora deve conter um método chamado \textbf{save},  que vai receber a instância da própria classe controladora com os valores preenchidos do formulário. 
O método \textbf{save} fica responsável então por chamar o fluxo que vai processar os dados, assim como pela chamada do componente que irá renderizar o resultado do processamento.\\

A classe controladora deve chamar o método \textbf{callback:}, da classe \textbf{WEBSubmitButton} pertencente ao formulário, passando ela mesma (a classe controladora) para permitir 
que \textbf{WEBSubmitButton} chame o seu método \textbf{save}. \\

No Seaside, os dados de um formulário são passados como um objeto com seus atributos modificados, e não via GET ou POST como tradicionalmente é feito na maioria dos frameworks web. 
Por isso a necessidade de passar o objeto da classe controladora para o \textbf{WEBSubmitButton}. Nos exemplos a seguir são apresentados exemplos dos métodos \textbf{save} de 
uma classe controladora e \textbf{renderContentOn:} de uma classe componente.


\subsubsection{Exemplos}

\begin{godCode}
'Codigo do metodo renderContentOn: da classe componente que representa uma pagina que contem um formulario. GODApp eh o nome da classe controladora'

|page|
page := GODApp new initialize: self; render.	
page render: html.

'Codigo do metodo save de uma classe controladora. O atributo component eh a classe que contem o formulario, que por sua vez chama o componente WEBResponse que ira 
renderizar o resultado do processamento. A classe controladora eh passada como parametro para que WEBResponse obtenha os dados preenchidos no formulario'

component call: (WEBResponse new object: self).

'Codigo do metodo estatico initialize de uma classe componente que representa a pagina inicial de uma aplicacao'

	WAAdmin register: self asApplicationAt: 'webapplication'

\end{godCode}
\newpage
{

\def\godss{\textsc{GODSpreadsheet}}

\def\classe#1{\textsc{#1}}
\def\code#1{\texttt{#1}}


\lstset{%
 basicstyle=\small\ttfamily\color{black!85},
 breaklines = true,
 keywordstyle=\bfseries\color{black},
 emphstyle=\color{blue},
 columns=fullflexible,
 showstringspaces=false
}%

\lstdefinelanguage{GODSpreadsheet}
  {keywords={SSSpreadsheetData, SSSheet, SSRow, SSCell},
  morestring=[b]{'},
  stringstyle=\color{purple},
  alsoletter={:}, 
  emph={new}
  }


% Python environment
\lstnewenvironment{godSS}[1][]
{
\lstset{language=GODSpreadsheet,
    #1}
}
{}


\section{Planilhas (GODSpreadsheet)}

O grupo de Spreadsheet preocupou-se, principalmente, em prover
funcionalidades de entrada e saída de planilhas, mas também implementou
alguns métodos que auxiliem a manipulação das mesmas.

Em caso de dúvida, entrar em contato com Diogo Haruki (\texttt{haruki} \textit{arroba} \texttt{ime} \textit{ponto} \texttt{usp} \textit{ponto} \texttt{br})

\subsection{Estrutura}
Nesta seção, explicaremos a organização das principais classes do
\godss. Quase todos os objetos das classes apresentadas
aqui possuem uma referência para seu pai (\code{parent})\footnote{Com
exceção apenas da SSSpreadsheetData, que não tem pai}.

\begin{tabular}{r p{0.6\textwidth}}
\classe{SSCell} &  Classe que corresponde a uma célula da planilha.
Assumimos que a célula tem sempre um conteúdo texto (\code{string}), e
consideramos como célula vazia aquelas que possuem \code{value} $=$
\texttt{nil}.\\

\classe{SSRow} & Classe que corresponde a uma linha da tabela.
Cada linha contém uma coleção de células, que foi mantida encapsulada.
Dessa forma, podemos garantir a coerencia entre os elementos dentro dessa
lista e a referência dos pais desses elementos.\\

\classe{SSSheet} & Classe que corresponde a uma tabela. Uma tabela
é formada por uma coleção de linhas (\classe{SSRow}), deve possuir um 
nome e, assim como em \classe{SSRow}, a coleção de linhas é encapsulada
para garantir a coesão.\\

\classe{SSSpreadsheetData} & Classe que corresponde a uma coleção de
tabelas. Cada tabela é indexida numericamente, mas há a possibilidade
de acessá-las a partir de seu nome. A coleção de tabelas
(\classe{SSSheet}) também é encapsulada, para manter a coesão entre os
diversos objetos.
\end{tabular}


\subsubsection{Indexação em todos os níveis}
Uma breve explicação dos modos de acesso aos filhos de cada classe.

Em \classe{SSSpreadsheetData}, as \classe{SSSheets} estão indexadas numericamente, mas podem ser acessadas por seu nome.

Em \classe{SSSheet}, as \classe{SSRows} estão indexadas numericamente, e seu acesso se dá unicamente pelo número da linha.

Em \classe{SSRow}, as \classe{SSCells} estão indexadas numericamente, mas podem ser acessadas pelo nome da coluna (\code{'A', 'B', \dots, 'Z, 'AA', ...}).


\subsection{How to do}
Nesta seção, vamos tentar mostrar o que pode ser feito\footnote{coisas
básicas que podem ser feitas} com o módulo e um modo de fazê-lo.

\subsubsection{Abrir arquivos}
A abertura de arquivos pelo módulo \godss~ é algo bem simples. Basta
utilizar o método de classe \code{fromFile:} da classe \classe{SSSpreadsheetData}.

Os formatos de arquivo suportados atualmente são: \code{csv, ods, xlsx}.

\begin{godSS}[moreemph={fromFile:,}]
csvSSData := SSSpreadsheetData fromFile: 'path/to/file.csv'.
odsSSData := SSSpreadsheetData fromFile: 'path/to/file.ods'.
xlsxSSData := SSSpreadsheetData fromFile: 'path/to/file.xlsx'.
\end{godSS}

\subsubsection{Salvar arquivos}
Salvar arquivos também é tarefa fácil para o módulo \godss. Para isto,
basta usar o método de instância \code{toFile:} de um objeto de
\classe{SSSpreadsheetData}. Dessa forma, suportamos saída para arquivo
\code{ods}.

\begin{godSS}[moreemph={toFile:,}]
ssData toFile: 'path/to/file.ods'.
\end{godSS}

Também podemos exportar uma \classe{SSSheet} para \code{csv}. Isso se deve
ao fato de \code{csv} guardar uma planilha simples, e não uma coleção de
planilhas como os outros formatos. Para isso, podemos usar o método de
instância \code{exportToCSV:} de um objeto de \classe{SSSheet}.

\begin{godSS}[moreemph={getSheet:,exportToCSV:}]
sheet := ssData getSheet: 1.
sheet exportToCSV: 'path/to/file.csv'.
\end{godSS}

\subsubsection{Criar uma SSSpreadsheetData e populá-la}
Caso seja necessário, também é possível criar uma \classe{SSSpreadsheetData} 
e populá-la manualmente de forma bem fácil. Primeiramente, para criar uma planilha (um objeto \classe{SSSheet}):

\begin{godSS}[moreemph={new,createSheetWithName:,createCellAtRow:,atColumn:,atColumnIndex:,withValue:,createRow}]
ssdata := SSSpreadsheetData new.
sheet := ssdata createSheetWithName: 'sheet'.
\end{godSS}

Depois de criada podemos popular a planilha de vários jeitos. O mais simples é criando célula por célula
diretamente:
\begin{godSS}[moreemph={createCellAtRow:,atColumn:,atColumnIndex:,withValue:}]
cell1 := sheet createCellAtRow: 1 atColumnIndex: 1 withValue: 'valor'.
cell2 := sheet createCellAtRow: 1 atColumn: 'B' withValue: 'valor2'.
\end{godSS}

Mas também podemos criar uma \classe{SSRow}, e definir as célular pela linha:
\begin{godSS}[moreemph={createRow,createRowAtIndex:,createCellAtColumn:,createCellAtColumnIndex:,withValue:}]
row := sheet createRow.
row := sheet createRowAtIndex: 3.
row createCellAtColumn: 'A' withValue: 'valor'.
row createCellAtColumnIndex: 2 withValue: 'valor2'.
\end{godSS}

Ou também podemos popular uma \classe{SSSheet} diretamente com valores de um tabela
smalltalk, que é uma coleção de \textit{linhas}, onde cada \textit{linha} é uma coleção
com os valores das suas células. Também é possível popular a planilha com um dicionário
smalltalk - nesse caso, cada par \code{(chave, valor)} do dicionário vira uma linha
da planilha, com as chaves na coluna ``A'' e os valores na coluna ``B''.
\begin{godSS}[moreemph={fillFromCollection:,fillFromDictionary:}]
sheet fillFromCollection: collection.
sheet fillFromDictionary: dictionary.
\end{godSS}

\subsubsection{Executar alguma ação em todas as células da planilha}
As classes \classe{SSSpreadsheetData}, \classe{SSSheet} e \classe{SSRow} contém
coleções das classes seguintes, mas cada uma encapsula essa coleção para manter coesão.

Porém, além das classes permitirem acesso de algum elemento específico dessas coleções
por mensagens específicas de cada classe, elas também permitem que o usuário
itere pelas coleções usando a mensagem \code{do:} do objeto.

Partindo de um \classe{SSSpreadsheetData} é trivial iterar por todas células de uma planilha:
\begin{godSS}[moreemph={getSheet:,do:,show:}]
sheet := ssData getSheet: 1.
sheet do: [ :row |
  row do: [ :cell |
    Transcript show: (cell value).
    ].
  ].
\end{godSS}


\subsubsection{Diferentes modos de acessar uma célula}
Temos acesso às células em diferentes níveis. Tanto em \classe{SSSheet} como em \classe{SSRow}

\begin{godSS}[moreemph={getCellAtRow:, atColumn:, atColumnIndex:, getCell:, getCellAtIndex:, getRow:}]
cell1 := sheet getCellAtRow: 1 atColumn: 'A'
cell2 := sheet getCellAtRow: 1 atColumnIndex: 3

row := sheet getRow: 6
cell3 := row getCell: 'C'
cell4 := row getCellAtIndex: 2
\end{godSS}


}
\newpage


\section{GODProcessors - Módulo de Processadores}
\begin{center}
Thiago Dias: tdsimao [at] ime.usp.br
\end{center}


O módulo \textbf{GODProcessors} tem a responsabilidade de realizar o processamento sobre
coleções de dados.

Possui duas classes principais, a \texttt{PCSStatisticsCalculator} que realiza os principais
cálculos estatísticos e a \texttt{PCSTagger} que realiza classificação de texto através de
métodos de recuperação de informação. Possui ainda duas classes auxiliares, a
\texttt{PCSPreprocessor} que realiza tratamento de strings e a \texttt{PCSGODTagger} uma fachada
para objetos \texttt{GODData}.

\subsection{PCSStatisticsCalculator} 

Esta classe calcula medidas estatísticas de tendencia central e dispersão para um determinada
coleção. Possui ainda um método para contagem de palavras.


\subsubsection{Métodos}
Todos os cálculos dessa classe podem ser realizados de duas formas, com ou sem bloco. A chamada
sem bloco realiza os cálculos sobre o valores absolutos da coleção, enquanto a chamada com
bloco executa os cálculos sobre o resultado da execução do bloco nos elementos da coleção.

Os cálculos que a classe realiza são:
\begin{description}
    \item[average]  média 
    \item[median] mediana
    \item[std] desvio padrão
    \item[var] variância
\end{description}

Além dos cálculos estatísticos, possui um método para contagem de palavras

\begin{description}
    \item[countIn: aString ocurrencesOf: aWord] conta o número de ocorrências da palavra
                   \texttt{aWord} na \textit{string} \texttt{aString}.
\end{description}

\subsubsection{Exemplos}
Para utilizar os métodos sobre uma coleção de números.
    \begin{verbatim}
    pcs := PCSStatisticsCalculator new.
    pcs average: aCollection.
    pcs median: aCollection.
    pcs std: aCollection.
    pcs var: aCollection.
    \end{verbatim}

Pode-se utilizar os métodos sobre os atributos de coleção uma coleção de objetos, passando um
bloco como parâmetro.
    \begin{verbatim}
    pcs := PCSStatisticsCalculator new.
    pcs average: aCollection key: [ :x | x width].
    pcs median: aCollection key: [ :x | x width].
    pcs std: aCollection key: [ :x | x width].
    pcs var: aCollection key: [ :x | x width].
    \end{verbatim}


    
\subsection{PCSTagger}

Essa classe utiliza métodos de treinamento não supervisionado, que recebe uma coleção de
\textit{strings} para treinamento e em seguida é capaz de retornar os objetos mais relevantes
de uma nova \textit{string} em relação a toda a coleção.

Essa classe usa a medida \textit{tf\_idf} para avaliar a relevância de cada termo. Detalhes sobre
essa técnica podem ser encontrados no livro Introduction to Information Retrieval
\footnote{http://nlp.stanford.edu/IR-book/}

\subsubsection{Variáveis de Instância}
    
\begin{description}
    \item[dictIdf] Dictionary<idf> -- um dicionário de \textit{idf}(raridade do termo na
                   coleção)
    \item[maxTf] float $[0..1]$ -- frequência máxima dos termos da string que será considerada
                 pelo algoritmo.
    \item[minTf] float $[0..1]$ -- frequência mínima dos termos da string que será considerada
                 pelo algoritmo.
    \item[minRelevance] float $[0..1]$ --  define o menor \textit{tf\_idf} (relevância) a ser
                        considerada
\end{description}

\subsubsection{Métodos}

\begin{description}
    \item [createDictIdf: aStringCollection] cria o \texttt{dictIdf}
    \item [getMoreRelevantsOf: aString] recupera os termos mais relevantes de \texttt{aString}
\end{description}


\subsubsection{Exemplos}
Para utilizar os métodos sobre uma coleção de \textit{strings}.
    \begin{verbatim}
    pcsTagger := PCSTagger new.
    pcsTagger createDictIdf: aStringCollection.
    bag := Bag new.
    bag := pcsTagger getMoreRelevantsOf: aString.
    \end{verbatim}
    
\subsection{PCSGODTagger}
Essa classe é uma fachada da classe \texttt{PCSTagger} para objetos \texttt{GODData}. A classe
\texttt{PCSGODTaggerExample} mostra um exemplo completo de uso dessa classe.

É importante notar que é possível alterar o \texttt{PCSTagger} segundo necessário.

\subsubsection{Variáveis de Instância}
\begin{description}
 \item  [tagger] PCSTagger -- objeto da classe \texttt{PCSTagger} que treina com uma coleção de 
                              strings recuperadas de uma coleção de \texttt{GODData} e recupera
                              os principais termos dessa \textit{String}
\end{description}


\subsubsection{Métodos}

\begin{description}
    \item [training: aGODDataCollection] realiza o treinamento com o atributo \texttt{content}                                  
                                          dos objetos de \texttt{aGODDataCollection}
    \item [addTagsTo: aGODData] define tags para o atributo \texttt{tags} de \texttt{aGODData}
    \item [tagCollection: aGODDataCollection] atalho para realizar o treinamento sobre uma
                                         coleção e em seguida adicionar tags a todos os objetos
                                         da mesma
 \end{description}


\subsubsection{Exemplos}
    Para adicionar \textit{tags} a um objeto \texttt{GODData}.
    \begin{verbatim}
    pcsTagger := PCSGODTagger new.
    pcsTagger training: aGODDataCollection.
    pcsTagger addTagsTo: aGODData.
    \end{verbatim}
    
    Para adicionar \textit{tags} a todos os elementos de uma coleção de \texttt{GODData} é
    possível chamar o método \texttt{tagCollection}.
    \begin{verbatim}
    pcsTagger := PCSGODTagger new.
    pcsTagger tagCollection: aGODDataCollection.
    \end{verbatim}


\subsection{PCSPreprocessor}
Essa classe realiza operações comuns de pré-processamento de \textit{strings}. 
Para utilizar essa classe você deve configurá-la definindo os valores de suas variáveis 
\ref{pre-variaveis} através de seus métodos de acesso e em seguida usar o método
\texttt{preprocess: aString} para recuperar uma coleção de \textit{tokens} da string 
\textit{aString}


\subsubsection{Variáveis de Instância} \label{pre-variaveis}
\begin{description}
 \item  [fileType] String $\in$ \{`TXT',`HTML'\} -- define o tipo de arquivo de entrada
 \item  [puctuation] String -- string com todos os caracteres de pontuação que  serão removidos,
                     ex: `,.!?'
 \item  [stopWords] Set<string>-- um conjunto com palavras comuns da língua que serão ignoradas 
\end{description}


\subsubsection{Métodos} 
Além do método principal \texttt{preprocess:} 
As operações que essa classe realiza são:
\begin{description}
    \item[treatType] realiza tratamento relativo ao tipo de arquivo, exemplo para arquivos HTML
                     remove as \textit{tags} HTML
    \item[removeStopwords] remove as palavras comuns de uma \textit{string}
    \item[tokenizer] quebra a \textit{string} em uma coleção de \textit{tokens}
\end{description}

\subsubsection{Examplos}
    \begin{verbatim}
    preprocessor := PCSPreprocessor new.
    aBag := preprocessor preprocess: aString.
    \end{verbatim}  


\newpage
\usepackage[utf8]{inputenc}
\usepackage[T1]{fontenc}
\usepackage{fixltx2e}
\usepackage{graphicx}
\usepackage{longtable}
\usepackage{float}
\usepackage{wrapfig}
\usepackage{soul}
\usepackage{textcomp}
\usepackage{marvosym}
\usepackage{wasysym}
\usepackage{latexsym}
\usepackage{amssymb}
\usepackage{hyperref}
\tolerance=1000
\usepackage{geometry}
\geometry{left=0.7in,right=0.7in,top=1in,bottom=1in}
\providecommand{\alert}[1]{\textbf{#1}}

\section{GODSocialNetIO}
\label{sec-1}
\subsection{Team}
\label{sec-1-1}

  Eduardo Alexandre, Aline Borges, Leonardo Haddad, Thiago Araujo
\subsection{Description}
\label{sec-1-2}

  Our module for the GOD Project is to allow others modules to communicate with the Facebook and Twitter servers. It does that by implementing both public APIs.\\

  The module is divided in two main parts, first are the fetchers, these classes are responsable to all the communication with the APIs.\\

  Then there are the GODData clases, these ones receive the JSON response from a request made by a fetcher and transforms it into a GODData type of object.\\

  It's important to mention that both Facebook and Twitter APIs need a client id and secret to work, this can be done creating a specific type of user in the developer section of both APIs.\\

  Last but not least, we have classes responsable to test all the methods within the fetchers and GODDatas, keep in mind that we don't have control of the Facebook and Twitter servers, so some test may fail some times simply because the server response is not what was spected.
\subsection{Class Diagram}
\label{sec-1-3}

\begin{center}
\includegraphics[width=387]{./figures/SocialNetIO-1.png}
\end{center}

\begin{center}
\includegraphics[width=.9\linewidth]{./figures/SocialNetIO-2.png}
\end{center}
\subsection{Documentation}
\label{sec-1-4}
\subsubsection{Fetchers}
\label{sec-1-4-1}
\begin{itemize}

\item Class SNETFetcher\\
\label{sec-1-4-1-1}%
SNETFetcher is a abstract class that contains the common parts used by SNETFacebookFetcher and SNETTwitterFetcher.

\begin{itemize}

\item Instance Variables
\label{sec-1-4-1-1-1}%
\begin{itemize}
\item \verb~accessToken~\\\\
\textbf{Type:}\\
     String.\\

     \textbf{Description:}\\
     This variable will store the access token necessary to access Facebook or Twitter servers.
\end{itemize}


\item Class Variables
\label{sec-1-4-1-1-2}%
\begin{itemize}
\item \verb~ApiUrl~\\\\
\textbf{Type:}\\
     String.\\

     \textbf{Description:}\\
     This variable will store the full URL to either Facebook or Twitter servers.
\end{itemize}


\item Functions
\label{sec-1-4-1-1-3}%
\begin{itemize}

\item private
\label{sec-1-4-1-1-3-1}%
\begin{itemize}
\item \verb~sendRequest: url~\\\\
\textbf{Description:}\\
      This abstract function job is to send some HTTP request to some url, it will be specialized in SNETFacebookFetcher and SNETTwitterFetcher.
\end{itemize}


\item initialize-release
\label{sec-1-4-1-1-3-2}%
\begin{itemize}
\item \verb~initialize~\\\\
\textbf{Description:}\\
      This is the default function to initialize the class.
\end{itemize}


\item connection
\label{sec-1-4-1-1-3-3}%
\begin{itemize}
\item \verb~connection~\\\\
\textbf{Description:}\\
      This abstract function job is to connect to some server.
\end{itemize}


\end{itemize} % ends low level
\end{itemize} % ends low level

\item Class SNETFacebookFetcher\\
\label{sec-1-4-1-2}%
SNETFacebookFetcher is a class that controls all the access to the Facebook API.\\
   
   It sends requests to the facebook server and returns the result as a SNETFacebookUsersData object, SNETFacebookPostsData, SNETFacebookPagesData, SNETFacebookPageDescriptionData, SNETFacebookGroupsData, SNETFacebookGroupDescriptionData and SNETFacebookEventsData.

\begin{itemize}

\item Class Variables
\label{sec-1-4-1-2-1}%
\begin{itemize}
\item \verb~ClientId~\\\\
\textbf{Type:}\\
     String.\\

     \textbf{Description:}\\
     This variable will store the client id needed to connect to the Facebook API.\\
\item \verb~ClientSecret~\\\\
\textbf{Type:}\\
     String.\\

     \textbf{Description:}\\
     This variable will store the client secret needed to connect to the Facebook API.\\
\item \verb~UserAgent~\\\\
\textbf{Type:}\\
     String.\\

     \textbf{Description:}\\
     This variable will store the user agent used by \verb~sendRequest~ function to send HTTP requests to the Facebook API.
\end{itemize}


\item Functions
\label{sec-1-4-1-2-2}%
\begin{itemize}

\item private
\label{sec-1-4-1-2-2-1}%
\begin{itemize}
\item \verb~sendRequest: url~\\\\
\textbf{Description:}\\
      This function job is to send some HTTP request to Facebook API.\\
\item \verb~byId: id~\\\\
\textbf{Description:}\\
      This function job is to be a helper to some functions that search in the Facebook API using some id.
\end{itemize}


\item initialize-release
\label{sec-1-4-1-2-2-2}%
\begin{itemize}
\item \verb~initialize~\\\\
\textbf{Description:}\\
      This is the default function to initialize the class, it will initialize all the class variables.
\end{itemize}


\item connection
\label{sec-1-4-1-2-2-3}%
\begin{itemize}
\item \verb~connection~\\\\
\textbf{Description:}\\
      This function job is to connect to Facebook API.\\
\end{itemize}


\item accessing
\label{sec-1-4-1-2-2-4}%
\begin{itemize}
\item \verb~events: event~\\\\
\textbf{Description:}\\
      This function job is to search for events with the parameter name.\\
\end{itemize}


\begin{itemize}
\item \verb~events: event since: since~\\\\
\textbf{Description:}\\
      This function job is to search for events with the parameter name since some date.\\
\item \verb~events: event since: since until: until~\\\\
\textbf{Description:}\\
      This function job is to search for events with the parameter name since some date until another date.\\
\item \verb~groupDescription: groupId~\\\\
\textbf{Description:}\\
      This function job is to get group information by id.\\
\item \verb~groups: group~\\\\
\textbf{Description:}\\
      This function job is to search for groups with the parameter name.\\
\item \verb~pageDescription: pageId~\\\\
\textbf{Description:}\\
      This function job is to get page information by id.\\
\item \verb~pages: page~\\\\
\textbf{Description:}\\
      This function job is to search for pages with the parameter name.\\
\item \verb~posts: user~\\\\
\textbf{Description:}\\
      This function job is to get posts from user.\\
\item \verb~posts: user since: since~\\\\
\textbf{Description:}\\
      This function job is to get posts from user since some date.\\
\item \verb~posts: user since: since until: until~\\\\
\textbf{Description:}\\
      This function job is to get posts from user since some date until another date.\\
\item \verb~users: user~\\\\
\textbf{Description:}\\
      This function job is to search for users with the parameter name.\\
\end{itemize}


\end{itemize} % ends low level
\end{itemize} % ends low level

\item Class SNETTwitterFetcher\\
\label{sec-1-4-1-3}%
SNETTwitterFetcher is a class that control all the access to the Twitter API.\\

   It send requests to the twitter server and return the result as data class SNETTwitterTweetsData.

\begin{itemize}

\item Instance Variables
\label{sec-1-4-1-3-1}%
\begin{itemize}
\item \verb~tokenType~\\\\
\textbf{Type:}\\
     String.\\

     \textbf{Description:}\\
     This variable will store the token type necessary to perform different requests from the Twitter server.\\
\item \verb~lang~\\\\
\textbf{Type:}\\
     String.\\

     \textbf{Description:}\\
     This variable will store the language used when searching for tweets.
\end{itemize}


\item Class Variables
\label{sec-1-4-1-3-2}%
\begin{itemize}
\item \verb~ClientId~\\\\
\textbf{Type:}\\
     String.\\

     \textbf{Description:}\\
     This variable will store the client id needed to connect to the Twitter API.
\item \verb~ClientSecret~\\\\
\textbf{Type:}\\
     String.\\

     \textbf{Description:}\\
     This variable will store the client secret needed to connect to the Twitter API.
\item \verb~UserAgent~\\\\
\textbf{Type:}\\
     String.\\

     \textbf{Description:}\\
     This variable will store the user agent used by \verb~sendRequest~ function to send HTTP requests to the Twitter API.
\end{itemize}


\item Functions
\label{sec-1-4-1-3-3}%
\begin{itemize}

\item private
\label{sec-1-4-1-3-3-1}%
\begin{itemize}
\item \verb~sendRequest: url~\\\\
\textbf{Description:}\\
      This function job is to send some HTTP request to Twitter API.\\
\item \verb~sendRequest: url method: method~\\\\
\textbf{Description:}\\
      This function job is to send some HTTP request with either GET or POST method to Twitter API.\\
\end{itemize}


\item initialize-release
\label{sec-1-4-1-3-3-2}%
\begin{itemize}
\item \verb~initialize~\\\\
\textbf{Description:}\\
      This is the default function to initialize the class, it will initialize all the class and instance variables.
\end{itemize}


\item connection
\label{sec-1-4-1-3-3-3}%
\begin{itemize}
\item \verb~connection~\\\\
\textbf{Description:}\\
      This function job is to connect to Twitter API.\\
\end{itemize}


\item accessing
\label{sec-1-4-1-3-3-4}%
\begin{itemize}
\item \verb~posts: user~\\\\
\textbf{Description:}\\
      This function job is to get posts from user.\\
\item \verb~posts: user maximum: max~\\\\
\textbf{Description:}\\
      This function job is to get posts from user specifying the maximum amount of posts.\\
\item \verb~tweets: text~\\\\
\textbf{Description:}\\
      This function job is to get tweets from text.\\
\item \verb~tweets: text maximum: max~\\\\
\textbf{Description:}\\
      This function job is to get tweets from text specifying the maximum amount of tweets.\\
\item \verb~tweets: text since: date maximum: max~\\\\
\textbf{Description:}\\
      This function job is to get tweets from text since date specifying the maximum amount of tweets.\\
\item \verb~tweets: text since: initDate until: endDate maximum: max~\\\\
\textbf{Description:}\\
      This function job is to get tweets from text since date until another date specifying the maximum amount of tweets.\\
\item \verb~tweets: text until: date~\\\\
\textbf{Description:}\\
      This function job is to get tweets from text until date.\\
\item \verb~tweets: text until: date maximum: max~\\\\
\textbf{Description:}\\
      This function job is to get tweets from text until date specifying the maximum amount of tweets.\\
\item \verb~tweetsFrom: text user: user~\\\\
\textbf{Description:}\\
      This function job is to get tweets from text from user.\\
\item \verb~tweetsFrom: text user: user maximum: max~\\\\
\textbf{Description:}\\
      This function job is to get tweets from text from user specifying the maximum amount of tweets.\\
\item \verb~tweetsFrom: text user: user since: date~\\\\
\textbf{Description:}\\
      This function job is to get tweets from text from user since date.\\
\item \verb~tweetsFrom: text user: user since: date maximum: max~\\\\
\textbf{Description:}\\
      This function job is to get tweets from text from user since date specifying the maximum amount of tweets.\\
\item \verb~tweetsFrom: text user: user since: initDate until: endDate maximum: max~\\\\
\textbf{Description:}\\
      This function job is to get tweets from text from user since date until another date specifying the maximum amount of tweets.\\
\item \verb~tweetsFrom: text user: user until: date~\\\\
\textbf{Description:}\\
      This function job is to get tweets from text from user until date.\\
\item \verb~tweetsFrom: text user: user until: date maximum: max~\\\\
\textbf{Description:}\\
      This function job is to get tweets from text from user until date specifying the maximum amount of tweets.
\end{itemize}


\item setting
\label{sec-1-4-1-3-3-5}%
\begin{itemize}
\item \verb~language: newLanguage~\\\\
\textbf{Description:}\\
      This function job is to set the language limiting the tweets searched by it.
\end{itemize}



\end{itemize} % ends low level
\end{itemize} % ends low level
\end{itemize} % ends low level
\subsubsection{GODDatas}
\label{sec-1-4-2}
\begin{itemize}

\item Class SNETTwitterTweetsData\\
\label{sec-1-4-2-1}%
SNETTwitterTweetsData is a class that contains a list of SNETTwitterTweet.

\begin{itemize}

\item Instance Variables
\label{sec-1-4-2-1-1}%
\begin{itemize}
\item \verb~tweets~\\\\
\textbf{Type:}\\
     Collection.\\

     \textbf{Description:}\\
     This variable will store a list of tweets.
\end{itemize}


\item Functions
\label{sec-1-4-2-1-2}%
\begin{itemize}

\item private
\label{sec-1-4-2-1-2-1}%
\begin{itemize}
\item \verb~addTweet: json~\\\\
\textbf{Description:}\\
      This function job is to add a new tweet from a json to the collection.\\
\end{itemize}


\item initialize-release
\label{sec-1-4-2-1-2-2}%
\begin{itemize}
\item \verb~initialize~\\\\
\textbf{Description:}\\
      This is the default function to initialize the class.
\end{itemize}


\item accessing
\label{sec-1-4-2-1-2-3}%
\begin{itemize}
\item \verb~size~\\\\
\textbf{Description:}\\
      This function job is to show how much tweets the collection contains.\\
\item \verb~tweet: position~\\\\
\textbf{Description:}\\
      This function job is to get a tweet at a position in the collection.\\
\item \verb~tweets~\\\\
\textbf{Description:}\\
      This function job is to get the list of tweets.
\end{itemize}

\end{itemize} % ends low level
\end{itemize} % ends low level

\item Class SNETTwitterTweet\\
\label{sec-1-4-2-2}%
SNETTwitterTweet is a class that contains a Twitter tweet.
   
\begin{itemize}

\item Instance Variables
\label{sec-1-4-2-2-1}%
\begin{itemize}
\item \verb~id~\\\\
\textbf{Type:}\\
     Integer.\\

     \textbf{Description:}\\
     This variable will store the id of the tweet.\\
\item \verb~message~\\\\
\textbf{Type:}\\
     String.\\

     \textbf{Description:}\\
     This variable will store the message of the tweet.\\
\item \verb~retweets~\\\\
\textbf{Type:}\\
     Integer.\\

     \textbf{Description:}\\
     This variable will store the number of retweets of the tweet.\\
\item \verb~favorited~\\\\
\textbf{Type:}\\
     Integer.\\

     \textbf{Description:}\\
     This variable will store the number of favorited of the tweet.
\end{itemize}


\item Functions
\label{sec-1-4-2-2-2}%
\begin{itemize}

\item private
\label{sec-1-4-2-2-2-1}%
\begin{itemize}
\item \verb~favorited: number~\\\\
\textbf{Description:}\\
      This function job is to set the number of favorited from tweet.\\
\item \verb~id: idNumber~\\\\
\textbf{Description:}\\
      This function job is to set the id from tweet.\\
\item \verb~message: text~\\\\
\textbf{Description:}\\
      This function job is to set the message from tweet.\\
\item \verb~retweets: number~\\\\
\textbf{Description:}\\
      This function job is to set the number of retweets from tweet.
\end{itemize}


\item accessing
\label{sec-1-4-2-2-2-2}%
\begin{itemize}
\item \verb~favorited~\\\\
\textbf{Description:}\\
      This function job is to get the number of favorited from tweet.\\
\item \verb~id~\\\\
\textbf{Description:}\\
      This function job is to get the id from tweet.\\
\item \verb~message~\\\\
\textbf{Description:}\\
      This function job is to get the message from tweet.\\
\item \verb~retweets~\\\\
\textbf{Description:}\\
      This function job is to get the number of retweets from tweet.
\end{itemize}

\end{itemize} % ends low level
\end{itemize} % ends low level

\item Class SNETFacebookPostsData\\
\label{sec-1-4-2-3}%
SNETFacebookPostsData is a class that contains a list of SNETFacebookPost.

\begin{itemize}

\item Instance Variables
\label{sec-1-4-2-3-1}%
\begin{itemize}
\item \verb~posts~\\\\
\textbf{Type:}\\
     Collection.\\

     \textbf{Description:}\\
     This variable will store a list of posts.
\end{itemize}


\item Functions
\label{sec-1-4-2-3-2}%
\begin{itemize}

\item private
\label{sec-1-4-2-3-2-1}%
\begin{itemize}
\item \verb~addPost: json~\\\\
\textbf{Description:}\\
      This function job is to add a new post from a json to the collection.\\
\end{itemize}


\item initialize-release
\label{sec-1-4-2-3-2-2}%
\begin{itemize}
\item \verb~initialize~\\\\
\textbf{Description:}\\
      This is the default function to initialize the class.
\end{itemize}


\item accessing
\label{sec-1-4-2-3-2-3}%
\begin{itemize}
\item \verb~size~\\\\
\textbf{Description:}\\
      This function job is to show how much posts the collection contains.\\
\item \verb~post: position~\\\\
\textbf{Description:}\\
      This function job is to get a post at a position in the collection.\\
\item \verb~posts~\\\\
\textbf{Description:}\\
      This function job is to get the list of posts.
\end{itemize}

\end{itemize} % ends low level
\end{itemize} % ends low level

\item Class SNETFacebookPost\\
\label{sec-1-4-2-4}%
SNETFacebookPost is a class that contains a Facebook post.
   
\begin{itemize}

\item Instance Variables
\label{sec-1-4-2-4-1}%
\begin{itemize}
\item \verb~message~\\\\
\textbf{Type:}\\
     String.\\

     \textbf{Description:}\\
     This variable will store the message of the post.\\
\item \verb~type~\\\\
\textbf{Type:}\\
     String.\\

     \textbf{Description:}\\
     This variable will store the type of the post.\\
\item \verb~caption~\\\\
\textbf{Type:}\\
     String.\\

     \textbf{Description:}\\
     This variable will store the caption of the post.\\
\item \verb~description~\\\\
\textbf{Type:}\\
     String.\\

     \textbf{Description:}\\
     This variable will store the description of the post.\\
\item \verb~picture~\\\\
\textbf{Type:}\\
     String.\\

     \textbf{Description:}\\
     This variable will store the picture URL of the post.\\
\item \verb~likes~\\\\
\textbf{Type:}\\
     Integer.\\

     \textbf{Description:}\\
     This variable will store the number of likes of the post.\\
\item \verb~shares~\\\\
\textbf{Type:}\\
     Integer.\\

     \textbf{Description:}\\
     This variable will store the number of shares of the post.\\
\item \verb~comments~\\\\
\textbf{Type:}\\
     Collection.\\

     \textbf{Description:}\\
     This variable will store a list of comments of the post.
\end{itemize}


\item Functions
\label{sec-1-4-2-4-2}%
\begin{itemize}

\item private
\label{sec-1-4-2-4-2-1}%
\begin{itemize}
\item \verb~addComment: json~\\\\
\textbf{Description:}\\
      This function job is to add a new comment from a json to the collection.\\
\item \verb~caption: text~\\\\
\textbf{Description:}\\
      This function job is to set the caption from post.\\
\item \verb~description: text~\\\\
\textbf{Description:}\\
      This function job is to set the description from post.\\
\item \verb~likes: number~\\\\
\textbf{Description:}\\
      This function job is to set the number of likes from post.\\
\item \verb~message: text~\\\\
\textbf{Description:}\\
      This function job is to set the message from post.\\
\item \verb~picture: url~\\\\
\textbf{Description:}\\
      This function job is to set the picture url from post.\\
\item \verb~shares: number~\\\\
\textbf{Description:}\\
      This function job is to set the number of shares from post.\\
\item \verb~type: postType~\\\\
\textbf{Description:}\\
      This function job is to set the type of post from post.
\end{itemize}


\item accessing
\label{sec-1-4-2-4-2-2}%
\begin{itemize}
\item \verb~comment: position~\\\\
\textbf{Description:}\\
      This function job is to get a comment from the collection.\\
\item \verb~caption~\\\\
\textbf{Description:}\\
      This function job is to get the caption from post.\\
\item \verb~description~\\\\
\textbf{Description:}\\
      This function job is to get the description from post.\\
\item \verb~likes~\\\\
\textbf{Description:}\\
      This function job is to get the number of likes from post.\\
\item \verb~message~\\\\
\textbf{Description:}\\
      This function job is to get the message from post.\\
\item \verb~picture~\\\\
\textbf{Description:}\\
      This function job is to get the picture url from post.\\
\item \verb~shares~\\\\
\textbf{Description:}\\
      This function job is to get the number of shares from post.\\
\item \verb~type~\\\\
\textbf{Description:}\\
      This function job is to get the type of post from post.
\end{itemize}

\end{itemize} % ends low level
\end{itemize} % ends low level

\item Class SNETFacebookComment\\
\label{sec-1-4-2-5}%
SNETFacebookComment is a class that contains a Facebook comment.
   
\begin{itemize}

\item Instance Variables
\label{sec-1-4-2-5-1}%
\begin{itemize}
\item \verb~message~\\\\
\textbf{Type:}\\
     String.\\

     \textbf{Description:}\\
     This variable will store the message of the comment.\\
\item \verb~id~\\\\
\textbf{Type:}\\
     Integer.\\

     \textbf{Description:}\\
     This variable will store the id of the comment.\\
\item \verb~likes~\\\\
\textbf{Type:}\\
     Integer.\\

     \textbf{Description:}\\
     This variable will store the number of likes of the comment.
\end{itemize}


\item Functions
\label{sec-1-4-2-5-2}%
\begin{itemize}

\item private
\label{sec-1-4-2-5-2-1}%
\begin{itemize}
\item \verb~id: idNumber~\\\\
\textbf{Description:}\\
      This function job is to set the id from comment.\\
\item \verb~likes: number~\\\\
\textbf{Description:}\\
      This function job is to set the number of likes from comment.\\
\item \verb~message: text~\\\\
\textbf{Description:}\\
      This function job is to set the message from comment.
\end{itemize}


\item accessing
\label{sec-1-4-2-5-2-2}%
\begin{itemize}
\item \verb~id~\\\\
\textbf{Description:}\\
      This function job is to get the id from comment.\\
\item \verb~likes~\\\\
\textbf{Description:}\\
      This function job is to get the number of likes from comment.\\
\item \verb~message~\\\\
\textbf{Description:}\\
      This function job is to get the message from comment.
\end{itemize}

\end{itemize} % ends low level
\end{itemize} % ends low level

\item Class SNETFacebookPagesData\\
\label{sec-1-4-2-6}%
SNETFacebookPagesData is a class that contains a list of SNETFacebookPage.

\begin{itemize}

\item Instance Variables
\label{sec-1-4-2-6-1}%
\begin{itemize}
\item \verb~pages~\\\\
\textbf{Type:}\\
     Collection.\\

     \textbf{Description:}\\
     This variable will store a list of pages.
\end{itemize}


\item Functions
\label{sec-1-4-2-6-2}%
\begin{itemize}

\item private
\label{sec-1-4-2-6-2-1}%
\begin{itemize}
\item \verb~addPage: json~\\\\
\textbf{Description:}\\
      This function job is to add a new page from a json to the collection.\\
\end{itemize}


\item initialize-release
\label{sec-1-4-2-6-2-2}%
\begin{itemize}
\item \verb~initialize~\\\\
\textbf{Description:}\\
      This is the default function to initialize the class.
\end{itemize}


\item accessing
\label{sec-1-4-2-6-2-3}%
\begin{itemize}
\item \verb~size~\\\\
\textbf{Description:}\\
      This function job is to show how much pages the collection contains.\\
\item \verb~page: position~\\\\
\textbf{Description:}\\
      This function job is to get a page at a position in the collection.\\
\item \verb~pages~\\\\
\textbf{Description:}\\
      This function job is to get the list of pages.
\end{itemize}

\end{itemize} % ends low level
\end{itemize} % ends low level

\item Class SNETFacebookPage\\
\label{sec-1-4-2-7}%
SNETFacebookPage is a class that contains a Facebook page.
   
\begin{itemize}

\item Instance Variables
\label{sec-1-4-2-7-1}%
\begin{itemize}
\item \verb~id~\\\\
\textbf{Type:}\\
     Integer.\\

     \textbf{Description:}\\
     This variable will store the id of the page.\\
\item \verb~category~\\\\
\textbf{Type:}\\
     String.\\

     \textbf{Description:}\\
     This variable will store the category of the page.\\
\item \verb~name~\\\\
\textbf{Type:}\\
     String.\\

     \textbf{Description:}\\
     This variable will store the name of the page.
\end{itemize}


\item Functions
\label{sec-1-4-2-7-2}%
\begin{itemize}

\item private
\label{sec-1-4-2-7-2-1}%
\begin{itemize}
\item \verb~category: pageCategory~\\\\
\textbf{Description:}\\
      This function job is to set the category from page.\\
\item \verb~id: idNumber~\\\\
\textbf{Description:}\\
      This function job is to set the id from page.\\
\item \verb~name: pageName~\\\\
\textbf{Description:}\\
      This function job is to set the name from page.
\end{itemize}


\item accessing
\label{sec-1-4-2-7-2-2}%
\begin{itemize}
\item \verb~category~\\\\
\textbf{Description:}\\
      This function job is to get the category from page.\\
\item \verb~id~\\\\
\textbf{Description:}\\
      This function job is to get the id from page.\\
\item \verb~name~\\\\
\textbf{Description:}\\
      This function job is to get the name from page.
\end{itemize}
\end{itemize} % ends low level
\end{itemize} % ends low level

\item Class SNETFacebookPageDescriptionData\\
\label{sec-1-4-2-8}%
SNETFacebookPageDescriptionData is a class that contains the description of a certain Facebook page.
   
\begin{itemize}

\item Instance Variables
\label{sec-1-4-2-8-1}%
\begin{itemize}
\item \verb~id~\\\\
\textbf{Type:}\\
     Integer.\\

     \textbf{Description:}\\
     This variable will store the id of the page.\\
\item \verb~category~\\\\
\textbf{Type:}\\
     String.\\

     \textbf{Description:}\\
     This variable will store the category of the page.\\
\item \verb~name~\\\\
\textbf{Type:}\\
     String.\\

     \textbf{Description:}\\
     This variable will store the name of the page.
\item \verb~description~\\\\
\textbf{Type:}\\
     String.\\

     \textbf{Description:}\\
     This variable will store the description of the page.
\item \verb~likes~\\\\
\textbf{Type:}\\
     Integer.\\

     \textbf{Description:}\\
     This variable will store the likes of the page.
\end{itemize}


\item Functions
\label{sec-1-4-2-8-2}%
\begin{itemize}

\item private
\label{sec-1-4-2-8-2-1}%
\begin{itemize}
\item \verb~category: pageCategory~\\\\
\textbf{Description:}\\
      This function job is to set the category of page.\\
\item \verb~description: text~\\\\
\textbf{Description:}\\
      This function job is to set the description of page.\\
\item \verb~id: idNumber~\\\\
\textbf{Description:}\\
      This function job is to set the id of page.\\
\item \verb~name: pageName~\\\\
\textbf{Description:}\\
      This function job is to set the name of page.\\
\item \verb~likes: number~\\\\
\textbf{Description:}\\
      This function job is to set the likes of page.
\end{itemize}


\item accessing
\label{sec-1-4-2-8-2-2}%
\begin{itemize}
\item \verb~category~\\\\
\textbf{Description:}\\
      This function job is to get the category of page.\\
\item \verb~description~\\\\
\textbf{Description:}\\
      This function job is to get the description of page.\\
\item \verb~id~\\\\
\textbf{Description:}\\
      This function job is to get the id of page.\\
\item \verb~name~\\\\
\textbf{Description:}\\
      This function job is to get the name of page\\
\item \verb~likes~\\\\
\textbf{Description:}\\
      This function job is to get the likes of page.
\end{itemize}

\end{itemize} % ends low level
\end{itemize} % ends low level

\item Class SNETFacebookGroupsData\\
\label{sec-1-4-2-9}%
SNETFacebookGroupsData is a class that contains a list of SNETFacebookGroup.

\begin{itemize}

\item Instance Variables
\label{sec-1-4-2-9-1}%
\begin{itemize}
\item \verb~groups~\\\\
\textbf{Type:}\\
     Collection.\\

     \textbf{Description:}\\
     This variable will store a list of groups.
\end{itemize}


\item Functions
\label{sec-1-4-2-9-2}%
\begin{itemize}

\item private
\label{sec-1-4-2-9-2-1}%
\begin{itemize}
\item \verb~addGroup: json~\\\\
\textbf{Description:}\\
      This function job is to add a new group from a json to the collection.\\
\end{itemize}


\item initialize-release
\label{sec-1-4-2-9-2-2}%
\begin{itemize}
\item \verb~initialize~\\\\
\textbf{Description:}\\
      This is the default function to initialize the class.
\end{itemize}


\item accessing
\label{sec-1-4-2-9-2-3}%
\begin{itemize}
\item \verb~size~\\\\
\textbf{Description:}\\
      This function job is to show how much groups the collection contains.\\
\item \verb~group: position~\\\\
\textbf{Description:}\\
      This function job is to get a group at a position in the collection.\\
\item \verb~groups~\\\\
\textbf{Description:}\\
      This function job is to get the list of groups.
\end{itemize}

\end{itemize} % ends low level
\end{itemize} % ends low level

\item Class SNETFacebookGroup\\
\label{sec-1-4-2-10}%
SNETFacebookGroup is a class that contains a Facebook group.
   
\begin{itemize}

\item Instance Variables
\label{sec-1-4-2-10-1}%
\begin{itemize}
\item \verb~id~\\\\
\textbf{Type:}\\
     Integer.\\

     \textbf{Description:}\\
     This variable will store the id of the group.\\
\item \verb~name~\\\\
\textbf{Type:}\\
     String.\\

     \textbf{Description:}\\
     This variable will store the name of the group.
\end{itemize}


\item Functions
\label{sec-1-4-2-10-2}%
\begin{itemize}

\item private
\label{sec-1-4-2-10-2-1}%
\begin{itemize}
\item \verb~id: idNumber~\\\\
\textbf{Description:}\\
      This function job is to set the id from group.\\
\item \verb~name: groupName~\\\\
\textbf{Description:}\\
      This function job is to set the name from group.
\end{itemize}


\item accessing
\label{sec-1-4-2-10-2-2}%
\begin{itemize}
\item \verb~id~\\\\
\textbf{Description:}\\
      This function job is to get the id from group.\\
\item \verb~name~\\\\
\textbf{Description:}\\
      This function job is to get the name from group.
\end{itemize}
\end{itemize} % ends low level
\end{itemize} % ends low level

\item Class SNETFacebookGroupDescriptionData\\
\label{sec-1-4-2-11}%
SNETFacebookGroupDescriptionData is a class that contains the description of a certain Facebook group.
   
\begin{itemize}

\item Instance Variables
\label{sec-1-4-2-11-1}%
\begin{itemize}
\item \verb~id~\\\\
\textbf{Type:}\\
     Integer.\\

     \textbf{Description:}\\
     This variable will store the id of the group.\\
\item \verb~name~\\\\
\textbf{Type:}\\
     String.\\

     \textbf{Description:}\\
     This variable will store the name of the group.
\item \verb~description~\\\\
\textbf{Type:}\\
     String.\\

     \textbf{Description:}\\
     This variable will store the description of the group.
\end{itemize}


\item Functions
\label{sec-1-4-2-11-2}%
\begin{itemize}

\item private
\label{sec-1-4-2-11-2-1}%
\begin{itemize}
\item \verb~description: text~\\\\
\textbf{Description:}\\
      This function job is to set the description of group.\\
\item \verb~id: idNumber~\\\\
\textbf{Description:}\\
      This function job is to set the id of group.\\
\item \verb~name: groupName~\\\\
\textbf{Description:}\\
      This function job is to set the name of group.\\
\end{itemize}


\item accessing
\label{sec-1-4-2-11-2-2}%
\begin{itemize}
\item \verb~description~\\\\
\textbf{Description:}\\
      This function job is to get the description of group.\\
\item \verb~id~\\\\
\textbf{Description:}\\
      This function job is to get the id of group.\\
\item \verb~name~\\\\
\textbf{Description:}\\
      This function job is to get the name of group\\
\end{itemize}

\end{itemize} % ends low level
\end{itemize} % ends low level

\item Class SNETFacebookEventsData\\
\label{sec-1-4-2-12}%
SNETFacebookEventsData is a class that contains a list of SNETFacebookEvent.

\begin{itemize}

\item Instance Variables
\label{sec-1-4-2-12-1}%
\begin{itemize}
\item \verb~events~\\\\
\textbf{Type:}\\
     Collection.\\

     \textbf{Description:}\\
     This variable will store a list of events.
\end{itemize}


\item Functions
\label{sec-1-4-2-12-2}%
\begin{itemize}

\item private
\label{sec-1-4-2-12-2-1}%
\begin{itemize}
\item \verb~addEvent: json~\\\\
\textbf{Description:}\\
      This function job is to add a new event from a json to the collection.\\
\end{itemize}


\item initialize-release
\label{sec-1-4-2-12-2-2}%
\begin{itemize}
\item \verb~initialize~\\\\
\textbf{Description:}\\
      This is the default function to initialize the class.
\end{itemize}


\item accessing
\label{sec-1-4-2-12-2-3}%
\begin{itemize}
\item \verb~size~\\\\
\textbf{Description:}\\
      This function job is to show how much events the collection contains.\\
\item \verb~event: position~\\\\
\textbf{Description:}\\
      This function job is to get a event at a position in the collection.\\
\item \verb~events~\\\\
\textbf{Description:}\\
      This function job is to get the list of events.
\end{itemize}

\end{itemize} % ends low level
\end{itemize} % ends low level

\item Class SNETFacebookEvent\\
\label{sec-1-4-2-13}%
SNETFacebookEvent is a class that contains a Facebook event.
   
\begin{itemize}

\item Instance Variables
\label{sec-1-4-2-13-1}%
\begin{itemize}
\item \verb~id~\\\\
\textbf{Type:}\\
     Integer.\\

     \textbf{Description:}\\
     This variable will store the id of the event.\\
\item \verb~location~\\\\
\textbf{Type:}\\
     String.\\

     \textbf{Description:}\\
     This variable will store the location of the event.\\
\item \verb~name~\\\\
\textbf{Type:}\\
     String.\\

     \textbf{Description:}\\
     This variable will store the name of the event.
\end{itemize}


\item Functions
\label{sec-1-4-2-13-2}%
\begin{itemize}

\item private
\label{sec-1-4-2-13-2-1}%
\begin{itemize}
\item \verb~location: eventLocation~\\\\
\textbf{Description:}\\
      This function job is to set the location from event.\\
\item \verb~id: idNumber~\\\\
\textbf{Description:}\\
      This function job is to set the id from event.\\
\item \verb~name: eventName~\\\\
\textbf{Description:}\\
      This function job is to set the name from event.
\end{itemize}


\item accessing
\label{sec-1-4-2-13-2-2}%
\begin{itemize}
\item \verb~location~\\\\
\textbf{Description:}\\
      This function job is to get the location from event.\\
\item \verb~id~\\\\
\textbf{Description:}\\
      This function job is to get the id from event.\\
\item \verb~name~\\\\
\textbf{Description:}\\
      This function job is to get the name from event.
\end{itemize}

\end{itemize} % ends low level
\end{itemize} % ends low level

\item Class SNETFacebookUsersData\\
\label{sec-1-4-2-14}%
SNETFacebookUsersData is a class that contains a list of SNETFacebookUser.

\begin{itemize}

\item Instance Variables
\label{sec-1-4-2-14-1}%
\begin{itemize}
\item \verb~users~\\\\
\textbf{Type:}\\
     Collection.\\

     \textbf{Description:}\\
     This variable will store a list of users.
\end{itemize}


\item Functions
\label{sec-1-4-2-14-2}%
\begin{itemize}

\item initialize-release
\label{sec-1-4-2-14-2-1}%
\begin{itemize}
\item \verb~initialize~\\\\
\textbf{Description:}\\
      This is the default function to initialize the class.
\end{itemize}


\item accessing
\label{sec-1-4-2-14-2-2}%
\begin{itemize}
\item \verb~size~\\\\
\textbf{Description:}\\
      This function job is to show how much users the collection contains.\\
\item \verb~user: position~\\\\
\textbf{Description:}\\
      This function job is to get a user at a position in the collection.\\
\item \verb~users~\\\\
\textbf{Description:}\\
      This function job is to get the list of users.
\end{itemize}

\end{itemize} % ends low level
\end{itemize} % ends low level
\end{itemize} % ends low level
\subsection{Contact}
\label{sec-1-5}

  thd.araujo@gmail.com

\newpage
\section{GODTextIO}

O \verb|GODTextIO| é o módulo integrante do GOD (Grande Organizador de
Dados) que se encarrega da conversão do conteúdo arquivos texto para
um objeto do tipo \verb|GODData| e vice-versa. Atualmente, o módulo
lida com arquivos \verb|odt|, \verb|pdf|, \verb|rtf| e \verb|txt|. A
seguir, daremos uma descrição sobre os pré-requisitos para seu
funcionamento e, logo depois, a composição do módulo, além de mostrar
como lidamos com os diversos tipos de arquivos.

\subsection{Equipe}

João Paulo Camargo, John
Gardenghi\footnote{\href{mailto:john@ime.usp.br}{john@ime.usp.br}},
Marcello Oliveira, José Eurípedes.


\subsection{Pré-requisitos}

Para que este módulo funcione corretamente, é necessário
\begin{itemize}
\item Ter o módulo \verb|OSProcess| instalado no Squeak. Ele pode ser
  obtido em \url{http://www.ime.usp.br/~john/OSProcessV4-3-7.sar}.
\item Ter o \verb|pdflatex| e o \verb|abiword| devidamente instalados
  e funcionando por chamada no terminal.
\end{itemize}

\subsection{Classes}

\vspace{1em}
\textbf{TXTIOConverter}

É a interface com o GOD. Qualquer outro módulo que deseje usar alguma
das nossas funcionalidades, deve fazer chamadas a métodos desta
classe. Ela é apenas a porta de entrada do nosso módulo, portanto, as
funcionalidades que ela implementa não estão codificadas nesta classe,
mas apenas são feitas chamadas a métodos competentes para cada
funcionalidade.

Esta classe possui apenas um atributo, que é o \verb|types|. É um
dicionário cujas chaves são os tipos suportados pelo módulo, a saber
\verb|odt|, \verb|pdf|, \verb|rtf| e \verb|txt|, e os valores são as
classes associadas a cada tipo de arquivo. Por outro lado, temos dez
métodos. A seguir, descrevemos os que interessam a outros módulos do
GOD. Logo depois, mostramos os demais métodos da classe, usados para
finalidades específicas da própria classe.
\begin{enumerate}

\item \verb|readFromFile: aPath type: fileType| é responsável por ler
  um arquivo cujo caminho é passado no argumento \verb|aPath| e cujo
  tipo é passado no argumento \verb|fileType|. Os tipos de arquivos
  válidos são \verb|odt|, \verb|pdf|, \verb|rtf| e \verb|txt|. Devolve
  um objeto do tipo \verb|GODData|. Sua variação é o método
  \verb|readFromFile: aPath|, em que o tipo do arquivo é extraído do
  caminho passado.

\item
  \verb|write: godData toFile: aPath type: fileType append: app overwrite: ovwr|
  é responsável por escrever o conteúdo do objeto \verb|godData| do
  tipo \verb|GODData| em um arquivo cujo caminho é passado pelo
  argumento \verb|aPath| do tipo \verb|fileType|. Os argumentos (a)
  \verb|append| e (b) \verb|overwrite| são valores lógicos que
  determinam, caso o arquivo a ser escrito já exista, se (a) o novo
  conteúdo deve ser adicionado ao final do arquivo já existente e (b)
  o arquivo deve ser sobrescrito. Suas variações são quatro métodos,
  variando apenas a combinação dos argumentos, que são
  \begin{itemize}

  \item \verb|write: godData toFile: aPath|,
  \item \verb|write: godData toFile: aPath type: fileType|,
  \item \verb|write: godData toFile: aPath type: fileType append: app|,
  \item \verb|write: godData toFile: aPath type: fileType overwrite: ovwr|.

  \end{itemize}

\end{enumerate}

Além desses sete métodos, temos
\begin{enumerate}

\item \verb|getFileTypeFromPath: aPath| extrai o tipo do arquivo a
  partir do caminho passado como argumento em \verb|aPath| e o
  retorna.

\item \verb|getPossibleTypes| retorna uma lista com as chaves do
  dicionário \verb|types|.

\item \verb|initialize| inicializa o dicionário \verb|types|.

\end{enumerate}

\vspace{1em}
\textbf{TXTMainConverter}

É a classe mãe que determina os métodos que cada classe filha deve
implementar de forma ler ou escrever arquivos de tipos
específicos. Possui sete atributos:
\begin{enumerate}
\item \verb|append| determina se o conteúdo novo deve ser adicionado
  ao final de um arquivo existente (lógico),
\item \verb|author| contém o autor do conteúdo (texto),
\item \verb|contents| contém o conteúdo do arquivo (texto),
\item \verb|date| contém a data (texto),
\item \verb|overwrite| determina se o arquivo deve ser sobrescrito
  (lógico),
\item \verb|path| contém o caminho do arquivo (texto),
\item \verb|title| contém o título (texto).
\end{enumerate}

Contém dois métodos principais:
\begin{enumerate}
\item \verb|readFile| lê um arquivo. Nesta classe, possui uma
  implementação genérica, em que converte o tipo do arquivo para
  \verb|odt|, usando o \verb|abiword|, e lê este arquivo usando o
  leitor implementado na classe \verb|TXTIOOdtConverter|. Todavia, é
  sobrescrito pelas classes filhas \verb|TXTIOOdtConverter| e
  \verb|TXTIOTextConverter|. O conteúdo lido do arquivo é salvo no
  atributo \verb|content| na classe, e este método retorna
  \verb|true|.

\item \verb|writeFile| é apenas um método abstrato.
\end{enumerate}

Além disso, temos \verb|convertFileType|, que faz a conversão do tipo
do arquivo usando o \verb|abiword|, e os métodos de acesso aos
atributos.

As classes que lidam com os tipos suportados são:
\begin{enumerate}
\item \verb|TXTIOOdtConverter| lida com arquivos \verb|odt|. Explora o
  padrão \textit{OpenDocument Format} (\verb|odf|). Esse formato é uma
  forma compactada de alguns arquivos, dos quais nos interessam o
  arquivo \verb|content.xml|. Na classe \verb|TXTOdtConverter|, o
  método \verb|readFile|, se encarrega primeiro de chamar o
  \verb|extractTextXML| para extrair o conteúdo interessante
  (\textit{tags} office:*) desse arquivo, primeiro descompactando o
  formato \verb|odt| (através da classe \verb|ZipArchive| do Squeak) e
  depois lendo como uma \textit{stream}. Depois o \verb|readFile|
  extrai a informação útil dessa \textit{stream}, verificando as
  \textit{tags}. O \verb|writeFile|, cria um diretório temporário e os
  arquivos que compoem o \verb|odt|, e escreve no \verb|content.xml| o
  conteúdo (conforme o formato \verb|xml|), compacta para \verb|odt| e
  o move para o caminho desejado pelo usuário.
\item \verb|TXTIOPdfConventer| lida com arquivos \verb|pdf|. Escreve
  arquivos usando o \verb|pdflatex| e lê usando o conversor do
  \verb|abiword| implementado na classe mãe.
\item \verb|TXTIORtfConverter| lida com arquivos \verb|rtf|. Para a
  escrita, criamos algumas tags básicas do formato e escrevemos,
  através da TXTTextConverter, em um arquivo, no formato rtf, todavia
  os lê usando o conversor \verb|abiword| implementado na classe mãe.
\item \verb|TXTIOTextConverter| lida com arquivos \verb|txt|. Escreve
  e lê os arquivos usando a classe \verb|FileStream| do Squeak
  Smalltalk.
\end{enumerate}

\newpage
\section{GODAcademics}

\textbf{Grupo:}\textit{Ígor Bonadio, Renato Cordeiro, Ruan Costa}

\textbf{Contato:}ibonadio@ime.usp.br\\

GOD Academics é um agregador de informações acadêmicas. A partir de um perfil do Google Scholar, esta aplicação constroi um relatório resumindo as informações obtidas de diversas fontes.

Atualmente GOD Academics apenas utiliza como fonte de informação o Google Scholar e o CAPES-Qualis.

\subsection{Classe ACADPaper}

Representa um artigo de um pesquisador.

\subsubsection{Atributos de instância}

\begin{itemize}
  \item name: nome do artigo.

  \item coauthors: string contendo os nomes de todos os autores.

  \item year: ano em que o artigo foi publicado.

  \item journal: é o nome do periódico onde o artigo foi publicado.

  \item impactFactor: é o fator de impacto (estrato) do journal no qual o arrigo foi publicado. Essa é uma    medida do sistema webQualis, da capes.

  \item NumberOfCitatioins: é o número de citacões que o artigo tem.
\end{itemize}

\subsubsection{Métodos de instância}

\begin{itemize}

  \item hasAttribute: aTag in: aHtmlPaper

  Verifica se a informaçãoi representada por aTag, existe em aHtmlPaper

  \item initializeFromHTML: aHtmlPaper

  Monta um paper a partir de uma string contento html. Não deve ser usado diretamente, veja   Researcher>initializeFromProfileURL.

  \item LoadAttribute: aTag from: aHtmlPaper

  Dada uma tag, indicando qual informação está sendo pedido, e um pedaço de html, retorna a   informação pedida.

  \item loadAttributesFrom: aHtmlPaper

  Carrega os atributos de um paper a partir de um html.

  \item loadImpactFactorFrom: journal

  Acha o fator de impacto do journal passado

  \item numberOfCitations

  Retorna o número de citações do artigo.

\end{itemize}

\subsection{Classe ACADQualis}

Representa a lista qualis de periódicos x fator de impacto.

É um singleton. Não deve ser instanciado com new, mas sim com o método de classe singleton.

\subsubsection{Atributos de classe}

\begin{itemize}
  \item uniqueInstance: guarda a única instância da classe.
\end{itemize}

\subsubsection{Atributos de instância}

\begin{itemize}
  \item qualis: é um dicionário. As chaves são os periódicos e o valor é o fator de impacto do periódico.
\end{itemize}

\subsubsection{Métodos de classe}

\begin{itemize}

  \item clear

  Limpa o atributo de classe que guarda a instancia da classe. Faz isso atribuindo null.

  \item singleton

  Cria uma instância da classe e a retorna.

\end{itemize}

\subsubsection{Métodos de instância}

\begin{itemize}

  \item initialize

  Carrega o hash qualis com a lista armazenada em listOfQualis

  \item journalIsSomethingLike: aJournal

  Dado o nome de um periódico, procura na no hash um periódico com nome parecido.
  Retorna a primeira chave que dê match em um dos prefixos do nome do periódico   procurado. Se não houver match, é retornado \'unknown\'.

  \item listOfQualis

  String gigante com a última lista lançada pela capes, com o nome dos periódicos e seus  respectivos fatores de impacto

  \item load: aQualisText

  Carrega um  texto qualis (que esteja no mesmo formato de listOfQualis) no dicionário  qualis.

  \item qualisOf: aJournal

  Dado o nome de um periódico, retorna seu fator de impacto.

  Se não existir no dicionário, unknown é retornado.

\end{itemize}
\newpage
\section{Agregador de conferências (GOD's Call)}
\subsection{Equipe}
	 Fabrício C. Machado\footnote{fabcm@ime.usp.br}, Phablo F. S. Moura e Camila M. de Sousa.

\subsection{Descrição geral}
  O GOD's Call é uma aplicação para o auxílio na decisão sobre qual conferência participar. 
  Essa aplicação acessa diferentes fontes na internet (e.g. WikiCFP, ConfSearch e lista de classificação da CAPES-Qualis) e combina os dados dessas fontes para criar uma lista de conferências com informações como Deadline, Palavras-Chave, Área, Ratings, Conceito Qualis, permitindo que o usuário faça buscas nessa base de dados.

\subsection{Tutorial de inicialização} %Fabricio

\subsubsection{Dependências}

Na versão atual do God's Call, usamos especialmente os seguintes pacotes (e versões):
\texttt{GODBases-lacm.31.cmz}, \texttt{GODCall-pfsm.92.mcz}, \texttt{GODCallTests-pfsm.36.mcz}, \texttt{GODFilter-ym. 23.mcz}, \texttt{GODKernel-ER17.mcz}, \texttt{GODSpreadsheet-foa.29.mcz}, \texttt{GODWeb-has.21.mcz}.

Também usamos um pacote adicional: Regex (versão \texttt{damienpollet.17}).

\subsubsection{inicialização da aplicação}

Na primeira vez que tentar abrir a página inicial do God's Call, o banco de dados estará vazio e a aplicação começará a rastrear as páginas da web. Essa etapa pode demorar algumas horas. 

\subsection{Implementação}

\subsubsection{Modelos do domínio}
\begin{itemize}
    \item \texttt{GCVenue} Essa classe modela as informações a respeito do local onde é realizada a conferência. Seus atributos de instância são: \texttt{place}, \texttt{city} e  \texttt{country}.
    \item \texttt{GCRating} Nesta classe ficam informações a respeito da avaliação (ou classificação) de uma conferência. No atributo \texttt{qualis} é guardado o conceito Qualis-Capes e no atributo \texttt{hIndex} fica o Índice~H da conferência. 
    \item \texttt{GCImportantDates} Representa as datas mais relevantes da conferência. No atributo \texttt{deadlineDate} é guardada a data limite para submissão de trabalhos. Os atributos \texttt{startDate} e \texttt{endDate} representam o período de realização (início e fim) da conferência.
    \item \texttt{GCConference} Essa classe modela uma conferência. Como principais atributos de instância, ela possui  \texttt{acronym}, \texttt{name}, \texttt{url}, \texttt{identifier} e \texttt{content}. Claramente, os 3 primeiros atributos representam o acrônimo, o nome e a url da página da conferência. O atributo \texttt{identifier} identifica unicamente uma conferência. Ele é gerado automaticamente a partir do acrônimo. No atributo \texttt{content} fica guardado o texto do ``call for papers''.
Além desses, a classe \texttt{GCConference} possui os atributos \texttt{venue} (um objeto da classe \texttt{GCVenue}), \texttt{rating} (um objeto da classe \texttt{GCRating}) e \texttt{importantDates} (um objeto da classe \texttt{GCImportantDates}).
\end{itemize}


\subsubsection{Crawling}%Camila
Obtemos informações de três fontes: ConfSearch\footnote{Desde o dia 29/11 até a data desse documento o site www.confsearh.org não estava disponizando as conferências, todas as categorias
apresenta uma lista vazia como resultado}, WikiCfp e lista de classificação CAPES-Qualis. 
Os processos de consultar e unir as informações é orquestrado pela classe \texttt{GCCrawler}. 
Essa é uma classe estática que consulta todas as fontes existentes segundo a lista retornada por \texttt{GCSourceCreator>>createAll}. 
Essas fontes são instâncias da classe \texttt{GCSource}. 

Cada instância de \texttt{GCSource} contém uma referência para um arquivo de texto que contém um dicionário de categorias. 
A chave é o nome da categoria na fonte e o valor o nome da 
categoria no sistema GOD's Call. Cada linha é um mapeamento, sendo o formato \textit{chave=valor}. Linhas em branco resultam em uma quebra do formato e portanto possíveis 
erros na execução. Além do arquivo de texto, a instância possui uma referência para uma instância de um \textit{crawler} que é capaz de buscar e transformar as conferências dessa 
fonte em uma \texttt{GCConference}.

Os \textit{crawlers} tem um método comum definido por sua superclasse \texttt{GCSourceCrawler>>crawl: categories}, que recebe uma lista de categorias e devolve uma lista de conferências.

Atualmente temos implementado \texttt{GCConfSearchCrawler}, \texttt{GCWikiCfpCrawler} e \texttt{GCQualisCrawler}. 
Os dois primeiros consultam os sites das fontes e tranformam os resultados obtidos em conferências. 
Quanto a lista de classificação da Qualis, como ela é uma fonte que sofre poucas alterações decidimos por utilizar um arquivo csv. Esse arquivo foi adicionado ao repositório usando uma subclasse de \texttt{WAFileLibrary}, \texttt{GCFileLibrary} e está acessível pelo método \texttt{GCFileLibrary>> qualisConferencesCSV}. 

Para adicionar uma nova fonte, crie uma instância de GCSource em \texttt{GCSourceCreator} e retorne-a no método \texttt{createAll}. Você provavelmente terá que criar um novo \textit{crawler} subclasse de \texttt{GCSourceCrawler}.

\subsubsection{Web Application} %Fabricio

A aplicação possui 3 páginas web, que são responsabilidade das classes \texttt{GCFirstPage}, \texttt{GCSecondPage} e \texttt{GCThirdPage}. Estas classes delegam a renderização das páginas para os métodos \texttt{GODCall>> renderFirstPage}, \texttt{GODCall>>renderSecondPage} e \texttt{GCConferenceResponse>>renderThirdPage} que usam as classes e métodos disponibilizados pelo pacote GODWeb.

\texttt{GODCall} representa uma sessão de usuário e é instanciado no momento que a página é acessada (veja \texttt{GCFirstPage>> renderContentOn:}). No momento de sua inicialização, (veja \texttt{GODCall>>initialize:}) verifica se a base de dados está carregada e ativa o crawler, caso esteja vazia. Seus atributos armazenam os parâmetros relevantes para a sessão, em especial os parâmetros de busca que são preenchidos no formulário da primeira página.

\texttt{GCConferenceResponse} é responsável pela renderização da terceira página, que é específica para uma conferência. É instanciado um objeto para cada conferência exibida na segunda página (veja \texttt{GODCall>>renderSecondPage}) e possui um método \texttt{save}, que é ativado quando seu respectivo botão na segunda página é selecionado.

\subsubsection{Testes}%Camila
Implementamos 25 testes unitários que possuem cobertura de 83\% do código. Praticamente todos os métodos que não possuem um teste são referentes à aplicação web.

Para o teste de algums métodos recorremos a utilização de \textit{mocks}. Esses são disponibilizados por meio de arquivos e permitem que métodos como o \texttt{GCConfSearchCrawler>>parse:page} recebam uma página html sem a que a requisição à página 
web seja realizada.

Para testar o método \texttt{crawl} das fontes WikiCfp e ConfSearch substituímos a referência ao objeto \texttt{WEBPageFetcher} pelos objetos \texttt{FakeWikiCfpFetcher} e \texttt{FakeConfSearchFetcher}. Isso permite que os testes rodem sem internet e que não sejam afetados por possíveis erros no pacote GODWeb.

\subsection{Exemplos}
\begin{godCode}

'Obtendo informacoes de conferencias do ConfSearch para software engineering'
csCrawler:=GCConfSearchCrawler new.
categCS:=Dictionary new.
categCS at:'category' put:'software engineering'.
csList:= csCrawler crawl: categCS.

'Obtendo informacoes de conferencias do WikiCfp para computer science'
wCrawler:=GCWikiCfpCrawler new.
categW:=Dictionary new.
categW at:'computer science' put:'computer science'.
wList:= wCrawler crawl: categW.

\end{godCode}


\newpage
\section{Análise de sentimentos de coonsumo e político (GODSentimentAnalysis)}

\textbf{Grupo:}\textit{Ana Luisa de Almeida Losnak, Arthur Branco Costa, Daniel Costa Bucher, Diego de Araújo Martinez Camarinha, Rafael Batista Carmo}

\textbf{Contato:}analosnak@gmail.com


\subsection{Introdução}
GOD Project é um sistema desenvolvido em \texttt{Smalltalk} que captura dados de diversas fontes da internet, como páginas web e redes sociais. GOD processa esses dados e fornece informações para diferentes aplicações.

Nosso grupo ficou responsável pelo desenvolvimento das seguintes aplicações:

\begin{itemize}
	\item Análise de Sentimento de Consumo:
	
	Aplicação na qual o usuário faz uma busca por determinado produto, e o GOD devolve as estatísticas a respeito dele, no formato desejado (planilha ou gráfico).
	
	Nos resultados são mostrados quantas postagens nas redes sociais foram a favor, contra e neutras a respeito desse produto.
	
	\item Análise de Sentimento Político:
	
	Nesta aplicação a busca consiste em dois campos: o nome de um candidato e seu partido político. É possível ainda procurar apenas pelo partido e ver as opiniões sobre ele de um modo geral.

	Também indica-se o formato de saída desejado (planilha ou gráfico) e após apertar o botão ``Go'', a pesquisa inicia-se.
\end{itemize}


A seguir listamos as classes que criamos (nos pacotes GODSentimentAnalysis e GODSentimentAnalysisTests) e redigimos um breve descrição sobre cada um dos seus principais métodos, tanto os de classe quanto os de instância. A saber suas diferenças:

Métodos de instância operam em objetos e conseguem acessar, processar e até modificar o estado do objeto tratado; em contrapartida, ainda que métodos de classe operem sobre a classe como um todo e também possa modificar atributos de classe, eles não têm acesso a variáveis de uma instância em particular. Em \texttt{Smalltalk}, métodos de classe correspondem a mensagens para o objeto representativo da classe, ao passo que métodos de instância correspondem a mensagens enviadas para um determinado objeto da classe em questão.

\subsection{SAInterface}
Classe principal que chama todas as outras, deve ser a única utilizada por outros pacotes. Ou seja, é a única classe que seria pública.

\texttt{analyse: as:} é o método principal de toda a aplicação, o único método de instância dessa classe e que deve ser chamado pelo usuário do pacote.

Além do analyse, ela possui mais cinco métodos de classe, todos eles responsáveis pela criação dos dicionários de termos positivos e negativos:

\subsubsection{Léxicos de Sentimento}
Todas as palavras dos dicionários desta classe foram retiradas do \textbf{OpLexicon}, um léxico de sentimento para a língua portuguesa, do OntoLP, um portal de ontologias.

Após baixarmos esse léxico tivemos que tratá-lo para adequá-lo melhor ao GOD, fazendo as seguintes alterações:

\begin{itemize}

	\item Remoção dos verbos:
	Os verbos dessa lista constavam apenas no infinitivo. Para sua utilização de forma eficiente, seria necessário incluir suas conjugações mais utilizadas.
	
	\item Português brasileiro:
	As palavras lá listadas estavam em português de Portugal, então tivemos que realizar modificações para o português brasileiro, tais como: género - gênero; projecto - projeto; tónico - tônico, etc.
	
\end{itemize}

\subsection{SAFetcher}
Devolve postagens relevantes da rede social escolhida (Twitter ou Facebook) que contenham o termo pesquisado, no caso um \texttt{tag}.

O método \texttt{getTweets} pega no máximo cem tweets (limitação da API) da classe SNETTwitterFetcher

O método \texttt{setSession} auxilia a correta implementação dos testes utilizando mock.

\subsection{CSAApp}
Estas classes são responsáveis por manter o conteúdo HTML da página Web da aplicação de Análise de Consumo.

\texttt{renderResponse}: este método faz a validação dos dados inseridos pelo usuário. Apenas verifica se o campo do produto não está vazio.

Os demais métodos de instância incluem: inicialização, renderização da tela do navegador e o \texttt{save} que retorna os valores de um formulário.

\subsection{PSAApp}
Classes análogas às anteriores, porém tratam sobre a Análise de Sentimento Político. Os métodos adicionais da \texttt{PSAApp} lidam com a validação dos dados retirados do TSE.

\texttt{validateCandidates} e \texttt{validateCandidates:using:} valida a entrada do usuário, verificando se o candidato está em alguma das listas de cargo (presidente, governador, senador, etc). Por último o método \texttt{validateParty}, que verifica se o partido está na lista de partidos válidos.

\subsection{PSAPoliticalDataLoader}
No pacote \texttt{accessor} encontram-se métodos que servem apenas para retornar os dicionários de candidatos por cargo, partidos e UFs. Já no pacote \texttt{initializer} localizam-se os métodos que buscam os candidatos por cargo e estado no site do TSE.

\subsection{SAApp}
Classe que cria e preenche a estrutura de dados desejada para ser apresentado na Web. Se o usuário escolhe ``planilha'', esta classe que se comunica com a WEBSpreadSheet e formata como deve ser a planilha, no fim das contas. O mesmo ocorre quando o parâmetro passado é ``Grafo''.

\subsection{SAOutputGenerator}
Esta classe é responsável por gerar um grafo ou uma planilha a partir dos resultados e avaliações das pesquisas sobre análise de sentimento recebidas, dependendo da escolha do usuário (\texttt{generateGraph:} e \texttt{generateSpreadsheet:}) e exporta o que foi gerado (\texttt{export:as:}).

\subsection{SASentimentAnalyser}
Nesta classe, ocorre a classificação das opiniões retiradas das redes sociais através da utilização dos dicionários de palavras positivas e negativas. Por enquanto, o único critério adotado foi a contagem de cada tipo de palavra. Por exemplo, se há mais palavras positivas na postagem, ela inteira será contabilizada como positiva.

Contudo, após essa contagem e analisada a proporção entre os dois tipos de palavras, é calculado ainda um viés. Se a diferença entre palavras positivas e negativas for menor que um índice que escolhemos, o texto é classificado como neutro.

\subsection{SASentimentLabel}
\texttt{SASentimentLabel} nada mais é do que uma classe para armazenar e retornar os tipos possíveis de classificação (GOOD, BAD ou NEUTRAL).

\subsection{SASentimentTable}
Esta classe é responsável por fazer uma tabela de sentimentos (GOOD, BAD ou NEUTRAL). Ela leva em consideração as avaliações feitas.

\subsection{UML}
Afim de esclarecer ainda mais as coisas, optamos por fazer um diagrama UML para facilitar a compreensão da  modelagem adotada por nosa equipe. A seguir pode-se observar a relação entre as classes, bem como todos os seus métodos que listamos anteriormente e ainda, é possível conferir os parâmetros passados e seus tipos, bem como o tipo de retorno de cada função.

\begin{figure}[h]
\caption{Diagrama UML}
\centering
\includegraphics[width=16cm]{figures/UML.png}
\label{fig:uml}
\end{figure}

\newpage
\subsection{Rodando as Aplicações}
Para iniciar as aplicações de Análise de Sentimento basta seguir o simples passo-a-passo abaixo:

\begin{enumerate}
	\item No Squeak, abrir o Seaside Control Panel
	
	\item Dentro dele, clicar com o botão direito e então na opção Add adaptor...
	
	\item Escolher o tipo: WAComancheAdaptor
	
	\item Pressionar o botão ``Start''
	
	\item Abrir o navegador no endereço: localhost:8080/god
\end{enumerate}

Feito isso aparecerá a tela inicial, figura \ref{fig:ini}.

\begin{figure}[h]
\caption{Tela Inicial}
\centering
\includegraphics[width=14cm]{figures/inicio.png}
\label{fig:ini}
\end{figure}

\subsubsection{Análise de Sentimento de Consumo}
Para iniciar essa aplicação, basta clicar na opção do cabeçalho GODConsumptionSentimentAnalysis, que então surgirá a tela para pesquisar algum produto, figura \ref{fig:cons-pesq}.

\begin{figure}[h]
\caption{Análise de Sentimento de Consumo}
\centering
\includegraphics[width=14cm]{figures/consumo-pesquisa.png}
\label{fig:cons-pesq}
\end{figure}

Um possível resultado dessa pesquisa seria, por exemplo, o da figura \ref{fig:cons-result}, mostrado a seguir:

\begin{figure}[h]
\caption{Pesquisa - Galaxy s4}
\centering
\includegraphics[width=14cm]{figures/consumo-resultado.png}
\label{fig:cons-result}
\end{figure}

\subsubsection{Análise de Sentimento de Político}
Para rodar essa aplicação é preciso selecionar a opção: GODPoliticalSentimentAnalysis, no cabeçalho. Assim, aparecerá a seguinte tela:

\begin{figure}[h]
\caption{Análise de Sentimento Político}
\centering
\includegraphics[width=14cm]{figures/politico-pesquisa.png}
\label{fig:pol-pesq}
\end{figure}

Um possível resultado dessa pesquisa seria, por exemplo, o da figura \ref{fig:pol-result}, mostrado a seguir:

\begin{figure}[h]
\caption{Pesquisa - Dilma PT}
\centering
\includegraphics[width=14cm]{figures/politico-resultado.png}
\label{fig:pol-result}
\end{figure}

\newpage

\section{Diagramas de classe do GOD}


\subsection{GOD}
% \begin{figure}
% \caption{Diagrama do GOD}
% \centering
% \includegraphics[\linewidth]{figures/GOD.png}
% \label{fig:god}
% \end{figure}


\subsection{Banco de dados (GODBases)}

\begin{figure}[H]
\centering
\includegraphics[width=\linewidth]{cd_GODBases.png}
\label{fig:cd-godbases}
\end{figure}

\subsection{E-mails (GODEmail)}
\begin{figure}[H]
\centering
\includegraphics[width=\linewidth]{cd_GODEmail.png}
\label{fig:cd-godemail}
\end{figure}

\subsection{Filtros (GODFilter)}
\begin{figure}[H]
\centering
\includegraphics[width=\linewidth]{cd_GODFilter.jpg}
\label{fig:cd-godfilter}
\end{figure}

\subsection{Gráficos (GODGraphGenerator)}
\begin{figure}[H]
\centering
\includegraphics[width=\linewidth]{cd_GODGraphGenerator.png}
\label{fig:cd-godgraphgenerator}
\end{figure}

\subsection{Planilhas (GODSpreadsheet)}
\begin{figure}[H]
\centering
\includegraphics[width=\linewidth]{cd_GODSpreadsheet.png}
\label{fig:cd-godspreadsheet}
\end{figure}

\subsection{Processadores (GODProcessors)}
\begin{figure}[H]
\centering
\includegraphics[width=\linewidth]{cd_GODProcessors.png}
\label{fig:cd-godprocessors}
\end{figure}

\subsection{Redes sociais (GODSocialNetIO)}
\begin{figure}[H]
\centering
\includegraphics[width=\linewidth]{cd_GODSocialNetIO.png}
\label{fig:cd-godsocialnetio}
\end{figure}

\subsection{Textos (GODTextIO)}
\begin{figure}[H]
\centering
\includegraphics[width=\linewidth]{cd_GODTextIO.png}
\label{fig:cd-godtextio}
\end{figure}

\subsection{Web (GODWeb)}
% \begin{figure}[H]
% \centering
% \includegraphics[width=\linewidth]{cd_GODWeb.png}
% \label{fig:cd-godweb}
% \end{figure}

\subsection{Agregação de conferências (GODAcademics)}
\begin{figure}[H]
\centering
\includegraphics[width=\linewidth]{cd_GODAcademics.png}
\label{fig:cd-godacademics}
\end{figure}

\subsection{Agregação de informações acadêmicas (GODsCall)}
\begin{figure}[H]
\centering
\includegraphics[width=\linewidth]{cd_GODCall.png}
\label{fig:cd-godcall}
\end{figure}

\subsection{Análise de sentimentos de consumo e político (GODSentimentAnalysis)}
\begin{figure}[H]
\centering
\includegraphics[width=\linewidth]{cd_GODSentimentAnalysis.png}
\label{fig:cd-godsentimentanalysis}
\end{figure}


\newpage
\section{Requisitos não implementados e sugestões de requisitos futuros}

Alguns requisitos previstos para o projeto não foram desenvolvidos. Também foram identificados requisitos que podem ser desenvolvidos no futuro para acrescentar novas funcionalidades ao projeto. Abaixo são listados esses requisitos de acordo com cada módulo.

\subsection{Banco de dados (GODBases)}
\begin{enumerate}
\item Busca por tags de um GODData.
\item Realização de backups do banco de dados.
\end{enumerate}

\subsection{E-mails (GODEmail)}
\begin{enumerate}
\item Envio de arquivos como anexos.
\item Envio de emails em HTML.
\end{enumerate}

\subsection{Filtros (GODFilter)}
\begin{enumerate}
\item Realizar buscas por uma coleção de strings, retornando GODDatas que contenham todas elas.
\item Uso de expressões regulares.
\item Integração com o banco de dados.
\item Filtrar HTML: Criar um método que recebe como parâmetro um arquivo HTML e um XPath e retorna o que encontrar com aquele caminho em todo o HTML.
\item Filtro completo: Criar um método de filtro da classe FILTMainFilter que receberia como parâmetro uma coleção de GODData e uma lista de Strings e retornaria todas as coleções de GODData que contém alguma das Strings em qualquer um dos seus atributos. Ou seja, o filtro buscaria em 'content', 'tags', 'title' e 'origin' ao mesmo tempo, sem precisar especificar em qual campo buscar.
\end{enumerate}

\subsection{Gráficos (GODGraphGenerator)}
\begin{enumerate}
\item Geração de gráfico de pizza.
\item Uso de um pacote de geração de gráfico para melhorar a qualidade visual.
\item Salvar os gráficos como byteArray no Squeak (usando a classe WAFileLibrary do Seaside).
\end{enumerate}

\subsection{Planilhas (GODSpreadsheet)}
\begin{enumerate}
\item Escrita em XLSX.
\item Tratamento de caracteres especiais Os caracteres $\&$, $>$, $<$ e alguns outros tem uma representação diferente em um xml. Portanto, faz-se necessário um tratamento especial para eles. Talvez o método asHtml da classe String resolva.
\end{enumerate}

\subsection{Processadores (GODProcessors)}
\begin{enumerate}
\item Uso do tf-idf como medidor de relevância: É possível utilizar o tf-idf (já implementado) para definir uma medida de relevância para os termos de um documento, isso pode ser interessante para ponderar a análise de um texto.
\item Extender PCSTagger para uma classe de busca de documentos: Criar um método para comparação de consulta e documentos. Esse método precisará de um vetor para cada documento e cada consulta onde as dimensões serão os termos da coleção e o valor será dado pelo tdidf daquele termo no respectivo documento ou consulta. Por fim a similaridade entre documento e consulta é dada pela distância angular dos mesmos, quanto menor a distância mais relevante aquele documento é para a consulta.
\item Stemming ou lemmatization: Na classe PCSPreprocessor criar meios para extrair o radical das palavras. Ex: 'processar' e 'processamento' devem ser tratados pelo mesmo radical, um exemplo pode ser 'processo' isso pode melhorar os resultados do PCSTagger. Esses métodos podem estar vinculados a língua, dessa forma é melhor adicioná-los nas subclasses de PCSPreprocessor.
\item Implementar novas técnicas de análise de recuperação de informação.
\end{enumerate}

\subsection{Redes sociais (GODSocialNetIO)}
\begin{enumerate}
\item Twitter - Pesquisa por mais tweets: O twitter limita a quantidade de tweets retornados em 100. Existem formas de contornar isso, pois ele permite paginar o resultado. Isso é interessante quando o resultado é muito grande, e poderia ser implementado como uma melhoria.
\item FB - Usuário de teste é limitado: É preciso utilizar um usuário ao fazer algumas buscas na APi do facebook. Como utilizamos um usuário de teste (criado no developer.facebook.com), ele tem acesso a poucas informações. Isso tem impacto em buscas de eventos, posts e principalmente de usuários. Talvez seja interessante utilizar um usuário real para essas buscas.
\item Twitter - Limite semanal: O twitter retorna apenas os tweets da última semana. Datas muito antigas não são retornadas. Por isso, talvez seja interessante salvar as buscas mais utilizadas semanalmente em algum banco de dados, para que pudessem ser utilizadas pelas aplicações.
\item FB - busca de usuários: A busca de usuários não retorna dados, provvelmente pois a API exige que um usuário seja autenticado e utilizado nas buscas. Portanto, só são retornados dados que esse usuário tem acesso. Como utilizamos um usuário de teste do app, ele não tem acesso a muitas coisas, e usuários reais é uma delas.
\end{enumerate}

\subsection{Textos (GODTextIO)}
\begin{enumerate}
\item Leitura e escrita em DOCX.
\item Escrita de arquivos PDF sem o uso do \verb|pdflatex|.
\item Leitura de arquivos RTF sem o uso de um \textit{parser} externo (abiword).
\item Leitura de arquivos PDF sem o uso de um \textit{parser} externo (abiword).
\end{enumerate}

\subsection{Web (GODWeb)}
\begin{enumerate}
\item Permitir o posicionamento de componentes web em diferentes lugares de uma página: Inicialmente a ideia era usar um grid do objeto GODData para isso, mas esse grid pode ser colocado na classe WEBPage. A ideia é permitir definir um tamanho e posição para os componentes web dinamicamente.
\item Incluir a string do conteúdo html em um objeto GODData, e retornar esse objeto ao invés da String. Isso é feito na classe WEBPageFetcher.
\item Criar classes para tags html que ainda não tenham sido criadas.
\item Disponibilizar páginas para upload de arquivos.
\item Receber CSSs específicos para cada aplicação: Permite personalizar cada aplicação.
\end{enumerate}

\subsection{Agregação de conferências (GODAcademics)}
\begin{enumerate}
\item Uso de tags sobre as publicações que devem ser apresentadas em uma conferência.
\item Busca mais informações: Usar outras fontes além do Google Scholar para obtenção de informações.
\item Busca fuzzy por journals: Hoje a busca é feita apenas pelos prefixos dos nomes dos journals. Seria interessante que essa busca fosse melhorada. Uma ideia é a busca fuzzy, onde, por exemplo, "plos comp biology" seria encontrado como "plos computational biology". Link: \url{http://en.wikipedia.org/wiki/Approximate_string_matching}.
\end{enumerate}

\subsection{Agregação de informações acadêmicas (GODsCall)}
\begin{enumerate}
\item Crawler incremental: Criar um sistema de crawling incremental, para que possa ser feito aos poucos. A ideia é reduzir o tempo para carregar a aplicação.
\item Pesquisa em novas fontes: Enriquecer mais o banco de dados, buscando conferências em novas fontes, como o \url{http://academic.research.microsoft.com}.
\item Buscar novos tipos de informação: Enriquecer as informações de uma conferência, com informações sobre o local da conferência, transporte, clima... Também informações acadêmicas, como estatísticas sobre os trabalhos apresentados.
\end{enumerate}

\subsection{Análise de sentimentos de consumo e político (GODSentimentAnalysis)}
\begin{enumerate}
\item Melhorar a análise de sentimentos: Atualmente a análise é feita somente pela frequência dos termos positivos e negativos em um texto.
\item Usar o método tf-idf de GODProcessors na análise de sentimentos.
\item Melhorar o layout: Deixar o leiaute mais amigável ao usuário.
\item Mostrar a planilha e o gráfico na mesma busca.
\item Adicionar cores à planilha: Usar cores para diferenciar melhor os rótulos (verde para GOOD, amarelo para NEUTRAL e vermelho para BAD).
\item Adicionar cores ao gráfico: Usar cores para diferenciar melhor os rótulos (verde para GOOD, amarelo para NEUTRAL e vermelho para BAD).
\item Melhorar as técnicas de análise de sentimento: Aprimorar a análise, avaliando relações de uma palavra com a outra.
\item Adicionar autocomplete para os candidatos e partidos: Ao invés de validar se o candidato/partido informado pelo usuário é válido somente após o envio do formulário, implementar uma opção de autocomplete no campo de busca. Assim, o usuário começa a digitar o nome de um partido/candidato e o campo vai buscando na base de dados quais opções casam com os caractéres já informados de forma que o usuário possa selecionar uma opção.
\item Criar opção de selecionar período de tempo: Permitir ao usuário selecionar o período de tempo que este quer realizar a análise. Para tanto, pode-se permitir que este especifique um intervalo de dada, ou selecione períodos pré-determinados, como por exemplo, primeiro turno das eleições do ano X. É provável que seja necessário alterar a forma como é feita a análise e coletar e persistir tweets periodicamente ao invés de buscar tweets no momento da pesquisa.
\item Adicionar técnicas alternativas de análise de sentimentos: Adicionar a opção de técnicas de análise de sentimento que não utilizem dicionários.
\end{enumerate}

\subsection{Sugestões de novos módulos de fontes de dados}
\begin{enumerate}
 \item Módulo para obtenção de dados a partir de web services.
\end{enumerate}



\subsection{Sugestões de novas aplicações}

Diversas aplicações podem ser implementadas usando os recursos fornecidos pelos módulos do GOD. O uso de fontes de emails, web-sites, arquivos texto, planilhas e redes sociais permite criar aplicações para diversos domínios.

\begin{enumerate}
\item Análise de informações sobre dados públicos do governo.
\item Análise de sentimmento de sobre empresas, usando fontes como o site reclameaqui.
\item Análise de dados da bovespa.
\item Criação de corpus com base em artigos de um domínio de interesse. A partir de um conjunto de artigos, que podem ser carregados em um repositório do GOD, pode-se permitir a verificação de frases e termos frequentes usando GODTextIO e GODProcessors.
\end{enumerate}

\newpage
\section{Instalação}

O projeto GOD funciona no Squeak 4.4, que pode ser obtido em \url{http://ftp.squeak.org/4.4/Squeak-4.4-All-in-One.zip}. 
O sistema operacional usado foi o Ubuntu. 

\subsection{Pré-requisitos}

\subsubsection{Instalação do Metacello}


\subsubsection{Instalação do Seaside}


\subsubsection{Instalação do Magma}
Nesta seção são descritos os requisitos para instalação e instruções principais para trabalhar com a biblioteca.

Para o uso do magma, é recomendável a instalação de uma máquina virtual (Virtual Machine - VM) para lidar com problemas de sistema operacional na parte de persistência de informações (uso de imagens). De uma forma geral, a VM proporciona um ganho significativo de performance.


\subsubsection{SqueakSSL-bin}



\subsubsection{Passos de Instalação}

Um modo de instalar é usando o "SqueakMap Catalog" que está em "Apps" na barra de menu do Squeak (ver Fig.~\ref{fig:passo1_InstallMagma1}).

\begin{figure}[!htb]
\centering
\includegraphics[width=0.7\textwidth]{passo1_InstallMagma1.png}
\caption{Instruções para instalação do magma.}
\label{fig:passo1_InstallMagma1}
\end{figure}

Logo, seguir os seguintes passos:
\begin{itemize}
\item{No SqueakMap clique com o botão direito do mouse no quadro esquerdo superior e garantir que todas as opções estejam desmarcadas. Isso permitirá visualizar os pacotes do Magma (Ver Fig.~\ref{fig:passo2_InstallMagma2}).}

\begin{figure}[!htb]
\centering
\includegraphics[width=0.7\textwidth]{passo2_InstallMagma2.png}
\caption{Instruções passo 2.}
\label{fig:passo2_InstallMagma2}
\end{figure}


\item{Clique com o botão direito do mouse no pacote "client" (versão 1.4) do Magma e escolher a opção "install".(Ver Fig.~\ref{fig:passo3_InstallMagmaClient})}

\begin{figure}[!htb]
\centering
\includegraphics[width=0.8\textwidth]{passo3_InstallMagmaClient.png}
\caption{Instruções passo 3.}
\label{fig:passo3_InstallMagmaClient}
\end{figure}


\item{Repetir o ponto 2 para instalar os pacotes  "server" e o "test" do Magma. (Ver Fig.~\ref{fig:passo4_InstallMagmaServer})}

\begin{figure}[!htb]
\centering
\includegraphics[width=0.8\textwidth]{passo4_InstallMagmaServer.png}
\caption{Instruções passo 4.}
\label{fig:passo4_InstallMagmaServer}
\end{figure}

\end{itemize}





\end{document}
