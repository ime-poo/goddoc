\section{Banco de dados (GODBases)}

O módulo de banco de dados orientado a objetos (GODBases) foi elaborado com o intuito de compor o Projeto GOD e cumprir os requisitos da disciplina de Programação Orientada a Objetos. Tem por objetivo fornecer um banco de dados capaz de armazenar informações sobre o projeto GOD, possibilitando operações de inserção, remoção e busca. Seu desenvolvimento foi realizado no Squeak Smalltalk, através do banco de dados orientado a objetos Magma.\\
Contato: Lucy Mansilla (lucyacm@ime.usp.br) e Silvia Scheunemann Silva (silviass@ime.usp.br).


\subsection{Magma}
Magma é uma biblioteca de banco de dados orientado a objetos disponível no Squeak, no modo monousuário (single-user) e também multiusuário (arquitetura cliente-servidor). Os pacotes e métodos disponíveis podem ser obtidos através do download na página:
\begin{itemize}
\item{http://map.squeak.org/packagesbyname.}
\end{itemize}

\subsection{ Classes do GODBases}
O pacote GODBases contém os principais métodos do banco de dados orientado a objetos do projeto GOD. A seguir fazemos uma pequena descrição sobre cada uma de suas classes.


\subsubsection{GDBServerController}

A classe GDBControllerServer permite fazer transações diferentes no servidor do banco de dados, bem como criar e eliminar um repositório. 


\begin{itemize}
\item Para criar um repositório padrão, execute a seguinte linha no workspace:\\
{GDBServerController createRepository.}
\item Para criar um repositório especifico, execute:\\
{GDBServerController createRepository: 'nomeDoRepositorio'.}
\item Para inicializar o servidor com um repositório padrão, execute:\\
{GDBServerController startServer.}
\item Para inicializar o servidor com um repositório específico, execute:\\
{GDBServerController startServer:'nomeDoRepositorio'.}
\item{Para saber se o servidor foi iniciado, execute:}
{GDBServerController isStarted.}\\
esse comando retorna 'true' se o servidor estiver inicializado.
\end{itemize}

Uma vez inicializado o servidor, salve como uma imagem-servidor. Assim, para fechar o servidor, só precisa fechar a imagem salva, e quando você abri-lo novamente, o servidor já estará funcionando.

\begin{itemize}
\item Uma outra opção para fechar o servidor, é executar:\\
{GDBServerController stopServer.}
\item Para apagar um repositório específico, execute:\\
{GDBServerController deleteRepository:'nomeDoRepositorio'}
\end{itemize}


\subsubsection{GDBClientController}

A classe GDBClientController permite fazer a conexão de um usuário/ cliente com o servidor do banco de dados.
Primeiro é necessário inicializar o servidor em uma imagem separada, como indicado anteriormente, em seguida, executar as seguintes instruções:

\begin{itemize}
\item Para conectar um cliente com o banco de dados, execute o seguinte:\\
{GDBClientController initializeSession}
\end{itemize}

Uma vez inicializada a sessão do usuário é possível fazer qualquer transação no banco de dados, ou seja, é possível fazer uso dos métodos contidos nas classes: GDBDatabase, GDBData e GDBConference.


\begin{itemize}
\item Para atualizar qualquer alteração de outro cliente, antes de qualquer transação execute:\\
{GDBClientController refresh.}
\item Para alterar a sessão do cliente, execute o seguinte:\\
{GDBClientController refresh: 'nomeDeOutroCliente'.}
\end{itemize}

Uma vez que não se precise de fazer nenhuma transação a mais no banco de dados, só temos que desconectar a sessão do cliente, para isso execute o seguinte:\\
{GDBClientController release.}



\subsubsection{GDBDatabase}
A classe GDBDatabase permite criar o banco de dados. Nosso banco de dados é representado por um dicionário que armazena duas coleções, uma coleção do tipo MagmaCollection para armazenar os objetos GODData e uma outra do tipo OrderedCollection para armazenar uma coleção que contém objetos GCConference.\\

\begin{itemize}
\item Para criar o banco de dados, execute:\\
{GDBDatabase createDatabase}\\
\item Se você quiser criar apenas o banco de dados para armazenar objetos GODData, execute:\\
{GDBData createGODDataCollection.}\\
\item Se você quiser criar apenas o banco de dados para armazenar a coleção que vai conter objetos GCConference, execute:\\
{GDBConference createConferenceCollection.}
\end{itemize}

\subsubsection{GDBData}
A classe GDBData permite fazer transações diferentes no banco de dados para objetos GODData. 


\begin{itemize}
\item Para inserir um novo objeto GODData no banco de dados, execute:\\
{ GDBData add: meuGODData.}
\item Para remover um objeto GODData do banco de dados, execute:\\
{ GDBData remove: meuGODData.}
\item Para atualizar um objeto GODData por um outro objeto GODData, execute:\\
{ GDBData update: meuAntigoGODData with: meuNovoGODData.}
\item Se você quiser apagar todos os objetos GODData armazenados no banco de dados, execute:\\
{GDBData resetData.}

\item Para buscar um objeto GODData usando parte do nome do autor, execute:\\
{ GDBData searchAuthor: 'parteDoNomeDoAutor'.}
\item Para buscar um objeto GODData usando parte do nome do titulo, execute:\\
{ GDBData searchTitle: 'parteDoNomeDoTitulo'.}
\item Para buscar um objeto GODData usando um bloco, execute:\\
{ GDBData searchFor: umBloco.}\\
onde um bloco deve ter a forma:\\
$[:objeto | objeto atributoDoObjetoGODData = valorDeConparacaoParaABusca]$.

\item Para buscar um objeto GODData usando seu ID, execute:\\
{ GDBData searchById: umIdDoTipoInteiro.}
\item Para fazer uma busca exata pelo nome do autor, execute:\\
{ GDBData exactSearchByAuthor: 'nomeExatoDoAutor'.}
\item Para fazer uma busca exata pelo titulo, execute:\\
{ GDBData exactSearchByTitle: 'nomeExatoDoTitulo'.}	

\end{itemize}


\subsubsection{GDBConference}
A classe GDBConference permite fazer diferentes transações no banco de dados para objetos GCConference. 

\begin{itemize}
\item Para inserir uma nova coleção que armazena objetos GCConference, execute:\\
{GDBConference saveConferences: minhaColeçãoGCConference.}
\item Para recuperar essa coleção armazenada no banco de dados, execute:\\
{GDBConference loadConferences.}
\item Se você quiser apagar essa coleção armazenada no banco de dados, execute:\\
{GDBConference resetConferenceData.}
\end{itemize}
