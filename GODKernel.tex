\def\godkernel{\textbf{GODKernel}}
\def\goddata{\textbf{GODData}}
\def\configGod{\textbf{ConfigurationOfGOD}}

\section{Núcleo (\godkernel)}

\godkernel~ é o módulo que contém a classe que representa a estrutura básica para manipulação de dados do projeto.
Essa classe é chamada \goddata~ contém os atributos para armazenamento comuns aos diferentes tipos de fontes de dados usados no projeto. 
\goddata~ pode ser estendido para atender às especificidades dessas diferentes fontes de informações.\\
Esse módulo também é responsável por gerenciar o repositório do projeto, realizando o controle de versões e suas dependências.\\
Em caso de dúvidas, entre em contato com \emph{Higor: hamario [at] ime.usp.br}.


\subsection{\goddata} 

Classe que representa a estrutura básica a serem armazenadas das diferentes fontes de informação, comoarquivos texto, planilhas, páginas html, redes sociais,e-mails, entre outras.

\subsubsection{Atributos}

\begin{itemize}
 \item \textbf{author} - autor de um conteúdo.
 \item \textbf{content} - armazena o conteúdo principal de uma fonte de informação.
 \item \textbf{height} - altura do objeto a ser renderizado.
 \item \textbf{id} - valor exclusivo do objeto.
 \item \textbf{layoutGrid} - matriz que contém objetos \goddata~ internos. Atributo que representa a posição na qual os objetos internos devem ser renderizados em uma página web.
 \item \textbf{origin} - origem da informação.
 \item \textbf{tags} - termos principais associados ao objeto \goddata.
 \item \textbf{timestamp} - data do \goddata.
 \item \textbf{title} - Titulo da informação.
 \item \textbf{width} - largura do objeto a ser renderizado.
\end{itemize}


\subsubsection{Métodos}

\begin{itemize}
 \item \textbf{addOnLayoutGrid:a\goddata~ at: aRow at: aColumn} - adiciona um \goddata~ ao layoutGrid.
 \item \textbf{addTag:} - adiciona uma nova tag à coleção de tags.
 \item \textbf{is\goddata} - verifica se um objeto é do tipo \goddata.
 \item \textbf{isWho} - informa o tipo de subclasse de \goddata.
 \item \textbf{layoutGridRowSize: rowNumber columnSize: columnNumber} - define as dimensões do layoutGrid.
 \item \textbf{tagExists:} - verifica se uma determinada tag existe.
\end{itemize}


\subsubsection{Exemplos}

\begin{godCode}
'Preenchendo um \goddata'
goddata := \godData~ new.
goddata title:'titulo do objeto'.
goddata author:'proprietario da informacao'
goddata content: 'Conteudo que faz parte da informacao'
goddata addTag: 'termo'.

'Definindo o tamanho do layoutGrid e atribuindo objetos \goddata~ a este grid'
goddata {layoutGridRowSize: 5 columnSize: 3.
goddata addOnLayoutGrid: godddata1 at: 1 at: 1.
goddata addOnLayoutGrid: godddata1 at: 3 at: 3.

'Adicionando uma tag a colecao'
goddata addTag: 'termo'.

\end{godCode}


\subsection{\configGod} 

Classe responsável pelo gerenciamento dos módulos do projeto. Possui módulos para fazer o download de versões dos módulos e dependências do projeto, assim como a realização de backup do repositório.

\subsubsection{Métodos}

\begin{itemize}
 \item \textbf{backupPackages:} - realiza o backup dos módulos do projeto.
 \item \textbf{baseline01:} -  inclui as descrições dos pacotes e dependências do projeto.
\end{itemize}

\subsubsection{Exemplos}

\begin{godCode}
'Baixando a ultima versao dos pacotes do projeto'
(\configGod~ project version: '0.1-baseline01') load.
\end{godCode}


