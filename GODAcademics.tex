\section{Agregador de informações acadêmicas (GODAcademics)}

\textbf{Grupo:}\textit{Ígor Bonadio, Renato Cordeiro, Ruan Costa}

\textbf{Contato:}ibonadio@ime.usp.br\\

GOD Academics é um agregador de informações acadêmicas. A partir de um perfil do Google Scholar, esta aplicação constroi um relatório resumindo as informações obtidas de diversas fontes.

Atualmente GOD Academics apenas utiliza como fonte de informação o Google Scholar e o CAPES-Qualis.

\subsection{Classe ACADPaper}

Representa um artigo de um pesquisador.

\subsubsection{Atributos de instância}

\begin{itemize}
  \item name: nome do artigo.

  \item coauthors: string contendo os nomes de todos os autores.

  \item year: ano em que o artigo foi publicado.

  \item journal: é o nome do periódico onde o artigo foi publicado.

  \item impactFactor: é o fator de impacto (estrato) do journal no qual o artigo foi publicado. Essa é uma  medida do sistema webQualis da capes.

  \item NumberOfCitatioins: é o número de citacões do artigo.
\end{itemize}

\subsubsection{Métodos de instância}

\begin{itemize}

    \item hasAttribute: aTag in: aHtmlPaper - Verifica se a informação representada por aTag existe em aHtmlPaper
 
   \item initializeFromHTML: aHtmlPaper - Monta um paper a partir de uma string contendo código html. Não deve ser usado diretamente, veja Researcher>initializeFromProfileURL.
  
  \item loadAttribute: aTag from: aHtmlPaper - Retorna o conteúdo da aTag a partir de um html.
  
  \item loadAttributesFrom: aHtmlPaper - Carrega os atributos de um paper a partir de um html.
  
  \item loadImpactFactorFrom: journal - Carrega o fator de impacto do journal passado
  
  \item numberOfCitations - Retorna o número de citações do artigo.

\end{itemize}

\subsection{Classe ACADQualis}

Representa a lista qualis de periódicos x fator de impacto. É um singleton. Não deve ser instanciado com new, mas sim com o método de classe singleton.

\subsubsection{Atributos de classe}

\begin{itemize}
  \item uniqueInstance: guarda a única instância da classe.
\end{itemize}

\subsubsection{Atributos de instância}

\begin{itemize}
  \item qualis: é um dicionário. As chaves são os periódicos e o valor é o fator de impacto do periódico.
\end{itemize}

\subsubsection{Métodos de classe}

\begin{itemize}

  \item clear - Limpa o atributo de classe que guarda a instancia da classe. Faz isso atribuindo null.
  \item singleton - Cria uma instância da classe e a retorna.

\end{itemize}

\subsubsection{Métodos de instância}

\begin{itemize}

  \item initialize - Carrega o hash qualis com a lista armazenada em listOfQualis

  \item journalIsSomethingLike: aJournal - Dado o nome de um periódico, procura na no hash um periódico com nome parecido. Retorna a primeira chave que dê match em um dos prefixos do nome do periódico procurado. Se não houver match, é retornado \'unknown\'.

  \item listOfQualis - String com a última lista lançada pela capes, com o nome dos periódicos e seus  respectivos fatores de impacto
  
  \item load: aQualisText - Carrega um texto qualis (que esteja no mesmo formato de listOfQualis) no dicionário  qualis.
  
  \item qualisOf: aJournal - Dado o nome de um periódico, retorna seu fator de impacto. Se não existir no dicionário, unknown é retornado.

\end{itemize}

\subsection{Classe ACADResearcher}

Representa um pesquisador.

\subsubsection{Atributos de instância}

\begin{itemize}
  \item name: nome do pesquisador

  \item job: profissão do pesquisador

  \item papers: lista de artigos

  \item valid: variável booleana que indica se este é um objeto bem formado ou não.

\end{itemize}

\subsubsection{Métodos de instância}

\begin{itemize}
  \item barGraphOfPapersByImpact - Retorna um gráfico com os fatores de impacto no eixo x, e a quantidade de artigos com um  dado fator de impacto no eixo y.

  \item forEachPaperDo: aBlock - Executa o bloco passado em toda a lista de artigos.

  \item hasAttribute: aTag in: aHtmlPage - Verifica se a informação representada por aTag existe em aHtmlPage.

  \item initialize - Aloca memória para a lista de artigos.

  \item intitializeFromHTMLPage: aHtmlPage - Monta um pesquisador extraindo as informações de aHtmlPage.

  \item initializeFromProfileURL: aProfileURL - Similar à initializeFromHTMLPage, mas pega a página passada em aProfileURL. Esta URL tem de ser um perfil do Google Scholar. Utilizamos um proxy para que o google scholar não bloqueie nosso acesso, pois nosso programa age como um robô

  \item loadAttribute: aTag from: aHtmlPage - Se a informação representada por aTag existir em aHtmlPage, retornamos ela. Caso contrário, retornamos um GODData vazio.

  \item loadAttributesFrom: aHtmlPage - Carrega os atributos do pesquisador a partir de aHtmlPage.

  \item LoadPapersFrom: aHtmlPage - Carrega os artigos do pesquisador a partir de aHtmlPage.

  \item papersByImpactDictionary - Retorna um dicionário, onde as chaves são todos os possíveis valores de  fatores de impacto e o valor é a quantidade de artigos que o pesquisador tem com cada fator de impacto.

  \item totalNumberOfCitations - Retorna o número total de citações ao pesquisador, isto é, a soma das citações de todos os  seus artigos.

\end{itemize}

\subsection{Classe ACADTag}

Representa uma o início e o fim de uma TAG. Entre essas marcas encontra-se o conteúdo necessário para carregar as classes ACADResearchers e ACADPapers.

\subsubsection{Atributos de instância}

\begin{itemize}
  \item star: marca de início

  \item end: marca de fim

\end{itemize}

\subsection{Classe ACADTagPool}

Representa a coleção de ACADTags necessárias para a aplicação GODAcademics. É um singleton. Não deve ser instanciado com new, mas sim com o método de classe singleton.

\subsubsection{Atributos de classe}

\begin{itemize}
  \item uniqueInstance: guarda a única instância da classe.
\end{itemize}

\subsubsection{Atributos de instância}

\begin{itemize}
  \item pool: um dicionário de ACADTags.

  \item filter: um FILTTextFilter

\end{itemize}

\subsubsection{Métodos de classe}

\begin{itemize}

  \item clear - Limpa o atributo de classe que guarda a instancia da classe. Faz isso atribuindo null.

  \item singleton - Cria uma instância da classe e a retorna.

\end{itemize}

\subsubsection{Métodos de instância}

\begin{itemize}
  \item get: aTagName - Retorna um ACADTag cujo nome é aTagName.

  \item getAttributes: aTagName from: aHTML - Utilizando a tag aTagName, retorna o valor encontrado em aHTML.

  \item initialize - Inicializa a instância.

  \item initializePaperTags - Adiciona ao dicionário pool todas as tags de artigos

  \item initializeResearcherTags - Adiciona ao dicionário pool todas as tags de pesquisadores.

\end{itemize}

\subsection{Classe ACADAcademicsHome}

Componente utilizado pelo Seaside para renderizar a página inicial da aplicação.

\subsection{Classe ACADAcademicsProfile}

Componente utilizado pelo Seaside para renderizar a página com o relatório final do pesquisador escolhido.

\subsection{Classe ACADAcademics}

Responsável por controlar o fluxo da aplicação GODAcademics.

\subsubsection{Atributos de instância}

\begin{itemize}
  \item component: um ACADAcademicsHome ou ACADAcademicsProfile

  \item input: um textfield que contem a URL do perfil do Google Scholar de um pesquisador

  \item profileUrl: URL do perfil do Google Scholar de um pesquisador

\end{itemize}

\subsubsection{Métodos de instância}

\begin{itemize}
  \item initialize: aComponent - Inicializa a instância.

  \item renderHome - Renderiza a página inicial

  \item renderProfile: researcher - Constrói o relatório do pesquisador

  \item save - Captura o valor atribuido ao formulário da página inicial.

\end{itemize}
