\section{GODAcademics}

\textbf{Grupo:}\textit{Ígor Bonadio, Renato Cordeiro, Ruan Costa}

\textbf{Contato:}ibonadio@ime.usp.br\\

GOD Academics é um agregador de informações acadêmicas. A partir de um perfil do Google Scholar, esta aplicação constroi um relatório resumindo as informações obtidas de diversas fontes.

Atualmente GOD Academics apenas utiliza como fonte de informação o Google Scholar e o CAPES-Qualis.

\subsection{Classe ACADPaper}

Representa um artigo de um pesquisador.

\subsubsection{Atributos de instância}

\begin{itemize}
  \item name: nome do artigo.

  \item coauthors: string contendo os nomes de todos os autores.

  \item year: ano em que o artigo foi publicado.

  \item journal: é o nome do periódico onde o artigo foi publicado.

  \item impactFactor: é o fator de impacto (estrato) do journal no qual o arrigo foi publicado. Essa é uma    medida do sistema webQualis, da capes.

  \item NumberOfCitatioins: é o número de citacões que o artigo tem.
\end{itemize}

\subsubsection{Métodos de instância}

\begin{itemize}

  \item hasAttribute: aTag in: aHtmlPaper

  Verifica se a informaçãoi representada por aTag, existe em aHtmlPaper

  \item initializeFromHTML: aHtmlPaper

  Monta um paper a partir de uma string contento html. Não deve ser usado diretamente, veja   Researcher>initializeFromProfileURL.

  \item LoadAttribute: aTag from: aHtmlPaper

  Dada uma tag, indicando qual informação está sendo pedido, e um pedaço de html, retorna a   informação pedida.

  \item loadAttributesFrom: aHtmlPaper

  Carrega os atributos de um paper a partir de um html.

  \item loadImpactFactorFrom: journal

  Acha o fator de impacto do journal passado

  \item numberOfCitations

  Retorna o número de citações do artigo.

\end{itemize}

\subsection{Classe ACADQualis}

Representa a lista qualis de periódicos x fator de impacto.

É um singleton. Não deve ser instanciado com new, mas sim com o método de classe singleton.

\subsubsection{Atributos de classe}

\begin{itemize}
  \item uniqueInstance: guarda a única instância da classe.
\end{itemize}

\subsubsection{Atributos de instância}

\begin{itemize}
  \item qualis: é um dicionário. As chaves são os periódicos e o valor é o fator de impacto do periódico.
\end{itemize}

\subsubsection{Métodos de classe}

\begin{itemize}

  \item clear

  Limpa o atributo de classe que guarda a instancia da classe. Faz isso atribuindo null.

  \item singleton

  Cria uma instância da classe e a retorna.

\end{itemize}

\subsubsection{Métodos de instância}

\begin{itemize}

  \item initialize

  Carrega o hash qualis com a lista armazenada em listOfQualis

  \item journalIsSomethingLike: aJournal

  Dado o nome de um periódico, procura na no hash um periódico com nome parecido.
  Retorna a primeira chave que dê match em um dos prefixos do nome do periódico   procurado. Se não houver match, é retornado \'unknown\'.

  \item listOfQualis

  String gigante com a última lista lançada pela capes, com o nome dos periódicos e seus  respectivos fatores de impacto

  \item load: aQualisText

  Carrega um  texto qualis (que esteja no mesmo formato de listOfQualis) no dicionário  qualis.

  \item qualisOf: aJournal

  Dado o nome de um periódico, retorna seu fator de impacto.

  Se não existir no dicionário, unknown é retornado.

\end{itemize}