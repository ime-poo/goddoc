\section{Instalação}

O projeto GOD funciona no Squeak 4.4, que pode ser obtido em \url{http://ftp.squeak.org/4.4/Squeak-4.4-All-in-One.zip}. 
O sistema operacional usado foi o Ubuntu. 

\subsection{Pré-requisitos}

\subsubsection{Instalação do Metacello}


\subsubsection{Instalação do Seaside}


\subsubsection{Instalação do Magma}
Nesta seção são descritos os requisitos para instalação e instruções principais para trabalhar com a biblioteca.

Para o uso do magma, é recomendável a instalação de uma máquina virtual (Virtual Machine - VM) para lidar com problemas de sistema operacional na parte de persistência de informações (uso de imagens). De uma forma geral, a VM proporciona um ganho significativo de performance.


\subsubsection{SqueakSSL-bin}



\subsubsection{Passos de Instalação}

Um modo de instalar é usando o "SqueakMap Catalog" que está em "Apps" na barra de menu do Squeak (ver Fig.~\ref{fig:passo1_InstallMagma1}).

\begin{figure}[!htb]
\centering
\includegraphics[width=0.7\textwidth]{passo1_InstallMagma1.png}
\caption{Instruções para instalação do magma.}
\label{fig:passo1_InstallMagma1}
\end{figure}

Logo, seguir os seguintes passos:
\begin{itemize}
\item{No SqueakMap clique com o botão direito do mouse no quadro esquerdo superior e garantir que todas as opções estejam desmarcadas. Isso permitirá visualizar os pacotes do Magma (Ver Fig.~\ref{fig:passo2_InstallMagma2}).}

\begin{figure}[!htb]
\centering
\includegraphics[width=0.7\textwidth]{passo2_InstallMagma2.png}
\caption{Instruções passo 2.}
\label{fig:passo2_InstallMagma2}
\end{figure}


\item{Clique com o botão direito do mouse no pacote "client" (versão 1.4) do Magma e escolher a opção "install".(Ver Fig.~\ref{fig:passo3_InstallMagmaClient})}

\begin{figure}[!htb]
\centering
\includegraphics[width=0.8\textwidth]{passo3_InstallMagmaClient.png}
\caption{Instruções passo 3.}
\label{fig:passo3_InstallMagmaClient}
\end{figure}


\item{Repetir o ponto 2 para instalar os pacotes  "server" e o "test" do Magma. (Ver Fig.~\ref{fig:passo4_InstallMagmaServer})}

\begin{figure}[!htb]
\centering
\includegraphics[width=0.8\textwidth]{passo4_InstallMagmaServer.png}
\caption{Instruções passo 4.}
\label{fig:passo4_InstallMagmaServer}
\end{figure}

\end{itemize}



