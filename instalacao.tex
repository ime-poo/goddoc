\section{Instalação}

O projeto GOD é compatível com o Squeak 4.4, que pode ser obtido em \url{http://ftp.squeak.org/4.4/Squeak-4.4-All-in-One.zip}. 
O GOD roda sobre o sistema operacional Ubuntu. Isso ocorre devido a algumas dependências externas à máquina virtual do Squeak. Mais detalhes sobre a instalação do projeto serão detalhadas a seguir.


\subsection{Pré-requisitos}

\subsubsection{Instalação do Metacello}

O pacote metacello é usado para gerenciar o controle de versões do projeto GOD. Para instalá-lo digite os seguintes comandos no workspace:
\begin{godCode}
Installer squeaksource
    project: 'MetacelloRepository';
    install: 'ConfigurationOfMetacello'. 
(Smalltalk at: #ConfigurationOfMetacello) perform: #load.
\end{godCode}

\subsubsection{Instalação do Seaside}

O Seaside é o framework web usado para construir a interface visual do GOD. Para instalá-lo, acesse o menu \textbf{Apps>>SqueakMap Catalog}. Em seguida digite ``seaside'' em Search Packages. Selecione a versão 3.0 e clique em Install.\\
Outra opção é fazer o download pelo Metacello. Basta executar os comandos abaixo em um workspace do Squeak.
\begin{godCode}
Installer squeaksource
 project: 'MetacelloRepository';
 install: 'ConfigurationOfSeaside30'.
 (Smalltalk at: #ConfigurationOfSeaside30) load.
 \end{godCode}
Execute os comandos (``do it'') e aguarde a finalização do processo. Para conferir se a instalação foi feita com sucesso, clique em Apps veja se aparece a opção Seaside Control Panel.\\
O Seaside vem com o servidor web chamado \textbf{comanche}. Para ativar o comanche, abra \textbf{Apps>>Seaside Control Panel}, clique com o botão direito do mouse em \textbf{Add adaptor}, selecione a porta e clique em \textbf{Start}.

\subsubsection{Instalação do Magma}
O Magma é o banco de dados orientado a objetos usado pelo GOD. Para instalar o Magma, acesse o menu \textbf{Apps>>SqueakMap Catalog}. Em seguida digite ``magma'' em Search Packages. Selecione a versão 1.4 e clique em \textbf{Install} (ver Fig.~\ref{fig:passo1_InstallMagma1} e Fig.~\ref{fig:passo2_InstallMagma2}).

\begin{figure}[!htb]
\centering
\includegraphics[width=0.7\textwidth]{passo1_InstallMagma1.png}
\caption{Instalação do magma}
\label{fig:passo1_InstallMagma1}
\end{figure}

\begin{figure}[!htb]
\centering
\includegraphics[width=0.7\textwidth]{passo2_InstallMagma2.png}
\caption{Instalação do magma (2)}
\label{fig:passo2_InstallMagma2}
\end{figure}

A seguir, selecione o pacote \textbf{client} e escolha a opção \textbf{install}.(Ver Fig.~\ref{fig:passo3_InstallMagmaClient}). Repetir o passo anterior para instalar os pacotes  \textbf{server} e o \textbf{test}. (Ver Fig.~\ref{fig:passo4_InstallMagmaServer})

\begin{figure}[!htb]
\centering
\includegraphics[width=0.8\textwidth]{passo3_InstallMagmaClient.png}
\caption{Instalação do magma (3)}
\label{fig:passo3_InstallMagmaClient}
\end{figure}

\begin{figure}[!htb]
\centering
\includegraphics[width=0.8\textwidth]{passo4_InstallMagmaServer.png}
\caption{Instalação do magma (4)}
\label{fig:passo4_InstallMagmaServer}
\end{figure}


\subsubsection{SqueakSSL-bin}
O SqueakSSL é um conjunto de pacotes usados para acessar conteúdo HTTPS. Alguns módulos do GOD usam o SqueakSSL: GODEmail, GODSocialNetIO e GODWeb.
Esse pacote é instalado junto com os demais módulos do GOD. No entanto, é necessário instalar um arquivo binário do SqueakSSL antes de fazer a instalação do GOD.
Baixe o arquivo em \url{https://squeakssl.googlecode.com/files/SqueakSSL-bin-0.1.5.zip}. A seguir, copie o arquivo \textbf{so.SqueakSSL} dentro dos diretórios do Squeak em 
\textbf{Contents/Linux-i686/lib/squeak/4.4.7-2357}. \\
Existem também os arquivos binários para OS X e Windows. Porém, nos testes que fizemos com o OS X o procedimento não funcionou.

\subsubsection{pdflatex e abiword}

Para realizar a leitura e escrita dos formatos de texto PDF e RTF é necessário ter instalado os programas pdflatex e abiword (em sua versão 2.9.2).  Ambos programas devem 
estar funcionando por linha de comando. O pacote \textbf{OSProcess} faz a chamada desses programas pelo Squeak para realizar as conversões. Infelizmente não existe uma 
versão do abiword compatível com as versões atuais do OS X.\\

O Abiword pode ser obtido em \url{http://www.abisource.com/downloads/abiword/2.9.2/source/abiword-2.9.2.tar.gz} e o pdflatex acompanha as principais distribuições latex.

\subsubsection{Repositórios do GOD}

O repositório do GOD fica no SmalltalkHub \url{http://smalltalkhub.com/mc/higoramario/GOD/main}. 
Há também um repositório usado para backup do projeto no SqueakSource3 (\url{http://ss3.gemstone.com/ss/GOD/}).

\subsubsection{Instalação do GOD}

Para instalar o GOD, inclua o repositório do projeto no Monticello. Clique em \textbf{Tools>>Monticello Browser>>+Repository>>HTTP} e digite a url. 
Selecione a url e clique em \textbf{open}. Na janela seguinte, selecione o pacote \textbf{GODKernel} do lado esquerdo e selecione o pacote mais atual na janela do lado direito.
Clique em \textbf{load}. \\
Para obter todos os pacotes mais atuais do projeto, digite os seguintes comandos no \textbf{workspace}:\\

\begin{godCode}
(ConfigurationOfGOD project version: '0.1-baseline') load.
\end{godCode}


As dependências externas necessárias para o funcionamento do GOD estão listadas abaixo. Todas elas são incluídas no download dos pacotes do GOD.
\begin{itemize}
 \item JSON
 \item OSProcess (Tests-OSProcess)
 \item SqueakSSL (SqueakSSL-Core, SqueakSSL-SMTP, SqueakSSL-Tests)
 \item VB-Regex
 \item WebClient (WebClient-Core)
\end{itemize}

\subsubsection{Inicialização do GOD}

Inicialize o servidor web do seaside. Em seguida, para iniciar o GOD digite no workspace o comando:

\begin{godCode}
WEBGodHome initialize.
\end{godCode}

Todas as aplicações e o GOD serão registrados no Seaside e estarão disponíveis para uso.

\subsubsection{Problemas identificados}

\begin{description}
 \item[Versões do Squeak:] A versão 4.5 do Squeak apresentou problema com o uso do Magma. Por esse motivo optamos por usar a versão 4.4 do Squeak. 
 Por sua vez, tivemos problemas com o uso do SqueakSSL na versão 4.4 (como descrito anteriormente), o que limita o uso do GOD ao Linux, no qual esse 
 problema de compatibilidade pode ser resolvido.
 \item[Abiword:] A versão 2.9.2 do abiword deve ser usada no projeto, já que a versão dos repositóprios oficiais do Ubuntu (3.0.0) não funciona via 
 linha de comando. Além disso, não existe versão compatível do programa para OS X em suas versões mais atuais. Esse é outro fator que limita o uso do GOD 
 ao Linux.
 \item[GODAcademics x Google Scholar:] O Google Scholar é usado pelo módulo GODAcademics como fonte de informação para os dados processados pelo módulo. 
 Em algumas tentativas de acesso o Google reconheceu a aplicação como um robô, impedindo a obtenção dos dados.
 \item[GODSocialNetIO:] As APIs do Facebbok e do Twitter possuem algumas limitações quanto às consultas que podem ser feitas. Há restrições de tempo e quantidade 
 de posts/tweets que podem ser acessados. No facebook só é possível visualizar dados dentro do escopo do usuário que é usado para o acesso à api. Além disso, 
 há rumores dizendo que a API do Twitter será cancelada.
 \item[GODsCall:] Durante o final da fase de desenvolvimento do projeto, o ConfSearch, uma das fontes de informação do módulo, ficou indisponível, limitando as 
 informações do GODsCall aos dados do WikiCFP.
\end{description}


